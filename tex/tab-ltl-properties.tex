\documentclass[../main.tex]{subfiles}%
\begin{document}%
\noindent%
\begin{tabularx}{\textwidth}{ r X }
\toprule
    & \thead{LTL-свойство и его описание}
\\\midrule
    \multicolumn{2}{c}{Фиксированные свойства}
\\\midrule
    $\Prop{1}$ & $\LTLProp{1}$ \\
    & Цилиндр~I не должен одновременно сжиматься и расширяться.
\\\cmidrule(r){2-2}
    $\Prop{2}$ & $\LTLProp{2}$ \\
    & Аналогичное свойство для цилиндра~II.
\\\cmidrule(r){2-2}
    $\Prop{3}$ & $\LTLProp{3}$ \\
    & Аналогичное свойство для вакуумной присоски~IV.
\\\cmidrule(r){2-2}
    $\Prop{4}$ & $\LTLProp{4}$ \\
    & Пока вертикальный цилиндр~III находится в промежуточном положении, горизонтальный цилиндр~I должен оставаться в своем крайнем положении.
\\\cmidrule(r){2-2}
    $\Prop{5}$ & $\LTLProp{5}$ \\
    & Пока горизонтальный цилиндр~I находится в промежуточном положении, вертикальный цилиндр~III должен оставаться в своем крайнем положении.
\\\cmidrule(r){2-2}
    $\Prop{6}$ & $\LTLProp{6}$ \\
    & Если все цилиндры манипулятора находятся в своих начальных положениях, то, пока нет необходимости в обработке (переносе) рабочей детали, не должно возникать команд управления для начала движения.
\\\cmidrule(r){2-2}
    $\Prop{7}$ & $\LTLProp{7}$ \\
    & Если рабочая деталь была поднята со входного лотка, то она должна быть когда-нибудь опущена в выходной лоток~V.
\\\midrule
    \multicolumn{2}{c}{Вариабельные свойства}
\\\midrule
    $\PropWP{1}$ & $\LTLPropWP{1}$ \\
    & Рабочая деталь во входном лотке~1 должна быть обработана в будущем.
\\\cmidrule(r){2-2}
    $\PropWP{2}$ & $\LTLPropWP{2}$ \\
    & Рабочая деталь во входном лотке~2 должна быть обработана в будущем.
\\\cmidrule(r){2-2}
    $\PropWP{3}$ & $\LTLPropWP{3}$ \\
    & Рабочая деталь во входном лотке~3 должна быть обработана в будущем.
\\\bottomrule
\end{tabularx}%
\end{document}
