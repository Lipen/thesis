\chapter*{ЗАКЛЮЧЕНИЕ}
\addcontentsline{toc}{chapter}{Заключение}

В данной диссертационной работе была достигнута поставленная цель \--- повышение эффективности полных алгоритмов решения задачи булевой выполнимости (SAT) применительно к синтезу и верификации моделей автоматных программ.
Разработаны оригинальные методы и техники декомпозиции булевых формул, что позволило существенно сократить время работы алгоритмов.

В ходе исследования созданы новые алгоритмы кодирования задач синтеза конечных автоматов и булевых схем, отличающиеся явным кодированием структуры охранных условий в виде деревьев разбора формул.
Предложены методы декомпозиции булевых формул, обеспечивающие точную оценку декомпозиционной трудности задач благодаря низкой дисперсии времени решения подзадач.
Масштабные вычислительные эксперименты подтвердили практическую значимость предложенных методов.
Разработанные методы успешно применены для решения сложных примеров синтеза и верификации моделей автоматных программ, включая конечные автоматы и логические схемы.

Созданная программная библиотека \texttt{kotlin-satlib} предоставляет унифицированный интерфейс для работы с современными SAT-решателями, облегчая процесс моделирования и решения задач синтеза и верификации. Библиотека включает модули для интеграции с SAT-решателей через технологию JNI, упрощения записи ограничений с использованием преобразований Цейтина, манипуляции переменными с конечными доменами и работы с многомерными массивами SAT-переменных.

Результаты исследования демонстрируют, что предложенные методы и алгоритмы могут быть эффективно интегрированы в существующие системы синтеза и верификации моделей автоматных программ.
Они также могут быть адаптированы для решения других задач, связанных с булевой выполнимостью, что открывает возможности для дальнейших исследований и разработок.

Перспективы дальнейшего развития темы включают углубленное исследование методов декомпозиции булевых формул, разработку новых алгоритмов для специфических классов задач и расширение функциональности программной библиотеки для поддержки большего числа SAT-решателей и разнообразных типов задач.
Таким образом, проделанная работа вносит значительный вклад в область синтеза и верификации моделей автоматных программ, предлагая более эффективные и гибкие инструменты для решения сложных задач.
