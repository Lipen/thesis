\chapter*{ЗАКЛЮЧЕНИЕ}
\addcontentsline{toc}{chapter}{Заключение}

В данной диссертационной работе была достигнута поставленная цель \--- повышение эффективности полных алгоритмов решения задачи булевой выполнимости (SAT) применительно к синтезу и верификации моделей автоматных программ.
Для~этого были разработаны оригинальные методы и техники, которые существенно сокращают время работы алгоритмов.

В~рамках исследования были разработаны методы кодирования в SAT задач синтеза конечных автоматов с заданным поведением и свойствами.
Эти методы включают кодирование структуры охранных условий в виде деревьев разбора соответствующих формул, что отличает их от существующих решений и повышает адаптивность и эффективность.
Также были созданы методы кодирования в SAT задач синтеза модульных конечных автоматов, включающие автоматизированное модульное разбиение, что дополнительно улучшает их адаптивность и эффективность.

Особое внимание было уделено разработке методов кодирования в SAT задач синтеза булевых схем и булевых формул по заданной таблице истинности.
В~отличие от существующих подходов, новые методы позволяют использовать произвольные элементарные гейты, что значительно расширяет область их применения.
Дополнительно были разработаны методы декомпозиции булевых формул, кодирующих задачи синтеза конечных автоматов и верификации булевых схем.
Эти методы позволяют строить оценки декомпозиционной трудности, что улучшает прогнозируемость времени работы SAT-решателей.

Важным достижением работы является создание программной библиотеки \texttt{kotlin-satlib}, которая обеспечивает взаимодействие с SAT-решателями через унифицированный программный интерфейс.
Библиотека предоставляет широкий выбор различных SAT-решателей, контроль за различными этапами построения SAT-кодировок и возможность манипуляции переменными с произвольными конечными доменами.
Также был разработан программный комплекс \smallcaps{fbSAT} для синтеза и верификации конечных автоматов с использованием SAT-решателей, который интегрирует все разработанные методы и алгоритмы.

Для подтверждения практической эффективности всех разработанных методов были проведены масштабные вычислительные эксперименты.
Результаты экспериментов демонстрируют, что предложенные методы и алгоритмы успешно применяются для решения сложных задач синтеза и верификации моделей автоматных программ, включая конечные автоматы и логические схемы.

Таким образом, проделанная работа вносит значительный вклад в область синтеза и верификации моделей автоматных программ, предлагая более эффективные и гибкие инструменты для решения сложных задач.
Результаты и методы, представленные в данной диссертации, могут быть адаптированы для решения других задач, связанных с булевой выполнимостью, что открывает новые возможности для дальнейших исследований и разработок в этой области.
Перспективы дальнейшего развития темы включают углубленное исследование методов декомпозиции булевых формул, разработку новых алгоритмов для специфических классов задач и расширение функциональности программной библиотеки для поддержки большего числа SAT-решателей, разнообразных типов задач и способов их моделирования.
