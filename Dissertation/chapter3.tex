\chapter{Методы оценивания декомпозиционной трудности булевых формул в применении к задачам тестирования и верификации логических схем}
\label{ch:decompositions}

ICCAD/SAT/IEEE article

\section{Общие стратегии декомпозиции булевых формул, кодирующих задачи верификации (проверки эквивалентности) логических схем}

Основной вывод: нужно объединять "входы" схем, чтобы уменьшить размерность пространства поиска, а также чтобы получить меньшую дисперсию подзадач.

\section{Конструкции разбиения формул, кодирующих задачи верификации логических схем}

Чанки, балансные переменные, интервалы, математические свойства, детали программной реализации ("простая" реализация, BDD).

\section{Вероятностный и статистический анализ свойств предложенных разбиений}

Статистические оценки, результаты из IEEE, доверительные интервалы, обоснование мощности выборки.

\section{Экспериментальное исследование}

Вычислительные эксперименты и их результаты.

Multipliers, Sorters, Crypto.
