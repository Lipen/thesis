\chapter{Методы оценивания декомпозиционной трудности булевых формул в применении к задачам тестирования и верификации логических схем}
\label{ch:decompositions}

ICCAD/SAT/IEEE article

\section{Общие стратегии декомпозиции булевых формул, кодирующих задачи верификации (проверки эквивалентности) логических схем}

\todo{Основной вывод: нужно объединять "входы" схем, чтобы уменьшить размерность пространства поиска, а также чтобы получить меньшую дисперсию подзадач.}

Для эффективного решения сложных экземпляров SAT часто разумно использовать некоторые техники разделения исходной проблемы на более простые.
% TODO: fix "partitioning" translation
Естественным способом декомпозиции экземпляра SAT на подзадачи является так называемый \textit{подход к разбиению} (\textit{partitioning approach})~\cite{hyvarinen2011}.

\todo{Описать различные стратегии разбиения: по переменным (входы / (не-)балансные гейты), по кубам (CnC?), по чанкам (объединение входов), по интервалам.}

\section{Конструкции разбиения формул, кодирующих задачи верификации логических схем}

Рассмотрим произвольную КНФ-формулу~$C$ над множеством булевых переменных~$X$ и множество $\Pi = \{G_1, \dots, G_s\}$, где $G_i$, $i \in\nobreak \{1, \dots, s\}$, представляют собой некоторые булевы формулы над~$X$.
Пусть $\Pi$ задает \textit{разбиение SAT~$C$}, если выполняются следующие условия:
\begin{enumerate}
    \item формулы $C$ и $C \land (G_1 \lor \dots \lor G_s)$ равновыполнимы (\textit{equisatisfiable});
    \item для всех $i \neg j \in \{1, \dots, s\}$ формула $C \land G_i \land G_j$ выполнима.
\end{enumerate}

Здесь и далее будем называть конъюнкцию произвольных литералов (без эквивалентных и контрарных литералов) из~$X$ как \textit{куб} над~$X$.
Для произвольной КНФ~$C$ над переменными~$X$ простым примером разбиения является множество $\Pi = \{G_1, \dots, G_{2^r}\}$, которое состоит из всех возможных различных кубов размера~$r$ над множеством~$B$, где $|B| = r$.

\todo{Чанки, балансные переменные, интервалы, математические свойства, детали программной реализации ("простая" реализация, BDD).}

\section{Вероятностный и статистический анализ свойств предложенных разбиений}

Статистические оценки, результаты из IEEE, доверительные интервалы, обоснование мощности выборки.

\section{Экспериментальное исследование}

Вычислительные эксперименты и их результаты.

Multipliers, Sorters, Crypto.
