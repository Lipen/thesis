\chapter{Обзор предметной области}
\label{ch:overview}

В данной главе представлены основные понятия и существующие результаты.


\section{Конечные автоматы}

Конечный автомат (КА) \--- это одна из простых моделей вычислений, свойствам которых посвящено огромное число работ.
% TODO: ссылки
КА описывает систему с конечным числом состояний и переходов между ними.
Конечный автомат может быть задан в виде пятерки $\mathcal{A} = \Tuple{\Sigma, Q, q_0, F, \delta}$, где:
\begin{itemize}
    \item $\Sigma$ --- алфавит входных символов;
    \item $Q$ --- (\textit{конечное}) множество состояний;
    \item $q_0 \in Q$ --- начальное состояние;
    \item $F \subseteq Q$ --- множество терминальных (принимающих) состояний;
    \item $\delta \colon Q \times \Sigma \to Q$ --- функция переходов.
\end{itemize}
Конечный автомат \textit{принимает} (\textit{accepts}) слово $w = w_1 w_2 \ldots w_n \in \Sigma^*$, если после прочтения~$w$ автомат оказывается в одном из терминальном состоянии $s_n \in F$, то есть существует последовательность состояний $s_0, s_1, \ldots, s_n$ такая, что $s_0 = q_0$, $s_{i+1} = \delta(s_i, w_{i+1})$ для всех $i \in \Set{0, 1, \ldots, n-1}$ и $s_n \in F$.

В настоящей работе исследуются задачи синтеза КА, обладающих определенными свойствами, а также задачи верификации уже построенных КА на предмет соответствия конкретным спецификациям~\cite{hachtel1996}.


\section{Булевы схемы}

Булевы схемы также являются простыми и естественными моделями вычислений.
Однако, в отличие от конечных автоматов, булевые схемы задают функции, получающие на вход двоичные слова фиксированной конечной длины.
История булевых схем берет свое начало в работах К.\,Э.~Шеннона~\cite{shannon1938} и В.\,И.~Шестакова~\cite{shestakov1941}.
Помимо огромного числа практических применений, булевы схемы тесным образом связаны со многими фундаментальными разделами современной информатики (Computer Science), в частности, с вычислительной сложностью~\cite{arora2009}.

Булева схема \--- это направленный ациклический ориентированный граф $G = \Pair{V, E}$, где $V$ \--- множество вершин, а $E \subseteq V^2$ \--- множество рёбер (\textit{дуг}).
Вершины такого графа делятся на три типа: входные вершины (\textit{inputs}), выходные вершины (\textit{outputs}) и внутренние вершины (\textit{gates}).
\textit{Ребро} (\textit{дуга}) представляет собой упорядоченную пару вершин.
Для каждой дуги $(u,v) \in E$, вершина~$u$ называется \textit{родителем}~$v$, а $v$ \--- \textit{потомком}~$u$.
Множество всех родителей вершины~$v$ обозначается как~$P_v$.
Вершина называется \textit{входной}, если у нее нет родителей, и \textit{выходной}, если у нее нет потомков\footnote{Здесь стоит отметить, что вполне возможны вариации данных определений.
В некоторых ситуациях \textit{входными}/\textit{выходными} вершинами в схеме считаются некоторые заранее выбранные вершины, но при этом у них могут быть родители/потомки, соотвественно.
В зависимости от контекста, эти родители/потомки игнорируются в соответствующих определениях, связанных с обходом вершин графа от \textit{входов} к \textit{выходам}.}.
Множества входов и выходов обозначаются как $\Vin \subseteq V$ и~$\Vout \subseteq V$ соответственно.
Любая вершина $v \in V \setminus \Vin$ называется \textit{гейтом} (\textit{логическим вентилем}).
В булевой схеме каждому гейту сопоставляется некоторый \textit{логический элемент} из предопределенного набора, называемого \textit{базисом} (например, $\{\land, \neg\}$).
Таким образом, любой логический элемент интерпретирует некоторую элементарную булеву функцию.
Пример булевой схемы представлен на Рисунке~\ref{fig:boolean-circuit-example}.

\begin{figure}[ht]
    \centering
    % \includegraphics[max width=0.5\textwidth]{example-image}
    \begin{adjustbox}{max width=\linewidth}
        \subfile{tex/tikz-circuit-example}
    \end{adjustbox}%
    \caption{Пример булевой схемы с тремя входами ($i_1$,~$i_2$,~$i_3$) и восьмью гейтами}
    \label{fig:boolean-circuit-example}
\end{figure}

Булева схема с $n$~входами и $m$~выходами естественным образом задает (тотальную) дискретную функцию $f \colon \{0, 1\}^n \to \{0, 1\}^m$, где $\{0,1\}^k$ \--- множество всех возможных двоичных слов длины~$k \in \Natural^{+}$.
Имея это в виду, будем использовать обозначение~$S_f$ для представления булевой схемы, задающей функцию~$f$.
Каждому гейту схемы~$S_f$ сопоставлена булева функция, которая соответствует логическому элементу, назначенному этому гейту.

Пусть $\alpha \in \{0,1\}^n$ произвольное слово, поданное на вход~$S_f$.
При проходе по гейтам схемы в фиксированном порядке (обычно указанном топологической сортировкой~\cite{cormen1990}) и вычислении значений элементарных функций, сопоставленных гейтам, получается значение функции~$f$ на входном слове~$\alpha$ в качестве результата.
Этот процесс называется \textit{интерпретацией} схемы~$S_f$ на входе~$\alpha$.

Каждой вершине в схеме~$S_f$ сопоставим булеву переменную.
Обозначим множество переменных, ассоциированных с входами $\Vin$ схемы~$S_f$, как $\Xin = \{ x_1,\dots,x_n \}$.
Переменные, связанные с гейтами, называются \textit{вспомогательными} (\textit{auxiliary}).
Пусть $u$ \--- некоторая вспомогательная переменная, соответствующая гейту~$v$, и $U_v$ \--- множество переменных, связанных с вершинами из~$P_v$.
Предположим, что $h_v$ \--- булева функция, соответствующая логическому элементу, назначенному гейту~$v$, и $F(h_v)$ \--- булева формула над~$U_v$, которая задает функцию~$h_v$.
Обозначим через~$C_v$ КНФ-представление формулы $F(h_v) \equiv u$.

Рассмотрим следующую КНФ:
\begin{equation}\label{eq1}
    C_f = \biglandclap{v \in V \setminus \Vin} C_v
\end{equation}
Будем называть~\eqref{eq1} \textit{шаблонной КНФ} для функции~$f$.
Заметим, что $C_f$~является КНФ-формулой, полученной применением преобразований Цейтина~\cite{tseitin1970} к~схеме~$S_f$.

Ниже, следуя работе~\cite{szeider2006}, будем использовать обозначение~$x^{\sigma}$, где $\sigma \in \{0,1\}$, предполагая, что $x^0$ обозначает отрицательный литерал~$\neg x$, а $x^1$ обозначает положительный литерал~$x$, а также обозначение $\{0,1\}^{|B|}$, что означает множество всех возможных назначений переменных из~$B$.
%Пусть~$F$ будет произвольной булевой формулой над переменными~$X$.
%Обозначим через~$F|_{x=\sigma}$ формулу, полученную подстановкой~$x$ на место~$\sigma$ в~$F$~\cite{chang1973}.
%Формы $x^\sigma\land F$ и $F|_{x=\sigma}$ являются равноправными по удовлетворимости.
%Таким образом, когда мы работаем с формулой $x^\sigma \land F$, мы можем рассматривать единичную дизъюнкцию~$x^\sigma$ как значение~$\sigma$ переменной~$x$ в~$F$.
%Для произвольного набора булевых переменных~$B$ через $\{0,1\}^{|B|}$ мы обозначаем множество всех возможных назначений переменных из~$B$.
Следующий факт был многократно установлен в литературе, например, см.~\cite{bessiere2009,drechsler2009}.
Он использует простой механизм булевой дедукции, известный как правило распространения единичного дизъюнкта (Unit Propagation \--- UP)~\cite{marques-silva2009}.

\begin{lemma}\label{lem1}
    Применение UP к КНФ-формуле $x_1^{\alpha_1} \land \dots \land x_n^{\alpha_n} \land C_f$ для любого $\alpha = (\alpha_1, \dots, \alpha_n)$, $\alpha \in \{0,1\}^{|\Xin|}$ выводит (в форме единичных дизъюнкций) значения всех переменных, связанных с гейтами из $V \setminus \Vin$, включая переменные $y_1, \dots, y_m$, связанные с выходами схемы $S_{f}$: $y_1=\gamma_1, \dots, y_m=\gamma_m$, $f(\alpha) = \gamma = (\gamma_1, \dots, \gamma_m)$.
\end{lemma}

Стоит отметить, что Лемма~\ref{lem1} в сущности означает, что процесс интерпретации схемы~$S_f$ на входном слове~$\alpha$ может быть смоделирован последовательным применением UP к КНФ $C_f \land x_1^{\alpha_1} \land \dots \land x_n^{\alpha_n}$ для любого $\alpha = (\alpha_1, \dots, \alpha_n)$.
Лемма~\ref{lem1} очень полезна при доказательстве свойств, связанных с булевыми схемами и SAT.


% \section{Задачи синтеза и верификации дискретных управляющих моделей}
% \label{sec:synthesis-and-verification}


\section{Международный стандарт IEC~61499}%
\label{sub:iec61499}

% TODO: Здесь должно быть описание стандарта IEC~61499, функциональных блоков и так далее.

Международный стандарт распределенных систем управления и автоматизации IEC~61499~\cite{vyatkin-tii} нацелен на упрощение разработки распределенных киберфизических систем.
Стандарт отличается от \enquote{предыдущего} стандарта IEC~61131\cite{iec-61131} тем, что в IEC~61499 используется событийная модель исполнения.
Этот стандарт предлагает использование так называемых \textit{функциональных блоков} (ФБ), являющихся, по-сути, контейнерами для базовых элементов \--- управляющих конечных автоматов.
Основные типы описываемых в стандарте IEC~61499 функциональных блоков \--- \textit{базовые} и \textit{композитные}.
Функционал композитных блоков определяется сетью базовых ФБ.
Базовые ФБ являются совокупностью интерфейса (описания входных и выходных событий и переменных) и управляющего конечного автомата (\textit{Execution Control Chart} \--- ECC).
Подробное описание формальной модели~\cite{dubinin-2006} базового функционого блока предоставлено в разделе~\ref{sec:basic-fb-model}.


\section{Модель базового функционального блока}%
\label{sec:basic-fb-model}

В данном разделе приводится описание формальной модели базового функционального блока~\cite{dubinin-2006}.
Базовый функциональный блок (\textit{Basic Function Block} \--- BFB) состоит из интерфейса, диаграммы управления выполнением (\textit{Execution Control Chart} \--- ECC), набора алгоритмов управления ($\mathit{Alg}$), а также множества внутренних переменных~$V$.
Отметим, что в данной работе рассматриваются только функциональные блоки без внутренних переменных, поэтому в дальнейшем большинство упоминаний множества внутренних переменных~$V$ будут опущены.

Интерфейс функционального блока включает в себя входные и выходные события (множества $\SetInputEvents$ и $\SetOutputEvents$) и переменные (множества $\SetInputVariables$ и $\SetOutputVariables$), а также отношения ассоциации событий с переменными ($\AssociationInput$ и $\AssociationOutput$).
Формально, $\mathit{Interface} = \Tuple{\SetInputEvents, \SetOutputEvents, \SetInputVariables, \SetOutputVariables, \AssociationInput, \AssociationOutput}$,
где $\AssociationInput \subseteq \SetInputEvents \times \SetInputVariables$,
$\AssociationOutput \subseteq \SetOutputEvents \times \SetOutputVariables$.
В данной работе рассматриваются ФБ, входные/выходные события которых ассоциированы со всеми входными/выходными переменными: $\AssociationInput = \SetInputEvents \times \SetInputVariables$, $\AssociationOutput = \SetOutputEvents \times \SetOutputVariables$.
Отметим, что это не влияет на вычислительную способность системы.
Ассоциация событий только с \enquote{нужными} переменными значительно упрощает восприятие системы человеком и способствует инкапсуляции отдельных частей системы.
Однако на текущем этапе разработки методов, предлагаемых в данной работе, полноценный учет ассоциации событий и переменных является нецелесообразным.

% TODO: ссылку на стандарт?
Алгоритмы управления функциональных блоков реализуются на языках стандарта IEC~61131\=/3 (SFC \--- \textit{Sequential Function Chart}, LD \--- \textit{Ladder Diagram}, FBD \--- \textit{Function Block Diagram}, ST \--- \textit{Structured Text}, IL \--- \textit{Instruction List}) с использованием типов данных, описанных в IEC~61131\=/3 (например, булевы значений, целые числа, числа с плавающей точкой, символы и строки, объекты даты и времени, а также пользовательские типы, такие как перечисления, массивы и структуры), и могут быть представлены функциями вида $f_{\mathit{alg}} \colon \bigprodnolim_{x \in \SetInputVariables} \Dom(x) \times \bigprodnolim_{z \in \SetOutputVariables} \Dom(z) \times \bigprodnolim_{v \in V} \Dom(v) \to \bigprodnolim_{z \in \SetOutputVariables} \Dom(z) \times \bigprodnolim_{v \in V} \Dom(v)$, которые изменяют значения выходных и внутренних переменных в зависимости от входных переменных.
Здесь $\Dom(v)$ означает домен (область значений) переменной~$v$.
В данной работе рассматриваются только логические контроллеры ($\forall x \in \SetInputVariables, z \in \SetOutputVariables : \Dom(x) = \Dom(z) = \Bool = \Set{\top, \bot}$) без внутренних переменных ($V = \emptyset$) и с алгоритмами управления, явно независящими от входных переменных, поэтому функции алгоритмов принимают простой вид $f_{\mathit{alg}} \colon \BoolVec{\SetOutputVariables} \to \BoolVec{\SetOutputVariables}$.
% [TODO: обоснование допущений об отсутствии внутренних переменных]
% FIXME: check
Отметим, что данный вид алгоритмов обладает способностью к преобразованию любого набора значений выходных переменных, что является достаточным в задачах логического управления.

Диаграмма управления выполнением (\textit{Execution Control Chart} \--- ECC) представляется в виде конечного автомата-преобразователя (\textit{finite-state transducer}), похожего на автомат Мура, в котором переходы дополнены охранными условиями (\textit{guard conditions}), а состояния имеют ассоциированные выходные события и описанные выше алгоритмы управления.
Формально, $\mathit{ECC} = \Tuple{\mathit{ECState}, s_0, \mathit{ECAction}, \mathit{ECTran}, \mathit{ECTCond}, \mathit{PriorT}}$,
где $\mathit{ECState} = (s_0, s_1, \dotsc, s_r)$ \--- множество состояний автомата;
$s_0$ \--- начальное состояние;
$\mathit{ECAction} \colon \mathit{ECState} \setminus\nobreak \Set{s_0} \to \mathit{ECA}^{*}$ \--- функция соответствия состояниям автомата действий,
где $\mathit{ECA} = \mathit{Alg} \times \SetOutputEvents \union \mathit{Alg} \union \SetOutputEvents$ \--- множество синтаксически корректных действий автомата,
$\mathit{ECA}^{*} = \bigunionnolim_{k = 0}^{\infty} \mathit{ECA}^k$ \--- множество всевозможных последовательностей действий (включая пустую последовательность действий, обозначаемую впоследствии $\mathit{ECA}^0 = \Tuple{\varepsilon})$;
$\mathit{ECTran} \subseteq \mathit{ECState}^2$ \--- множество переходов автомата;
$\mathit{ECTCond} \colon \mathit{ECTran} \to \bigl[ \bigprodnolim_{e_i \in \SetInputEvents} \Dom(e_i) \times \bigprodnolim_{x \in \SetInputVariables} \Dom(x) \times \bigprodnolim_{z \in \SetOutputVariables} \Dom(z) \times \bigprodnolim_{v \in V} \Dom(v) \to \Bool \bigr]$ \--- функция соответствия переходам автомата охранных условий \--- логических функций, зависящих от входных событий и (потенциально) всех переменных; $\mathit{PriorT} \colon \mathit{ECTran} \to \mathbb{N}_{+}$ \--- функция приоритетов переходов.

% TODO: Описание семантики выполнения

Говорят, что ECC находится в \textit{канонической форме}~\cite{dubinin-2006}, если с каждым состоянием автомата ассоциировано не более одного действия.
В данной работе в основном рассматриваются канонические конечно-автоматные модели, состояния которых имеют в точности одно выходное действие, так как такие модели в большистве случаев значительно проще для реализации, использования и анализа.
% TODO: add missing section
% В разделе~\ref{sec:non-canonical-fb-model} дополнительно приводится расширение разрабатываемых методов применительно к неканоническим~ECC\@.


% TODO: \section{Сценарии выполнения и дерево сценариев}

\section{Сценарии выполнения}%
\label{sec:scenarios}

Желаемое поведение системы может быть описано с помощью примеров поведения.
Одним из самых простых способов представления примеров поведения являются \textit{трассировки} \--- последовательности входных воздействий и реакций на них конкретной системы в виде выходных действий.
Зачастую отсутствие выходного действия также является допустимой реакцией, однако трассировки являются лишь последовательностями действий и не обладают какой-либо дополнительной структурой.
Поэтому в данной работе для описания примеров поведения используются более структурированные объекты \--- так называемые \textit{сценарии выполнения}.

\textit{Сценарий выполнения} $s$ \--- это последовательность \textit{элементов сценария} $s_i = \Pair{\InputAction, \OutputAction}$.
Каждый элемент сценария представляет собой пару \enquote{воздействие\--реакция}: входное действие~$\InputAction$ содержит входное событие~$\InputEvent \in \SetInputEvents$ и \textit{вход} \--- набор значений входных переменных~$\InputValues \in \boolvec{\SetInputVariables}$; выходное действие~$\OutputAction$ содержит выходное событие~$\OutputEvent \in \SetOutputEvents$ и \textit{выход} \--- набор значений выходных переменных~$\OutputValues \in \boolvec{\SetOutputVariables}$.

Конвертация трассировок в сценарии выполнения происходит следующим образом.
Каждая последовательная пара входное\--выходное действие $\InputAction$\==$\OutputAction$ образует \textit{активный} элемент сценария $s_i^{\text{(active)}} = \Pair{\InputAction, \OutputAction}$.
Если в трассировке входные воздействия следуют одно за другим, без промежуточных реакций системы, то такие действия соответствуют \textit{пассивным} элементам сценариев $s_i^{\text{(passive)}} = \Pair{\InputAction, \Action{\varepsilon}{\PreviousOutputValues}}$, где $\PreviousOutputValues$ \--- выход, равный выходу в предыдущем элементе сценариев.
Стоит отметить, что если первый элемент сценария является пассивным, то \enquote{предыдущим} выходом~$\PreviousOutputValues$ считается инициализирующий набор~$\InitialOutputValues$, соответствующий начальному состоянию системы, и обычно равный~$\Vector{0 \dots 0}$.
Также стоит отметить, что в дальнейшем пассивное выходное действие $\Action{\varepsilon}{\PreviousOutputValues}$ может быть обозначено просто~$\varepsilon$.

В качестве примера рассмотрим следующий набор сценариев исполнения:
\begin{itemize}[itemsep=1pt]
    \small
    \item \texttt{% s1
        \colorbox{myr}{\dashuline{%
            in=\ActionTT{R}{10}%
        }},%
        \colorbox{myb}{%
            in=\ActionTT{R}{01},out=\ActionTT{B}{1}%
        },%
        \colorbox{myb}{%
            in=\ActionTT{R}{11},out=\ActionTT{A}{0}%
        },%
        \colorbox{myb}{%
            in=\ActionTT{R}{11},out=\ActionTT{A}{1}%
        }%
    }
    \item \texttt{% s2
        \colorbox{myg}{%
            in=\ActionTT{R}{01},out=\ActionTT{B}{1}%
        },%
        \colorbox{myr}{\dashuline{%
            in=\ActionTT{R}{01}%
        }},%
        \colorbox{myb}{%
            in=\ActionTT{R}{11},out=\ActionTT{A}{0}%
        },%
        \colorbox{myb}{%
            in=\ActionTT{R}{00},out=\ActionTT{B}{1}%
        }%
    }
    \item \texttt{% s3
        \colorbox{myg}{%
            in=\ActionTT{R}{01},out=\ActionTT{B}{1}%
        },%
        \colorbox{myr}{\dashuline{%
            in=\ActionTT{R}{01}%
        }},%
        \colorbox{myr}{\dashuline{%
            in=\ActionTT{R}{00}%
        }},%
        \colorbox{myb}{%
            in=\ActionTT{R}{01},out=\ActionTT{A}{1}%
        }%
    }
\end{itemize}
Здесь цветами выделены последовательные пары входных\--выходных действий.
Красным цветом отмечены (и подчеркнуты) входные действия без пары, то есть те воздействия, на которые нет реакции системы.
Зеленым цветом выделены совпадающие первые пары во второй и третьей последовательностях.
Остальные пары событий выделены синим.
% Здесь различными цветами выделены последовательные пары входных\--выходных действий. Красным цветом отмечены (и подчеркнуты) входные действия без пары, то есть те воздействия, на которые нет реакции системы.
Данные трассировки соответствуют множеству сценариев выполнения $\SetScenarios =\nobreak \Set{s_1, s_2, s_3}$:
\begin{equation}
    \label{eq:example-scenarios}
    \def\LineGapScenarios{-2pt}
    % \small
    \begin{aligned}
        & s_1 = [
            \colorbox{myr}{\(\Pair{\ActionTT{R}{10}, \varepsilon}\)},
            \colorbox{myb}{\(\Pair{\ActionTT{R}{01}, \ActionTT{B}{1}}\)},
            \colorbox{myb}{\(\Pair{\ActionTT{R}{11}, \ActionTT{A}{0}}\)},
            \colorbox{myb}{\(\Pair{\ActionTT{R}{11}, \ActionTT{A}{1}}\)}
        ], \\[\LineGapScenarios]
        & s_2 = [
            \colorbox{myg}{\(\Pair{\ActionTT{R}{01}, \ActionTT{B}{1}}\)},
            \colorbox{myr}{\(\Pair{\ActionTT{R}{01}, \varepsilon}\)},
            \colorbox{myb}{\(\Pair{\ActionTT{R}{11}, \ActionTT{A}{0}}\)},
            \colorbox{myb}{\(\Pair{\ActionTT{R}{00}, \ActionTT{B}{1}}\)}
        ], \\[\LineGapScenarios]
        & s_3 = [
            \colorbox{myg}{\(\Pair{\ActionTT{R}{01}, \ActionTT{B}{1}}\)},
            \colorbox{myr}{\(\Pair{\ActionTT{R}{01}, \varepsilon}\)},
            \colorbox{myr}{\(\Pair{\ActionTT{R}{00}, \varepsilon}\)},
            \colorbox{myb}{\(\Pair{\ActionTT{R}{01}, \ActionTT{A}{1}}\)}
        ].
    \end{aligned}
\end{equation}

Можно заметить, что если система является детерминированной, то последовательности одинаковых пассивных элементов сценария могут быть заменены одним пассивным элементом с общим входным действием \--- это является простейшей операцией \emph{предобработки}, что позволяет сократить размеры сценариев, не искажая их смысла.
Аналогично, последовательности пассивных элементов с чередующимися входными действиями могут быть заменены множеством элементов с уникальными входными действиями.
Несмотря на простоту, такие техники предобработки оказываются крайне эффективными на реальных данных, изобилирующих повторяющимися пассивными элементами.
В дальнейшем в данной работе считается, что все используемые сценарии выполнения подвергаются описанной предобработке, если не указано обратное.

\emph{Дерево сценариев} $\Tree$ \--- префиксное дерево, построенное из сценариев~$\SetScenarios$.
Стоит отметить, что перед построением дерева сценарии выполнения дополняются фиктивными пассивными элементами~$\Action{\varepsilon}{\InitialOutputValues}$.
Это приводит к тому, что у всех сценариев образуется общий префикс \--- добавленные фиктивные элементы соответствуют корню префиксного дерева сценариев~$\Tree$.
Каждая вершина в дереве и входящее в нее ребро соответствуют элементам сценария \--- вершины отмечены выходными действиями, а входящие ребра входными действиями.
\emph{Позитивным деревом сценариев}~$\PositiveTree$ называется дерево сценариев, построенное по позитивным сценариям~$\SetPositiveScenarios$.

Здесь и далее используется следующая нотация для дерева сценариев:
$\SetTreeNodes$ \--- множество вершин дерева;
$\TreeRoot \in \SetTreeNodes$ \--- корень дерева;
$\tp{v} \in \SetTreeNodes$ \--- родитель вершины~$v \in \SetTreeNodes$ ($v \neq \TreeRoot$);
$\tie{v} \in \SetInputEvents$ \--- входное событие на входящем ребре вершины $v \in \SetTreeNodes$ ($v \neq \TreeRoot$);
$\toe{v} \in \SetOutputEvents \union \Set{\varepsilon}$ \--- выходное событие в вершине $v \in \SetTreeNodes$, где $\varepsilon$~обозначает пустое событие;
$\SetTreeNodesActive = \Set{v \in \SetTreeNodes \setminus \Set{\TreeRoot} \given \toe{v} \neq \varepsilon}$ \--- множество \textit{активных} вершин;
$\SetTreeNodesPassive = \Set{v \in \SetTreeNodes \setminus \Set{\TreeRoot} \given \toe{v} = \varepsilon}$ \--- множество \textit{пассивных} вершин;
$\SetTreeInputs \subseteq \boolvec{\SetInputVariables}$ \--- множество \textit{входов} (\textit{вход} \--- набор значений входных переменных), встречающихся в сценариях;
$\tin{v} \in \SetTreeInputs$ \--- вход на входящем ребре вершины $v \in \SetTreeNodes$ ($v \neq \TreeRoot$);
$\tov{v,z} \in \Bool$ \--- значение выходной переменной~$z \in \SetOutputVariables$ в вершине~$v \in \SetTreeNodes$.
У корня дерева нет родительской вершины, поэтому значения $\tp{\TreeRoot}$, $\tie{\TreeRoot}$ и~$\tin{\TreeRoot}$ неопределены.
На рисунке~\ref{fig:scenario-tree} изображен пример дерева сценариев, построенного по сценариям~$\SetScenarios$~\eqref{eq:example-scenarios}.

%% Picture: Scenario tree example
\begin{figure}[!htb]
    \centering
    \begin{adjustbox}{max width=\linewidth}
        \subfile{tex/tikz-scenario-tree}%
    \end{adjustbox}
    \caption{Дерево сценариев, построенное по сценариям выполнения~$\SetScenarios$~\eqref{eq:example-scenarios}}
    \label{fig:scenario-tree}
\end{figure}



\section{Линейная темпоральная логика}%
\label{sub:ltl}

% Здесь должно быть описание LTL~\cite{ltl} и контрпримеров, получаемых с помощью NuSMV~\cite{nusmv}. Негативные сценарии выполнения и соотвествующее дерево негативных сценариев описываются отдельно в разделе~\ref{sec:negative-scenarios}

Формальная спецификация может быть проверена с помощью верификатора (\textit{model checker}) \--- специализированного программного средства, которое проверяет выполнение заданных свойств в системе и генерирует контрпримеры к нарушенным свойствам.
В данной работе был использован символьный верификатор NuSMV~\cite{nusmv}, а рассмотренные спецификации систем были составлены на языке линейной темпоральной логики (\textit{Linear Temporal Logic} \--- LTL)~\cite{ltl}, полностью поддерживаемом NuSMV\@.
Для так называемых \enquote{свойств безопасности} (\textit{safety properties}), выражающих отсутствие нежелательного поведения (например, \enquote{с системой никогда не произойдёт ничего плохого}), контрпримером является конечная последовательность вычислительных состояний (\textit{execution state}), приводящая к нежелательному поведению.
Для так называемых \enquote{свойств живости} (\textit{liveness properties}), выражающих присутствие желаемого поведения (например, \enquote{с системой точно произойдет что-то хорошее}), контрпримером является бесконечная, но циклическая последовательность состояний, представляющая нежелательное циклическое поведение системы, и которая может быть представлена в виде конечного префикса с последующим циклом конечной длины~\cite{clarke1999}.


\section{Формальная верификация с использованием проверки моделей}

В данной работе система специфицируется с помощью набора формул линейной темпоральной логики (LTL)~\cite{ltl}.
LTL-формула описывает некоторые темпоральные свойства путей исполнения формальной системы.
Формулы на языке LTL включают атомарные высказывания (некоторые элементарные утверждения о системе), логические связки ($\land$,~$\lor$, $\neg$, $\implies$ и другие) и темпоральные операторы:
$\Temp{X}$ \=== \enquote{\textit{next}},
$\Temp{U}$ \=== \enquote{\textit{until}},
$\Temp{G}$ \=== \enquote{\textit{always}},
$\Temp{F}$ \=== \enquote{\textit{in the future}}.
С помощью LTL-формул можно специфицировать свойства \textit{безопасности} (\enquote{что-то плохое никогда не происходит}) и \textit{живучести} (\enquote{что-то хорошее в конечном итоге произойдёт}) данной системы.
Примером свойства безопасности является формула $\Temp{G} \neg P$, которая утверждает, что некоторый предикат~$P$ всегда ложен (во всех путях исполнения).
Примером свойства живучести является формула $\Temp{G} (P \implies \Temp{F} Q)$, которая утверждает, что если предикат~$P$ истиннен, то предикат~$Q$ в конечном итоге также станет истинным в любом пути исполнения.

Проверка моделей (\textit{model checking})~\cite{clarke1999} \--- это техника, которая может использоваться для верификации заданной модели конечного автомата относительно заданной спецификации (в данном случае набора LTL-формул) и получения контрпримера траектории выполнения, если эта спецификация нарушена.
Контрпримеры могут быть преобразованы в \textit{негативные сценарии} \--- сценарии выполнения, представляющие нежелательное поведение автомата или системы автоматов.
% Этот процесс подробно описан в разделе~\ref{sec:monolithic:trees:negative}.

Проверка моделей является одним из важнейших методов формальной верификации программ и аппаратных систем.
Основная идея состоит в автоматической проверке, соответствует ли модель системы формальной спецификации.
% Для этого используется полное перечисление всех возможных состояний системы и проверка всех возможных переходов между этими состояниями.
Существует два основных подхода к проверке моделей: символьная проверка моделей (\textit{symbolic model checking}) и ограниченная проверка моделей (\textit{Bounded Model Checking} \--- BMC).


\subsection{Символьная проверка моделей}

Символьная проверка моделей (\textit{symbolic model checking}) использует булевы функции и бинарные диаграммы решений (Binary Decision Diagram \--- BDD) для представления и манипулирования множествами состояний системы.
Классическими инструментами символьной проверки моделей являются NuSMV~\cite{nusmv} и SPIN~\cite{holzmann1997}.

NuSMV (открытый инструмент для символьной проверки моделей) позволяет описывать системы на языке SMV и проверять их на соответствие LTL и CTL (Computational Tree Logic) спецификациям \--- наборам свойств на указанных языках темпоральной логики.
Он использует BDD для представления переходных систем и позволяет анализировать большие пространства состояний.

SPIN (Simple Promela INterpreter) предназначен для проверки моделей, описанных на языке Promela.
SPIN использует оптимизации для поиска путей и проверки LTL спецификаций, а также может генерировать исходный код на языке~C для исполнения моделей.


\subsection{Ограниченная проверка моделей (BMC)}

% TODO: links
Ограниченная проверка моделей (\textit{Bounded Model Checking} \--- BMC) используется для поиска контрпримеров ограниченной длины.
BMC заключается в сведении задачи проверки модели к задаче выполнимости SAT, что позволяет эффективно искать ошибки в моделях.

BMC итеративно проверяет, существует ли трасса (сценарий выполнения), нарушающая спецификацию, длиной не более заданного числа шагов.
% Если такая трасса найдена, она представляется в виде контрпримера.
% В~противном случае, длина искомой трассы увеличивается, и процесс повторяется.
Процесс решения BMC через сведения к SAT включает несколько этапов:
\begin{enumerate}
    \item Построение булевой формулы, описывающей переходную систему и спецификацию.
    \item Ограничение длины траектории выполнения (глубина поиска).
    \item Преобразование построенной формулы в CNF (Conjunctive Normal Form) для SAT-решателя.
    \item Решение полученной SAT-задачи с помощью SAT-решателя.
    \item Анализ результата: если SAT-решатель нашел удовлетворяющую подстановку, она интерпретируется как контрпример; если нет, то глубина поиска (длина искомого контрпримера) увеличивается, и процесс повторяется.
\end{enumerate}

Классическими инструментами для BMC являются CBMC (C~Bounded Model Checker) и Eldarica.
Эти инструменты используют современные SAT-решатели для поиска контрпримеров и позволяют анализировать сложные программы на языке~C с большим пространством состояний.


\subsection{LTL-синтез}

LTL-синтез (Linear Temporal Logic synthesis) представляет собой процесс генерации моделей, которые непосредственно удовлетворяют заданной формальной спецификации в виде LTL-свойств.
Этот подход позволяет автоматически строить системы, которые гарантированно выполняют все указанные свойства, без необходимости в последующей проверке.

Процесс синтеза LTL включает следующие этапы:
\begin{enumerate}
    \item Спецификация свойств системы с использованием LTL-формул.
    \item Построение автомата, эквивалентного данной LTL-формуле.
    \item Генерация контроллера или системы автоматов, которые удовлетворяют построенному автомату.
\end{enumerate}

Существуют различные инструменты и методы для LTL-синтеза, включая $\mathrm{GR}(1)$ синтез и автоматный синтез на основе игр.
Эти методы активно исследуются и развиваются, так как позволяют создавать надежные системы с гарантированными свойствами.

Примерами инструментов для LTL-синтеза являются:
\begin{itemize}
    \item BoSy~\cite{bosy,not-bosy}, инструмент для синтеза систем переходов по LTL\-/спецификациям, основанный на сведении к Quantified SAT \--- QSAT.
    \item Strix~\cite{strix}, инструмент для реактивного синтеза автоматов Мили, основанный на теории игр.
    \item SLUGS~\cite{ehlers2016}, который поддерживает GR(1) синтез и предоставляет эффективные алгоритмы для генерации контроллеров.
    \item Spectra~\cite{maoz2021}, язык спецификаций и соответствующий инструмент, который позволяет описывать и синтезировать реактивные системы.
\end{itemize}

LTL-синтез и проверка моделей являются мощными инструментами для создания и анализа надежных систем, обеспечивая соответствие сложным формальным спецификациям и позволяя автоматически обнаруживать ошибки на ранних стадиях разработки.


\section{Методы синтеза конечно-автоматных моделей}
\label{sub:automata-synthesis}

% TODO: More intro here?

Задача поиска минимального детерминированного конечного автомата по примерам поведения является NP-полной задачей~\cite{gold}, а сложность задачи LTL-синтеза дважды экспоненциальная от размера LTL\-/спецификации~\cite{rosner-phd}.
Несмотря на это, синтез различных типов конечно-автоматных моделей по примерам поведения и/или формальной спецификации был исследован во многих научных работах~\cite{heule2010,efsm-tools,zakirzyanov2019,buzhinsky-tii,bosy,tsarev-egorov-gecco,giantamidis-tripakis,petrenko,petrenko2,neider,g4ltl-st,smetsers-lata}, где используются методы, основанные на эвристическом объединении состояний (\textit{state merging}), эволюционные алгоритмы, а также методы, основанные на применении SAT- и SMT-решателей.
В~данной работе рассматриваются только точные и эффективные методы, поэтому внимание уделяется методам с применением SAT-решателей.

Расширенный конечный автомат (\textit{Extended Finite State Machine} \--- EFSM) является моделью, наиболее близкой к рассматриваемой в данной работе модели ECC\@. EFSM является объединением автомата Мили и Мура, расширенный условными переходами.
Переходы в EFSM помечены входными событиями и охранными условиями \--- булевыми функциями от входных переменных, а состояния EFSM имеют ассоциированные выходные действия.
Для синтеза EFSM по примерам поведения и LTL\-/спецификации существует несколько подходов~\cite{efsm-tools,walkinshaw}, основанных на сведении к задаче SAT.
В~\cite{efsm-tools} LTL\-/спецификация учитывается путём применения итеративного подхода запрета контрпримеров.
Существенным недостатком~\cite{efsm-tools} является то, что охранные условия должны быть известны заранее, а также то, что синтезируемые алгоритмы в состояниях EFSM являются лишь указаниями на некоторые внешние процедуры.
В~\cite{walkinshaw} решается задача синтеза вычислимых выходных алгоритмов, однако предполагается, что базовая модель автомата (его структура \--- состояния и переходы между ними) известна заранее или получается отдельно.
В~общем случае, при использовании исходных данных, получаемых при black-box тестировании системы, информация о внутреннем устройстве системы, а также о доступных внешних процедурах и их поведении, оказывается недоступной, поэтому существующие методы синтеза EFSM не подходят для решения задачи синтеза модели ECC, рассматриваемой в данной работе.

Программное средство BoSy~\cite{bosy,not-bosy} реализует так называемый ограниченный синтез (\textit{bounded synthesis}) системы переходов (\textit{transition system}) по LTL\-/спецификации.
Синтез \enquote{ограничен} в том смысле, что позволяет синтезировать систему заданного размера, либо гарантировать отсутствие решения заданного размера.
В~BoSy реализовано не только сведение задачи LTL-синтеза к SAT, но также разработано более эффективное (при рассмотренной авторами постановке задачи) сведение с использованием Quantified SAT (QSAT).
При использовании SAT-кодировки, синтезируемые системы переходов являются \enquote{явными} (\textit{explicit}) \--- охранные условия на переходах являются полными зависят от всех входных переменных.
При использовании QSAT-кодировки, системы получаются \enquote{символьными} (\textit{symbolic}) \--- охранные условия синтезируются в виде полноценных булевых формул.
Используемые в BoSy подход ограниченного синтеза позволяет синтезировать минимальные модели в терминах числа состояний, однако важный вопрос о размере охранных условий обходится стороной \--- синтезируемые модели, как правило, обладают огромными охранными условиями, что сильно затрудняет их восприятие человеком, а также ограничивает применимость таких моделей во встраиваемых системах.
В~\cite{bounded-cycle} предлагается способ упрощения генерируемых моделей, заключающийся в дополнении SAT сведения специальными ограничениями для минимизации числа циклов в системе переходов, однако это слабо влияет на размеры и форму охранных условий.
Также стоит упомянуть, что отличительной особенностью LTL-синтеза является то, что в качестве входных данных не используются примеры поведения, так как предполагается полнота входной спецификации \--- в том смысле, что она описывает все желаемое поведение системы.
Несмотря на то, что примеры поведения могут быть представлены в виде LTL-свойств, этот подход становится крайне неэффективным уже на небольших наборах данных.
Другие программные средства для LTL-синтеза, например G4LTL\=/ST~\cite{g4ltl-st} и Strix~\cite{strix}, обладают аналогичными недостатками по отношению к рассматриваемой задаче: отсутствие минимизации охранных условий и невозможность эффективного учета примеров поведения.

В статье~\cite{fbCSP} предлагается метод \smallcaps{fbCSP} для синтеза конечно-автоматных моделей функциональных блоков по примерам поведения, основанный на сведении к задаче удовлетворения ограничений (\textit{Constraint Satisfaction Problem} \--- CSP)~\cite{montanari1974}.
Однако методу \smallcaps{fbCSP} присущи следующие ограничения.
Получаемые модели обладают \textit{полными} охранными условиями \--- соответствующие булевы формулы зависят от \textit{всех} входных переменных.
Такие модели практически не обобщаются (\textit{generalize}) \--- некорректно работают на входных данных, которые не были использованы в процессе \enquote{обучения} (синтеза).
В~\cite{fbCSP} это отчасти исправляется дополнительной жадной минимизацией охранных условий, однако жадный подход не гарантирует, что охранные условия будут наименьшими.
В работе~\cite{chivilikhin-18} метод \smallcaps{fbCSP} был расширен процедурой запрета контрпримеров для учета LTL\-/спецификации (в~дальнейшем это расширение будет называться \smallcaps{fbCSP+LTL}), аналогично работе~\cite{efsm-tools}.
При этом охранные условия в генерируемых моделях представляются в виде конъюнкции литералов входных переменных.
Основным недостатком этого подхода является его низкая эффективность в тех случаях, когда темпоральная спецификация покрыта сценариями выполнения не полностью.

В работе~\cite{chivilikhin-19} разработан двухэтапных подход:
сначала генерируется \textit{базовая} модель с использованием метода, основанного на SAT, а затем охранные условия полученной модели отдельно минимизируются с помощью CSP \--- деревья разбора булевых формул, соответствующих охранным условиям, кодируются в CSP, а затем минимизируется их суммарный размер.
Таким образом, получаемая модель является минимальной, однако независимость двух этапов приводит к тому, что модель не является наименьшей (в терминах суммарного размера охранных условий).

Резюмируя, ни один из рассмотренных методов, каждый из которых по своему хорош при конкретной постановке задачи, не позволяет \textit{одновременно} и \textit{эффективно} учитывать при синтезе конечно-автоматных моделей как (1)~примеры поведения, так и (2)~LTL\-/спецификацию, а также (3)~минимальность генерируемых моделей.
В~ходе выполнения данной работы был разработан метод, который фактически является расширением~\cite{chivilikhin-19} \--- объединением двух независимых этапов в один \--- и вносит вклад в расширение \textit{state-of-the-art} конечно-автоматного синтеза с применением SAT-решателей, а именно: \textit{одновременно} поддерживает учет позитивных примеров поведения, реализует индуктивный синтез, основанный на контрпримерах \--- для учета LTL\-/спецификации, а также позволяет генерировать минимальные модели \--- как в терминах числа состояний, так и в терминах суммарного размера охранных условий.

% \subsection{Верификация конечно-автоматных моделей}
% \label{sub:automata-verification}


\section{Задача проверки эквивалентности булевых схем}
\label{sub:lec}

Задача проверки эквивалентности булевых схем (Logical Equivalence Checking \--- LEC) является одной из ключевых комбинаторных проблем в автоматизации проектирования электроники (Electronic Design Automation \--- EDA) и важной частью процесса верификации цифровых схем.
В этом разделе даны основные понятия о LEC, которые будут использованы в дальнейшем.

Рассмотрим две булевы схемы $S_f$ и~$S_h$, определяющие функции $f, h: \{0,1\}^n \to \{0,1\}^m$.
Задача LEC заключается в том, чтобы определить, являются ли две заданные схемы эквивалетными \--- обладают одинаковыми выходами на всех возможных входах, что выражается в том, что соответствующие функции поточечно равны, $f \cong h$.
Задача LEC может быть сведена к задаче выполнимости булевых формул (SAT), ниже это показано на примерах.

Используя $S_f$ и~$S_h$, построим новую схему, обозначаемую~$S_{f \glue h}$ (см. Рисунок~\ref{fig:glued}), которая получается из $S_f$ и~$S_h$ путем \enquote{склейки} вместе входных вершин \--- обозначим её~$S_{f \glue h}$.
Она имеет тот же набор входов~$\Vin$, как и схемы $S_f$ и~$S_h$, и определяет следующую функцию:
\begin{equation}\label{eq:f-glue-h}
    f \glue h \colon \{0,1\}^n \to \{0,1\}^{2m}
\end{equation}

\begin{figure}[!htb]
    \centering
    \subfile{tex/tikz-glued}
    \caption{Склеенная схема $S_{f \glue h}$, построенная с использованием одного и того же набора входов для двух схем $S_f$ и~$S_h$}
    \label{fig:glued}
\end{figure}

Обозначим через $\Vout_f$ и~$\Vout_h$ множества выходов схем $S_f$ и~$S_h$, а через $Y_f = \{y_1^f, \dots, y_m^f\}$ и $Y_h = \{y_1^h, \dots, y_m^h\}$ множества переменных, связанных с вершинами из $\Vout_f$ и~$\Vout_h$ соответственно, упорядоченные согласно семантике схем.
Теперь рассмотрим формулу $(y_1^f \xor y_1^h) \lor \dots \lor (y_m^f \xor y_m^h)$, которая задает булеву функцию $\mathcal{M} \colon \{0,1\}^{2m} \to \{0,1\}$, называемую \textit{miter}~\cite{brand1983}.
Мы обозначим булеву схему, реализующую функцию $\mathcal{M} \circ (f \glue h)$ как~$S_{f \xor h}$ и будем ссылаться на нее как на \textit{miter-схему}.
Рассмотрим формулу
\begin{equation}\label{eq:miter-cnf}
    C_{f\xor h} = C_{f \glue h} \land C(\mathcal{M}) ,
\end{equation}
где $C_{f \glue h}$ \--- шаблонная CNF для функции~\eqref{eq:f-glue-h}, а $C(\mathcal{M})$ выглядит следующим образом:
\begin{align*}
    C(\mathcal{M}) = & C(w_1 \equiv (y_1^f \xor y_1^h)) \land \\
    \land & \ldots \land \\
    \land & C(w_m \equiv (y_m^f \xor y_m^h)) \land \\
    \land & (w_1 \lor \ldots \lor w_m) ,
\end{align*}
% где $C(w_j \equiv (y_j^f \xor y_j^h))$, $j\in \{1,\dots,m\}$ \--- CNF-представление булевой функции, заданной формулой $w_j \equiv (y_j^f \xor y_j^h)$.
где $C(f)$ \--- CNF-представление булевой функции, заданной формулой~$f$.
Из Леммы~\ref{lem1} непосредственно следует, что $S_f$~и~$S_h$ эквивалентны тогда и только тогда, когда $C_{f \xor h}$~невыполнима.


% \section{Задача генерации тестовых шаблонов для верификации булевых схем}
% \label{sub:atpg}

% В этом разделе рассматривается задача \textit{автоматической} генерации тестовых шаблонов (Automatic Test Pattern Generation \--- ATPG) для верификации булевых схем.
% Сначала вводится понятие модели неисправностей (\textit{fault model}).
% Затем формулируется задача генерации шаблонов (ATPG) для комбинационных (\textit{combinational}) булевых схем.
% Также упоминается секвенциальная (\textit{sequential}) постановка задачи ATPG для схем с элементами памяти, такими как триггеры (\textit{flip-flops}).
% Наконец, кратко рассматриваются классические алгоритмы ATPG, работающие на структуре схемы.
% % Представление оставлено кратким, для дальнейшего чтения мы ссылаемся на [JG03].
% В~разделе~\ref{sub:sat-atpg} отдельно рассматриваются методы решения задачи ATPG, основанные на сведении к задаче выполнимости (SAT), как наиболее релевантные к текущей работе.

% \subsubsection{Модель неисправности Stuck-At}

% После производства чипа необходимо проверить его функциональную корректность относительно спецификации схемы на уровне булевых элементов.
% Без этой проверки некорректный чип будет доставлен заказчикам, что может привести к неполадкам в конечном продукте. Это, конечно же, недопустимо.
% С другой стороны, из-за дефектов материала, вариаций процесса во время производства и~т.\@\:д.\@ возможен широкий диапазон неисправностей.
% Но непосредственная проверка всех возможных физических дефектов невозможна.
% Поэтому вводится абстракция в виде модели неисправности.
% Модель неисправности Stuck-At (SAFM) [BF76] хорошо известна и широко используется на практике.
% В этой модели неисправности предполагается, что одна линия застряла на фиксированном значении вместо зависимости от значений входов.
% Когда линия застряла на значении~0, это называется неисправностью stuck-at-0 (SA0).
% Аналогично, если линия застряла на значении 1, это называется неисправностью stuck-at-1 (SA1).

% % TODO: \begin{example}
% \textbf{Пример.}
% Рассмотрим схему, показанную на рисунке 27.7(a). Когда на линии d вводится неисправность SA0, получается неисправная схема, показанная на рисунке 27.7(b). Выход элемента И отключается, и вход элемента ИЛИ постоянно принимает значение 0.

% Помимо SAFM было предложено ряд других моделей неисправностей, например, клеточная модель неисправностей [Fri73], где меняется функция одного элемента, или модель мостовой неисправности [KP80], где предполагается, что две линии устанавливаются в одно значение.
% Эти модели неисправностей в основном охватывают статические физические дефекты, такие как обрывы или замыкания.
% Динамические эффекты охватываются моделями неисправностей задержки.
% В~модели неисправности по задержке пути [Smi85] одна неисправность означает, что изменение значения вдоль пути от входов к выходам в схеме не приходит в течение времени цикла тактового сигнала.
% Вместо путей модель неисправности задержки на входах элементов [HRVD77, SB77] учитывает задержку в элементах.

% Далее рассматривается только SAFM из-за его высокой значимости в практических применениях.
% Эту значимость можно объяснить двумя наблюдениями: количество неисправностей имеет порядок размера схемы, и моделирование неисправностей в SAFM относительно просто, то есть для статической модели неисправностей вычислительная сложность генерации тестовых шаблонов ниже по сравнению с динамическими моделями неисправностей.

% \subsubsection{Комбинационный ATPG}

% Автоматическая генерация тестовых шаблонов (ATPG) \--- это задача определения всех набора тестовых шаблонов для заданной схемы с учетом модели неисправностей.
% Тестовый шаблон для конкретной неисправности \--- это назначение на основные входы схемы, которое приводит к различным выходным значениям в зависимости от наличия неисправности.
% Вычисление булевой разности между бездефектной и неисправной схемами дает все тестовые шаблоны для конкретной неисправности.
% Эта конструкция аналогична схеме \textit{miter}~\cite{brand1983}, поскольку ее можно использовать для проверки эквивалентности комбинационных схем.

% % TODO: \begin{example}
% \textbf{Пример.}
% Снова рассмотрим неисправность SA0 в схеме на рисунке 27.7.
% Назначение входов $a = 1$, $b = 1$, $c = 1$ приводит к значению выхода $f = 1$ для корректной схемы и к значению выхода $f = 0$ в случае наличия неисправности.
% Поэтому это назначение входов является тестовым шаблоном для неисправности SA0 на линии~$d$.
% Конструкция для вычисления булевой разности бездефектной и неисправной схем показана на рисунке 27.8.

% Когда тестовый шаблон существует для конкретной неисправности, эта неисправность классифицируется как \textit{тестируемая} (\textit{testable}).
% Если тестовый шаблон отсутствует, неисправность называется \textit{избыточной} (\textit{redundant}).
% Проблема классификации неисправности как тестируемой или избыточной является NP-полной.
% Задача ATPG заключается в классификации всех неисправностей и создании набора тестовых шаблонов, содержащего хотя бы один тестовый шаблон для каждой испытуемой неисправности.

% % TODO: \subsubsection{Секвенциальный ATPG}

% Генерация тестовых шаблонов для схем, содержащих элементы состояния (памяти), такие как триггеры (\textit{flip-flops}), вычислительно более сложна, потому что элементы памяти не могут быть непосредственно установлены в определенное значение.
% Вместо этого поведение схемы во времени должно рассматриваться во время ATPG.
% Было предложено ряд инструментов, которые непосредственно решают эту секвенциальную проблему, например, HITEC [NP91].
% Но на практике получаемая модель часто слишком сложна для обработки с помощью инструментов ATPG.
% Поэтому обычно рассматривается полный режим сканирования для преодоления этой проблемы путем подключения всех элементов состояния в цепочку сканирования [WA73, EW77].
% В режиме тестирования цепочка сканирования объединяет все элементы состояния в сдвиговый регистр, в нормальном режиме работы элементы состояния управляются обычной логикой в схеме.
% В результате элементы состояния могут рассматриваться как основные входы и выходы для тестирования, и получается комбинационная постановка задачи ATPG, уже рассмотренная выше.

% TODO: \subsubsection{Классические алгоритмы ATPG}


\section{Задача генерации тестовых шаблонов для верификации булевых схем}
\label{sec:atpg}

Задача автоматической генерации тестовых шаблонов (Automatic Test Pattern Generation \--- ATPG) тесно связана с задачей проверки эквивалентности LEC~\cite{drechsler2021}.
Вкратце, она вытекает из следующего контекста.
В~процессе производства цифровых схем могут возникать некоторые дефекты.
В~результате дефекта один (или несколько) логических элементов (гейтов) в схеме могут \enquote{застрять} \--- стать постоянными и выдавать только~$0$ или только~$1$ для любой комбинации входов.
Эта модель дефектов называется в литературе \textit{моделью застревания на уровне} (\textit{stuck-at-fault}, а~конкретнее, \textit{stuck-at-zero} или \textit{stuck-at-one}, в зависимости от значения, на котором \enquote{застрял} гейт).
Возможно, что в некоторых случаях получившаяся \enquote{дефектная} схема будет эквивалентна оригинальной.
Однако проблема возникает тогда, когда её поведение отличается от оригинальной схемы.
К~счастью, проверка на наличие застреваний на уровне в элементах достаточно проста с помощью специальных тестов (наборов входных значений), нацеленных на выявление таких дефектов.
Однако генерация этих тестов является весьма нетривиальной задачей.

Пусть $S_f$ \--- это оригинальная схема, определяющая функцию $f \colon \{0,1\}^n \to \{0,1\}^m$.
Обозначим через~$S_{[f,v,\delta]}$ схему, полученную из~$S_f$ заменой элемента~$v$ на константу $\delta \in \{0,1\}$; эта схема определяет некоторую функцию $f' \colon \{0,1\}^n \to \{0,1\}^m$, и, следовательно, будем использовать обозначение $S_{f'} = S_{[f,v,\delta]}$.
Назовём элемент~$v'$ в схеме~$S_{f'}$, который застрял на уровне~$\delta$, \textit{образом застревания элемента}~$v$.
Если~$S_f \cong S_{f'}$, то ошибка в элементе~$v$ не является критической.
Предположим, что для определённого значения $\delta$ схемы $S_f$ и~$S_{f'}$ не эквивалентны.
Тогда существует такой вход $\alpha_\delta \in \{0,1\}^n$, что $\gamma_\delta \neq \gamma_{\delta'}$, где $\gamma_\delta = f(\alpha_\delta)$, $\gamma_\delta' = f'(\alpha_\delta)$.
Тогда $(\alpha_\delta, \gamma_\delta, \gamma_\delta')$ представляет собой эффективно проверяемый сертификат застревания элемента~$v$ на уровне $\delta$.
Мы будем называть множество таких сертификатов для всех элементов \textit{полным набором тестов} для рассматриваемой схемы (в контексте модели застревания на уровне).
Задача построения такого набора естественным образом называется задачей автоматической генерации тестовых шаблонов (ATPG).
Из приведенного описания ясно, что для построения полного набора тестов для~$S_f$ необходимо решить $2 \cdot K$ задач LEC, где $K$ \--- это число элементов в~$S_f$.

Предположим, что для оригинальной схемы~$S_f$ и конкретного элемента~$v$ построена схема~$S_{f'}$.
Затем для этих двух схем строится \textit{miter}-схема $S_{f \oplus f'}$ и рассматривается выполнимость КНФ~$C_{f \oplus f'}$.
Однако из формулировки задачи ATPG видно, что $S_f$ и~$S_{f'}$ могут быть очень похожими.
Учитывая этот факт, можно существенно упростить задачу LEC в форме SAT для этих двух схем.
Описанная ниже процедура хорошо известна в области автоматического дизайна цифровых схем (Electronic Design Automation \--- EDA).

Пусть $V$ и $\tilde{V}$ \--- это множества вершин схем $S_f$ и~$S_{f'}$, соответственно.
Существует естественная взаимно-однозначная соответствие~$\theta$ между $V$ и~$\tilde{V}$: неформально, зафиксируем один и тот же порядок относительно топологической сортировки на~$V$ и~$\tilde{V}$ и предположим, что $\theta(v) = \tilde{v}$, если вершины $v$ и~$\tilde{v}$ имеют одинаковый номер.
Далее будем говорить, что вершины $v$ и~$\tilde{v}$, где $\tilde{v} = \theta(v)$, называются \textit{одноименными}.

Заметим, что после объединения входов схем $S_f$ и $S_{f'}$ при переходе к схеме $S_{f\glue f'}$, последняя схема может содержать элементы, которые также можно объединить.
Действительно, предположим, что элементы $v \in S_f$ и~$\tilde{v} \in S_{f'}$ имеют одинаковые координаты \--- им назначены одни и те же булевы функции и они имеют одних и тех же родителей~$P_v = P_{\tilde{v}}$.
В~этом случае можно удалить один из этих элементов и ассоциировать с оставшимся элементом потомков удаленного.
В~графах эта операция аналогична процедуре, используемой для объединения вершин при построении ROBDD (Reduced Ordered Binary Decision Diagram)~\cite{bryant1986}, и может быть эффективно выполнена с использованием хеш-таблиц.
Если схемы $S_f$ и~$S_{f'}$ не сильно отличаются, то при переходе к LEC для этих схем в форме SAT целесообразно объединить все элементы, для которых это возможно в соответствии с вышеуказанным.

Для произвольного элемента $v \in V \setminus \Vin$ рассмотрим все пути, соединяющие~$v$ с вершинами из~$\Vout$, и обозначим через~$D_v$ множество всех вершин, через которые проходят эти пути (включая~$v$).
Назовем~$D_v$ \textit{тенью} вершины~$v$.

\begin{lemma}\label{lem2}
    Пусть $v \in V \setminus \Vin$ \--- произвольный элемент, а $v'$ \--- образ застревания элемента~$v$ на уровне.
    При применении к схемам $S_f$ и~$S_{f'}$ описанной выше процедуры будут объединены все одноименные вершины, кроме (возможного) множества одноименных вершин, которые лежат в тенях $D_v$ и~$D_{v'}$.
\end{lemma}

Доказательство этого утверждения следует непосредственно из определения процедуры объединения и факта, что~$v \neq v'$ в смысле эквивалентности соответствующих координат.

Обозначим через $\tilde{S}_{f \glue f'}$ схему, полученную из~$S_{f \glue f'}$ объединением всех возможных вершин с использованием описанной выше процедуры.
Заметим, что некоторые одноименные выходы схем $S_f$ и~$S_{f'}$ также могут быть объединены.
Построим для~$\tilde{S}_{f \glue f'}$ её шаблонную КНФ~$\tilde{C}_{f \glue f'}$.
Обозначим через~$\tilde{C}(\mathcal{M})$ КНФ-формулу, представляющую (в вышеупомянутом смысле) \textit{miter} для одноименных выходов всех выходных переменных $S_f$ и~$S_{f'}$, которые не были объединены.
Тогда справедлива следующая теорема.

\begin{theorem}\label{thm:atpg}
    Если формула $\tilde{C}_{f \glue f'} \land \tilde{C}(\mathcal{M})$ выполнима, то, решив SAT для этой формулы, получим триплет $(\alpha_{\delta}, \gamma_{\delta}, \gamma'_{\delta})$ для обнаружения соответствующего дефекта в элементе~$v$.
\end{theorem}

\begin{proof}
    Подобно многим другим фактам, устанавливающим взаимосвязь между свойствами схем и формулами, построенными для этих схем, это доказательство основано на Лемме~\ref{lem1}.
    Из неё следует, что число присваиваний, удовлетворяющих КНФ~$\tilde{C}_{f \glue f'}$, равно~$2^n$, так как каждый входной вектор $\alpha \in \{0,1\}^n$ вызывает (в~смысле Леммы~\ref{lem1}) одно присваивание, удовлетворяющее $\tilde{C}_{f \glue f'}$.
    Легко видеть, что для любого присваивания, удовлетворяющего~$\tilde{C}_{f \glue f'}$, существует единственный вход~$\alpha \in \{0,1\}^n$, который его вызывает.

    Как следует из Леммы~\ref{lem2}, некоторый входной вектор~$\alpha \in \{0,1\}^n$ может вызвать различные значения одноименных выходов только для выходных вершин, лежащих в тенях $D_v$ и~$D_{v'}$.
    Пусть $y$ и~$y'$ \--- переменные, назначенные таким выходным вершинам.
    Предположим, что некоторый~$\alpha \in \{0,1\}^n$ вызывает различные значения $y$ и~$y'$.
    В~этом случае такое присваивание также вызовет присваивание, удовлетворяющее формулу $\tilde{C}_{f\glue f'} \land \tilde{C}(\mathcal{M})$, что приведет к соответствующему триплету $(\alpha_\delta, \gamma_\delta, \gamma_\delta')$ для обнаружения дефекта в элементе~$v$.
    В~противном случае, если каждое $\alpha \in \{0,1\}^n$ вызывает присваивание, в котором каждая пара выходных элементов с одинаковым номером имеют одинаковые значения, то формула $\tilde{C}(\mathcal{M})$ примет значение \texttt{False} на соответствующем присваивании, и таким образом, формула $\tilde{C}_{f \glue f'} \land \tilde{C}(\mathcal{M})$ будет невыполнима.
\end{proof}


\section{Задача булевой выполнимости}
\label{sec:sat}

Задача выполнимости булевых формул (Boolean satisfiability problem \--- SAT) формулируется следующим образом~\cite{handbook-sat}: для произвольной булевой формулы $\phi(x_1, \dots, x_n)$ необходимо определить, существует ли подстановка значений переменных~$X_\text{SAT}$, при которой формула становится истинной:
\[
    \exists X_\text{SAT} \in \{0,1\}^n : \phi(X_\text{SAT}) = 1 %.
\]
Если такая подстановка $X_\text{SAT}$ существует, то она называется \textit{удовлетворяющей} (\textit{satisfying assignment}; также используются термины \textit{модель} и \textit{интерпретация}), а формула~$\phi$ называется \textit{выполнимой} (\textit{SATisfiable}).
В~противном случае, если удовлетворяющая подстановка не существует, $\phi$~называется \textit{невыполнимой} (\textit{UNSATisfiable}).

% Задача SAT является первой задачей, для которой была доказана NP-полнота~\cite{cook}.
% \todo{Описание универсальности задачи SAT}

% \subsection{Базовые определения}
% \label{sub:sat-definitions}

% \todo{Литерал, дизъюнкция, конъюнкция, КНФ, ДНФ}

Если булева формула~$\phi$ представлена в конъюктивной нормальной форме (КНФ), то соответствующую задачу называют CNF-SAT.
Любая булева формула может быть преобразована в эквивалентную КНФ, однако при этом размер формулы может увеличиться экспоненциально, например:
\[
    \text{$n$ конъюнкций}
    \left\{
    \begin{aligned}
        & (x_1 \land y_1) \lor \\
        & (x_2 \land y_2) \lor \\
        & ~\dots \\
        & (x_n \land y_n)
    \end{aligned}
    \right.
    \quad
    \xRightarrow{\text{КНФ~}}
    \quad
    \left.
    \begin{aligned}
        & (x_1 \lor x_2 \lor \dotsb \lor x_n) \land \\
        & (y_1 \lor x_2 \lor \dotsb \lor x_n) \land \\
        & ~\dotsb \\
        & (y_1 \lor y_2 \lor \dotsb \lor y_n)
    \end{aligned}
    \right\}
    \text{$2^n$ дизъюнкций}
\]
С~помощью преобразований Цейтина~\cite{tseitin1970} возможно привести любую булеву формулу в КНФ \--- с сохранением выполнимости (\textit{equisatisfiable CNF}), но с добавлением новых переменных (\textit{auxiliary variable}) \--- при этом размер формулы увеличится лишь линейно.
В данной работе подразумевается, что все булевы выражения, кодирующие задаваемые ограничения, представляются в виде КНФ \--- либо подвергаются эквивалентным логическим преобразованиям, либо преобразованиям Цейтина.


\section{Основные алгоритмы решения SAT}

С учетом сказанного в предыдущем разделе, везде далее под задачей булевой выполнимости (SAT) в распознавательном варианте понимается задача распознавания выполнимости произвольной булевой формулы в КНФ.
SAT является исторически первой NP-полной задачей.
Данный факт был установлен С.\,А.~Куком (без привлечения точного определения NP-полноты) в статье~\cite{cook}, которая является основополагающей работой для структурной теории сложности алгоритмов.
Помимо распознавательного варианта далее нас будет интересовать поисковый вариант SAT \--- когда требуется распознать выполнимость КНФ, и в случае ее выполнимости найти произвольный выполняющий набор.
Данная задача, соответственно, NP-трудна.
Сказанное означает, что \--- в предположении $P \neq NP$ \--- SAT не может быть решена в общем случае за полиномиальное время.
При всем этом существует масса разнообразных аргументов в пользу того, что SAT (как и многие другие NP-трудные задачи) не является сложной в большинстве своих частных случаев.
Именно этот факт позволяет использовать современные алгоритмы решения SAT в задачах, комбинаторная размерность которых может быть колоссальной.
Так, в символьной верификации удается успешно решать SAT в отношении КНФ, включающих миллионы дизъюнктов.
Число переменных, встречающихся в таких КНФ, может исчисляться десятками и сотнями тысяч.
Существуют два больших класса алгоритмов решения SAT \--- полные и неполные.
Неполные алгоритмы плохо подходят или не подходят совсем для решения задач, в которых требуется доказывать невыполнимость (соответственно, например, для символьной верификации).
Однако они вполне могут использоваться для обращения функций.

В настоящей работе основным вычислительным инструментом являются полные алгоритмы решения SAT.
Именно такие алгоритмы позволяют точно решать задачи верификации автоматов и булевых схем \--- находить решение, если оно существует, либо гарантировать его отсутствие при заданных параметрах, что в том числе позволяет точно решать задачу минимизации.


\subsection{Неполные алгоритмы решения SAT}

Неполные алгоритмы решения SAT не гарантируют ответ SAT/UNSAT за конечное время для произвольной КНФ.
Существует целый ряд различных концепций, лежащих в основе таких алгоритмов.
Статья~\cite{kautz2009} представляет собой детальный обзор по данному вопросу, содержащий ключевые ссылки.
В~дальнейшем нас будут интересовать только те неполные алгоритмы решения SAT, в основе которых лежит идеология локального поиска или (в~отдельных случаях) эволюционные стратегии.
В~таких алгоритмах задача SAT в отношении КНФ~$C$ над множеством из $n$~переменных, состоящей из $m$~дизъюнктов, рассматривается в форме задачи максимизации псевдобулевой функции следующего вида:
\begin{equation}\label{eq:eq9}
    f_{C} \colon \{ 0,1 \}^{n} \to \{ 0,1,\dots,m \}
\end{equation}
Псевдобулевой называется~\cite{boros2002} любая функция следующего вида:
\begin{equation}\label{eq:eq10}
    f \colon \{ 0,1 \}^{n} \to \mathbb{R}
\end{equation}

На произвольном наборе $\alpha \in \{ 0,1 \}^{n}$ значение функции~\eqref{eq:eq9} равно числу дизъюнктов в~$C$, которые на этом наборе обращаются в единицу.
Фактически в данном случае рассматривается задача SAT в оптимизационной постановке \--- известная как MaxSAT.

Рассматривая $\{ 0,1 \}^{n}$ в роли пространства поиска, можно ввести на нем \textit{функцию окрестностей}~\cite{burke2014} $\aleph \colon \{ 0,1 \}^{n} \to 2^{\{ 0,1 \}^{n}}$.
Проще всего для этой цели использовать метрику Хэмминга~\cite{macwilliams2007}, в рамках которой для произвольной точки~$\alpha \in \{ 0,1 \}^{n}$ ее окрестность Хэмминга радиуса~$r \geq\nobreak 1$, определяется следующим образом:
\begin{equation}\label{eq:eq11}
    \aleph_{r}(\alpha) = \Set{
        \alpha' \in \{ 0,1 \}^{n}
        \given
        \rho_{H}(\alpha,\alpha') \leq r
    }
\end{equation}
В~\eqref{eq:eq11} через $\rho_{H}(\alpha,\alpha')$ обозначено расстояние Хэмминга между словами $\alpha$ и~$\alpha'$.
Чаще всего рассматриваются окрестности радиуса~1.
В~такой постановке для решения SAT и MaxSAT может использоваться, без преувеличения, огромный арсенал методов локального поиска.
Проиллюстрируем общую идею, лежащую в основе таких методов, на примере простейшего алгоритма, известного как \enquote{восхождение к вершине} (\textit{Hill Climbing} \--- далее HC)~\cite{russell2021}.
Опишем вариант HC, применимый для максимизации произвольной псевдобулевой функции вида~\eqref{eq:eq10}.

\begin{enumerate}
    \item Выбираем (вообще говоря, произвольным образом, например, случайно в соответствии с равномерным распределением на $\{ 0,1 \}^{n}$) стартовую точку~$\alpha_{0} \in \{ 0,1 \}^{n}$, вычисляем~$f(\alpha_{0})$; считаем~$\alpha_{0}$ текущей точкой, а~$f(\alpha_{0})$ текущим значением~$f$.

    \item Пусть $\alpha \in \{ 0,1 \}^{n}$ \--- текущая точка.

    Обходим в некотором порядке $\aleph_{1}(\alpha) \setminus \{ \alpha \}$, вычисляя для каждой точки $\alpha'$ из данного множества $f(\alpha')$.
    Если найдена такая точка $\alpha' \in \aleph_{1}(\alpha) \setminus \{ \alpha \}$, что $f(\alpha') > f(\alpha)$, перейти на шаг~3, в противном случае перейти на шаг~4.

    \item $\alpha \gets \alpha'$, $f(\alpha) \gets f(\alpha')$, перейти на шаг~2.

    \item $(\alpha,f(\alpha))$ \--- локальный максимум функции~$f$ на $\{ 0,1 \}^{n}$ (поскольку для любой $\alpha' \in \aleph_{1}(\alpha) \setminus \{\alpha\}$ имеет место $f(\alpha') \leq f(\alpha)$).
    В~этом случае алгоритм либо останавливается и выдает в качестве ответа $(\alpha, f(\alpha))$, либо запускает некоторую процедуру выхода из локального максимума.
\end{enumerate}

Для выхода из локальных экстремумов можно дополнять приведенный выше базовый алгоритм различными техниками, основанными на эвристических и метаэвристических соображениях.
Зачастую, выйдя из точки локального экстремума, можно оказаться в точке с худшим значением~$f$.
Такого сорта \enquote{выпрыгивания} из локальных экстремумов могут осуществляться в процессе поиска неоднократно.
В~этом случае обычно хранится точка с самым лучшим значением функции~$f$, достигнутым за все историю поиска.
Такое значение называется \textit{рекордом} (\textit{best known value} \--- BKV).
Современные техники выхода из локальных экстремумов позволяют даже при решении весьма трудных задач многократно улучшать рекорд в процессе поиска.
Если некоторое число попыток выхода из локальных экстремумов не дает улучшения текущего рекорда~$f^{*}$, достигнутого в точке~$\alpha^{*}$, то алгоритм останавливается и выдает в качестве ответа пару $(\alpha^{*}, f^{*})$.

Предположим, что Hill Climbing применяется к задаче максимизации произвольной функции вида~\eqref{eq:eq9}.
Легко понять, что даже если дополнить HC какой-либо процедурой выхода из точек локального максимума, это не позволит безошибочно распознавать невыполнимость невыполнимых КНФ.
С~другой стороны, если рассматривается КНФ с большим числом выполняющих наборов, то HC (даже в самом простом варианте) может случайно натолкнуться на такой набор за приемлемое время.
Известный пример данного типа \--- успешное использование HC для решения SAT в отношении КНФ, кодирующих задачу размещения $k$~ферзей на шахматной доске размерности $k \times k$~\cite{gu1992}.
Однако, к сожалению, HC и известные его модификации напрямую неприменимы к КНФ, кодирующим обращение интересных с практической точки зрения криптографических функций.
Причем это верно даже в отношении КНФ с огромным числом выполняющих наборов \--- например, для КНФ, кодирующих задачи обращения криптографических хеш-функций.

Если некоторый алгоритм локального поиска, решающий задачу максимизации функции вида~\eqref{eq:eq9}, слишком долго не может улучшить текущий рекорд $(\alpha, f_{C}(\alpha))$, где $f_{C}(\alpha) < m$, то данный алгоритм можно остановить с ответом UNSAT в отношении КНФ~$C$.
Этот ответ может оказаться ошибочным.
Однако существуют специальные техники рандомизации локального поиска, использование которых позволяет оценивать вероятность ошибки указанного типа и даже снижать эту вероятность за счет выполнения алгоритмом большого числа некоторых случайных шагов.
Один из самых известных примеров такого рода \--- алгоритм, предложенный У.~Шёнингом в~\cite{schoning1999}.
Данный алгоритм решает задачу о выполнимости произвольной $k$-КНФ \--- такой КНФ, где каждый дизъюнкт которой имеет длину~$k \geq 2$.
Алгоритм Шёнинга пытается улучшить значение функции~\eqref{eq:eq9}, достигнутое в точке~$\alpha \in \{ 0,1 \}^{n}$, за счет случайного выбора и модификации тех дизъюнктов в КНФ~$C$, которые обращаются в ноль на наборе~$\alpha$.
Получаемый в результате процесс интерпретируется в рамках хорошо изученной в теории вероятностей модели случайных блужданий~\cite{feller1971}.
Как результат, если алгоритм Шёнинга, сделав $\mathcal{O}(n \cdot (2 - \frac{2}{k})^{n})$ простых случайных действий, не находит выполняющий набор, то вероятность ошибиться, заключив, что $C$~невыполнима, не превосходит $1 - \frac{1}{\fun{poly}(n)}$, где через $\fun{poly}(\cdot)$ обозначен некоторый полином.
Если повторение упомянутого выше списка действий, допустим, ${n \cdot p(n)}$ раз не дает выполняющий набор, то заключение о невыполнимости~$C$ будет справедливым с вероятностью, которая близка к ${1 - e^{-n}}$.
Идеи, лежащие в основе алгоритма Шёнинга, позволили для SAT в отношении 3\=/КНФ построить нетривиальные верхние оценки сложности, которые долгое время оставались лучшими среди аналогичных по смыслу оценок (см.~\cite{dantsin2009}).

Алгоритмы, в которых базовые схемы локального поиска (например, HC) дополняется различными стратегиями рандомизации, относятся к классу, известному как \textit{Stochastic Local Search methods} (SLS).
К~сожалению, имеющиеся на сегодняшний день SLS-алгоритмы плохо подходят для обращения криптографических функций.
Как будет показано далее, лучшие такие алгоритмы позволяют успешно решать задачи криптоанализа лишь весьма простых генераторов ключевого потока.

\subsection{Полные алгоритмы решения SAT}

Алгоритм решения SAT называется полным, если для любой КНФ~$C$ он за конечное число шагов выдает верный ответ вида SAT/UNSAT.
Как и в ситуации с неполными, для построения полных алгоритмов решения SAT можно использовать множество различных базовых концепций.
Так, довольно естественно решать SAT, основываясь на широко применяемой в комбинаторной оптимизации идеологии ветвей, границ и отсечений~\cite{papadimitriou1982}.
Хорошо известна сводимость SAT к задаче 0\=/1 целочисленного линейного программирования (0\=/1\=/ЦЛП), после осуществления которой можно использовать для решения полученной задачи из семейства 0\=/1\=/ЦЛП богатый набор программных средств.
Также довольно просто перейти от SAT к задаче поиска решений системы алгебраических уравнений степени не выше двух над полем $\mathrm{GF}(2)$.
К~таким системам можно применять алгоритм Б.~Бухбергера (он же \enquote{метод баз Грёбнера})~\cite{buchberger2006}, а также специальные техники работы с разреженными квадратичными системами над $\mathrm{GF}(2)$ (см., например,~\cite{goos1999,courtois2002}).

Многие авторы сходятся во мнении, что лучших результатов в решении трудных вариантов SAT из целого ряда областей, среди которых символьная верификация и криптоанализ, удается добиться за счет использования алгоритмов, относящихся к направлению, которое правильнее всего назвать \enquote{Вычислительная логика}.
Ранние алгоритмы из данного класса лежали в основе первых программных реализаций систем автоматического доказательства теорем (Automated Theorem Proving \--- ATP).
Одним из таких алгоритмов был \enquote{метод Девиса-Патнема} (Davis-Putnam method)~\cite{davis1960}.
Усовершенствованная версия данного алгоритма, известная как DPLL (от фамилий Davis, Putnam, Logemann, Loveland)~\cite{davis1962}, до настоящего момента продолжает использоваться в основе высокоэффективных полных SAT-решателей.
DPLL представляет собой обход дерева поиска, в рамках которого выход из тупиковых ветвей организован в форме процедуры, называемой \textit{хронологическим бэктрекингом} (\textit{chronological backtracking}).
Остановимся подробнее на тех деталях DPLL, которые потребуются нам в дальнейшем.

\subsection{Алгоритм DPLL}

Пусть $C$ \--- произвольная КНФ над множеством переменных~$X$.
Пусть $S \subset\nobreak L_{X}$ \--- произвольное множество литералов над переменными из~$X$, которое не содержит контрарных литералов.
Пусть $X_{S}$ \--- множество, включающее все те переменные из~$X$, литералы над которыми содержатся в~$S$, то есть $X_{S} \subseteq X$.
Рассмотрим отображение $\sigma \colon X_{S} \to \{ 0,1 \}$, заданное по следующему правилу:
\begin{equation}\label{eq:eq12}
    \sigma(x \in X_{S}) = \begin{cases}
        1 &\text{если } x \in S \\
        0 &\text{если } \neg x \in S
    \end{cases}
\end{equation}

Иными словами, отображение~\eqref{eq:eq12} связывает с выбираемыми из~$L_{X}$ литералами значения соответствующих переменных: выбор литерала~$x$ интерпретируется как присвоение переменной~$x$ значения~1, выбор же~$\neg x$ соответствует принятию~$x$ значения~0.
Множество~$S$ будем называть списком литералов, выбранных из~$L_{X}$.
Теперь опишем собственно алгоритм DPLL.

На начальном шаге список выбранных литералов пуст.
Выберем (вообще говоря, произвольный) литерал~$l_{1} \in L_{X}$, поместим его в список~$S_{1}$ и рассмотрим КНФ $l_{1} \land C$.
Удалим из данной КНФ каждый дизъюнкт вида $(l_{1} \lor D)$, а из каждого дизъюнкта вида $(\neg l_{1} \lor D)$ удалим литерал $\neg l_{1}$ (здесь через~$D$ обозначен произвольный непустой дизъюнкт над~$X$).
Обозначим результирующую КНФ через~$C'$.
Будем говорить, что данная КНФ получена из~$l_{1} \land C$ в результате применения правила \textit{единичного дизъюнкта} (Unit Propagation rule \--- UP~\cite{dowling1984}) к~$C$ и литералу~$l_{1}$.
Заметим, что если в~$C$ содержался дизъюнкт вида $D = (\neg l_{1} \lor l')$, то удаление из~$D$ литерала~$\neg l_{1}$ дает единичный дизъюнкт, состоящий из литерала~$l'$.
В~описанной ситуации литерал~$l_{1}$ называют \textit{угаданным}, а про литерал~$l'$ говорят, что он был \textit{выведен по правилу единичного дизъюнкта}.
К~КНФ~$C'$ и литералу~$l'$ можно снова применить правило единичного дизъюнкта.
Если в результате применения UP вывелось несколько литералов, они все ставятся в очередь, после чего к ним и соответствующим КНФ последовательно применяется UP.

Пусть $S_{k} = \{ l_{1},\ldots,l_{k} \}$ \--- список угаданных литералов.
Обозначим через ${\widetilde{S}}_{k}$ список всех литералов, которые были выведены по правилу UP в соответствии с описанной выше процедурой.
Пусть $C_{k}$ \--- полученная в результате КНФ.
Проанализируем несколько ситуаций, которые могут при этом возникнуть.

\begin{enumerate}
    \item Пусть ${\widetilde{S}}_{k}$ не содержит контрарных литералов, а~$C_{k}$~содержит только единичные дизъюнкты.
    Тогда $C$~выполнима.
    Следовательно, существует выполняющий~$C$ набор, в котором значения части (или всех) переменных определяются при помощи отображения~\eqref{eq:eq12}, применяемого к литералам из~$S_{k} \union {\widetilde{S}}_{k}$.

    \item Пусть ${\widetilde{S}}_{k}$ не содержит контрарных литералов, а~$C_{k}$~содержит дизъюнкты длины не менее~2.
    Пусть ${\widetilde{C}}_{k}$ \--- КНФ, составленная из этих дизъюнктов, а ${\widetilde{X}}_{k}$ \--- множество переменных, встречающихся в ${\widetilde{C}}_{k}$.
    Выберем из~$L_{{\widetilde{X}}_{k}}$ произвольный литерал~$l_{k + 1}$, построим список $S_{k + 1} = S_{k} \union \{ l_{k + 1} \}$ и~применим UP к~${\widetilde{C}}_{k}$ и~$l_{k + 1}$.

    \item Список ${\widetilde{S}}_{k}$ содержит контрарные литералы \--- пару вида $(l,\neg l)$ для некоторого~$l \in L_{X}$.
    В~этой ситуации будем говорить, что список угаданных литералов~$S_{k}$ породил конфликт.
    Сам по себе конфликт еще не означает, что исходная КНФ невыполнима \--- вполне возможно, что она выполнима, но были угаданы такие литералы, что сочетание соответствующих им в~смысле отображения~\eqref{eq:eq12} значений переменных не встречается ни в одном из выполняющих~$C$ наборов.
\end{enumerate}

Пусть $S_{k} = \{ l_{1},\ldots,l_{k} \}$ \--- список угаданных литералов, такой что после угадывания~$l_{k}$ по UP был выведен конфликт.
В~этой ситуации перейдем от списка~$S_{k}$ к списку $S_{k}' = \{ l_{1},\ldots,\neg l_{k} \}$, помечая литерал~$\neg l_{k}$ как \enquote{инвертированный}.
Предположим, что для некоторого~$k$ оба списка $S_{k} = \{ l_{1},\ldots,l_{k} \}$ и $S_{k}' = \{ l_{1},\ldots,{\neg l}_{k} \}$ порождают конфликты.
Пусть $l_{r}$ \--- ближайший предшествующий $l_{k}$ литерал в списке $S_{k}$, который ранее не был инвертирован.
Тогда новым списком является $S_{r}' = \{ l_{1},\ldots,{\neg l}_{r} \}$ (везде здесь предполагалось, что $k,r \geq 2$).

Описанная выше процедура перехода к списку $S_{r}'$ называется хронологическим (обычным) бэктрекингом. Можно заметить, что процесс бэктрегинга соответствует обходу с возвратом бинарного дерева специального вида. Вершинам этого дерева приписаны переменные из $X$. Из произвольной вершины, которой приписана переменная~$x$, выходит два ребра, символизирующие литералы $x$ и~$\neg x$.
Корню данного дерева приписана переменная, над которой берется литерал~$l_{1}$.
Произвольная ветвь соответствует некоторому списку угаданных литералов.
Если такой список порождает конфликт, то соответствующую ветвь назовем тупиковой.

Если для некоторого~$k$ списки $S_{k}$ и~$S_{k}'$ порождают конфликты, и при этом все литералы, предшествующие~$l_{k}$, включая первый, были инвертированы, то каждая ветвь описанного выше дерева поиска является тупиковой.
Легко понять, что данный факт означает невыполнимость КНФ~$C$.
Также в силу всего сказанного выше можно заметить, что число вершин в таком дереве поиска не превосходит $M = 2^{n + 1} - 1$.
Отметим, что в реальности это число может быть существенно меньше за счет большого числа литералов, выведенных по~UP.
Таким образом, применив UP не более $M$~раз, либо достигнем ситуации, описанной в пункте~1, и это будет означать, что $C$~выполнима, либо докажем невыполнимость~$C$.
Все сказанное означает полноту алгоритма DPLL.

\subsection{Концепция CDCL и базирующиеся на ней современные SAT-решатели}

Концепция решения SAT, известная сегодня как Conflict Driven Clause Learning (CDCL), включает в себя ряд важных техник, дополняющих алгоритм DPLL.
Первая и главная из них позволяет записывать информацию о конфликте, который возник в процессе обхода дерева поиска, в виде специальным образом построенного дизъюнкта.
Такой дизъюнкт называется \textit{конфликтным} (conflict-induced clause).
Если $C$ \--- исходная КНФ, а $D$ \--- конфликтный дизъюнкт, то имеет место $C \implies D$, то есть $D$ \--- это логическое следствие (импликация) из~$C$.
Соответственно, КНФ~$C$ выполнима тогда и только тогда, когда выполнима КНФ~$C \land D$.

Конфликтные дизъюнкты можно строить различными способами.
Приведем простейший пример.
Пусть список угаданных литералов $S_{k} = \{ l_{1},\ldots,l_{k} \}$ породил конфликт в рамках алгоритма DPLL.
Построим следующий конфликтный дизъюнкт: $D =\nobreak (\neg l_{1} \lor\nobreak \ldots \lor\nobreak \neg l_{k})$.
Данный дизъюнкт запрещает одновременный выбор всех литералов из списка~$S_{k}$.
Рассмотрим КНФ $C \land D$ ($C$ \=== исходная КНФ).
Если при применении к КНФ $C \land D$ алгоритма DPLL используется список угаданных литералов $S_{k - 1} = \{ l_{1},\dots,l_{k - 1} \}$, то единичный дизъюнкт~$\neg l_{k}$ будет выведен по правилу~UP.
В~этом случае говорят, что вывод литерала~$\neg l_{k}$ индуцирован конфликтом.
Очень важно, что в данном случае литерал~${\neg l}_{k}$ не угадывается, а выводится по правилу UP на основе информации, полученной в результате анализа конфликта.

Впервые идея использовать конфликтные дизъюнкты для записи информации о тупиковых ветвях в DPLL-поиске была предложена в конференционной статье Ж.~Маркеша-Сильвы и К.~Сакаллы в 1996 году~\cite{grasp}.
В более детальном виде она была представлена этими же авторами в журнальной статье~\cite{marques-silva1999}.
По сути, именно в этих двух работах были заложены основы концепции CDCL.
Еще одна важная техника, которая была описана в~\cite{grasp,marques-silva1999}, заключается в использовании для представления процесса решения SAT специальных графов, называемых \textit{графами вывода} (\textit{implication graph} \--- IG).
Граф вывода позволяет эффективно выявлять литералы (причем, что важно, как угаданные, так и выведенные по~UP), которые ответственны за рассматриваемый конфликт.
Графы вывода очень информативны.
Разные способы обхода IG соответствуют различным эвристикам формирования конфликтных дизъюнктов. Некоторые такие эвристики также были приведены в~\cite{grasp,marques-silva1999}.

Основное концептуальное отличие CDCL~от DPLL заключается в том, что CDCL~использует память для хранения информации о ходе поиска в форме конфликтных дизъюнктов.
Это позволяет вместо хронологического бэктрекинга в ряде случаев осуществлять \textit{нехронологический бэктрекинг} (\textit{non-chronological backtracking}), называемый также \textit{бэкджампингом} (backjumping).
Бэкджампинг \--- это ситуация, когда после анализа конфликта откат в списке угаданных литералов происходит не к ближайшему (от конфликта) литералу, который не был ранее инвертирован, а к угаданному еще раньше (иногда существенно раньше).
Во многих случаях бэкджампинг позволяет эффективно отсекать значительные части дерева поиска, запрещая при помощи конфликтных дизъюнктов последующий поиск в этих областях.
Первым SAT-решателем, фактически использующим CDCL, был GRASP~\cite{grasp}.

Следующий шаг был сделан в работах~\cite{moskewicz2001,zhang2001}, где были введены еще несколько важных техник, дополняющих базовый CDCL.
Так, в~\cite{moskewicz2001} был описан механизм выбора порядка угадывания литералов, основанный на их \enquote{мере конфликтности}.
Соответствующая эвристика получила называние VSIDS (Variable State Independent Decaying Sum).
Также в~\cite{moskewicz2001} была описана весьма эффективная техника итеративного применения правила UP, использующая так называемые \enquote{ленивые} (\textit{lazy}) структуры данных, известные как \textit{watched literals}, применяются в настоящее время в большинстве CDCL SAT-решателей.
Еще одно важное достижение~\cite{moskewicz2001} \--- экспериментальная аргументация пользы рестартов.
В~статье~\cite{zhang2001} было проведено детальное исследование различных способов построения конфликтных дизъюнктов за счет анализа графов вывода.
На основе результатов работ~\cite{moskewicz2001,zhang2001} был построен решатель zchaff \--- первый по-настоящему высокоэффективный SAT-решатель, базирующийся на концепции CDCL.
С~использованием zchaff еще в 2002~году удавалось решать задачи криптоанализа некоторых поточных шифров существенно быстрее простого перебора.

В 2003 году в работе~\cite{minisat} были описаны общие принципы построения высокоскоростного CDCL SAT-решателя с эффективно модифицируемой архитектурой.
Исходный код соответствующего решателя, получившего название MiniSat, был представлен авторами~\cite{minisat} в открытом доступе.
Решатель MiniSat на протяжении многих лет остается де-факто стандартом программной основы эффективных SAT-решателей как широкого профиля, так и нацеленных на конкретную прикладную область.
Еще одной важной частью MiniSat стали процедуры периодической чистки баз конфликтных дизъюнктов.
Здесь следует отметить, что грамотно реализованный CDCL в процессе работы может генерировать огромные массивы конфликтной информации.
Чрезмерное число конфликтных дизъюнктов увеличивает число срабатываний правила UP и, как следствие, приводит к падению эффективности вывода.
Соответственно, часть конфликтной информации можно попытаться удалить.
Однако неудачное удаление конфликтных дизъюнктов может привести к их повторной генерации.
В данном контексте особенно ценны эвристики, которые позволяют удалять большое число слабо релевантных конфликтных дизъюнктов.
Первые относительно нетривиальные такие эвристики были предложены в статье~\cite{glucose}.

\begin{algorithm}[!ht]
    \caption{CDCL \--- расширенный алгоритм DPLL с анализом конфликтов и изучением дизъюнктов}
    \label{alg:cdcl}
    \DontPrintSemicolon
    \SetKwInput{Input}{Входные данные}
    \SetKwInput{Output}{Результат}
    \SetKwFunction{AnalyzeConflict}{АнализКонфликта}

    \Input{булева формула $F$, текущая модель $\sigma$ (изначально пустая)}
    \Output{SAT и удовлетворяющая модель~$\sigma$, либо UNSAT}

    \While{не все переменные назначены}{
        Выбираем неозначенную переменную $v$ \;
        \eIf{формула $F$ становится невыполнимой при $\sigma \cup \{v\}$}{
            $\beta \gets \AnalyzeConflict{$F, \sigma, v$}$ \;
            $F \gets F \cup \{\beta\}$ \;
            Возврат на предыдущий уровень решения \;
        }{
            Продолжение рекурсивного поиска \;
        }
    }

    \eIf{все переменные назначены}{
        \Return{$(\text{SAT}, \sigma)$}
    }{
        \Return{$\text{UNSAT}$}
    }
\end{algorithm}

Алгоритм CDCL (Conflict-Driven Clause Learning) представляет собой расширение классического алгоритма DPLL (Davis-Putnam-Logemann-Loveland), включающее в себя механизмы анализа конфликтов и изучение новых дизъюнктов.
Его псевдокод представлен в виде Алгоритма~\ref{alg:cdcl}.
Входными данными алгоритма являются булева формула~$F$ и (изначально пустая) модель~$\sigma$ (означивание переменных).
Алгоритм рекурсивно ветвится по переменным \--- выбирает очередную неназначенным переменную и пытается присвоить ей какое-то значение (можно выбирать это значение эвристически~\cite{moskewicz2001}; простейший же подход \--- всегда выбирать отрицательный литерал).
Если при добавлении означенной переменной в текущую модель~$\sigma$ формула $F$ становится невыполнимой \--- возникает конфликт, то выполняется анализ этого конфликта, в результате чего выводится новый дизъюнкт~$\beta$.
Этот дизъюнкт добавляется к формуле~$F$ для предотвращения повторения той же конфликтной ситуации в будущем, после чего происходит возврат на предыдущий уровень решений для продолжения поиска.
Если конфликтов не возникает, алгоритм продолжает рекурсивный поиск.
Процесс повторяется, пока не будут назначены все переменные, что приводит к нахождению удовлетворяющей модели.
При получении конфликта на нулевом уровне рекурсии алгоритм возвращает \textit{UNSAT} \--- доказывает невыполнимость исходной формулы~$F$.

Алгоритмы решения SAT, основанные на CDCL, оказались весьма удачно приспособленными для применения к ним различных концепций распараллеливания.
Основными двумя такими концепциями являются так называемый \enquote{портфолио-подход} (\textit{portfolio approach}) и \enquote{подход на основе разбиений} (\textit{partitioning approach}).
Детальное сравнение эффективности этих двух подходов предпринято в диссертации А.~Хиваринена~\cite{hyvarinen2011}.

Портфолио-подход предполагает запуск нескольких копий решателя на исходном пространстве поиска, при этом каждая копия начинает работу, используя некоторый набор значений входных параметров SAT-решателя (разным копиям соответствуют различные наборы значений параметров).
В~процессе работы копии решателей могут обмениваться друг с другом конфликтными дизъюнктами.
Для достижения высокой скорости такой обмен обычно организуется через оперативную память вычислительного устройства с использованием технологий многопоточного программирования.
Соответствующая техника получила название \textit{clause sharing} (\enquote{обмен конфликтными дизъюнктами}).
Одна из первых эффективных реализаций обмена конфликтными дизъюнктами была представлена в~\cite{hamadi2011}.

Подход на основе разбиений \textit{(SAT-partitioning)} предполагает разбиение пространства поиска (фактически, множества $\{ 0,1 \}^{n}$, где $n$ \--- число переменных в КНФ) на непересекающиеся подобласти, которые обрабатываются независимо друг от друга.
Данный подход позволяет организовать решение SAT в параллельной среде со слабо связанными или даже независимыми рабочими процессами (в~частности, в~грид-средах).
Как будет показано далее, подход на основе разбиений дает хорошие результаты при решении SAT-задач, кодирующих криптоанализ блочных и поточных шифров, поскольку естественным образом ассоциируется с атаками, относящимися к классу \enquote{угадывай и определяй} (\textit{guess and determine})~\cite{bard2009}.

Скажем здесь несколько слов по поводу теоретических аргументов эффективности CDCL.
Соответствующие результаты относятся к теории сложности пропозициональных доказательств.
В этой области исследуется задача доказательства невыполнимости невыполнимой формулы в КНФ.
Пусть $C$ \--- произвольная невыполнимая КНФ и $x_{C}$ \--- двоичное слово, представляющее~$C$ в некоторой \enquote{разумной} системе кодирования.
Пусть $\Sigma_{U} \subset \{ 0,1 \}^{*}$ \--- множество слов~$x_{C}$ по всем возможным невыполнимым КНФ~$C$.
Пусть $A$ \--- произвольный полный алгоритм решения SAT.
Любой такой алгоритм называется также системой пропозиционального доказательства (\textit{propositional proof system}).
Получив на вход~$x_{C}$, алгоритм~$A$ выдает двоичное слово~$s$, которое будем называть $A$\=/доказательством невыполнимости~$C$ (см., например,~\cite{razborov2002}).
Рассмотрим функцию $\omega_{A} \colon \{ 0,1 \}^{*} \to \{ 0,1 \}^{*}$, определенную следующим образом.
Если слово является $A$\=/доказательством невыполнимости некоторой КНФ~$C$, то $\omega_{A}$ сопоставляет этому слову слово~$x_{C}$.
В~противном случае выходом~$\omega_{A}$ является двоичный код символа~$\varnothing$.
Несложно понять, что для одной и той же КНФ~$C$ могут существовать различные $A$\=/доказательства ее невыполнимости (особенно хорошо это видно на примере метода резолюций).
С~произвольным $x_{C} \in \Sigma_{U}$ свяжем длину кратчайшего $A$\=/доказательства невыполнимости~$C$.
Можно заметить, что если для некоторого $A$ функция длины кратчайшего $A$\=/доказательства по всем $x_{C} \in \Sigma_{U}$ растет как полином от~$|x_{C}|$, то $\mathrm{NP} = \mathrm{coNP}$ (см.~\cite{cook1979}).
Однако для целого ряда алгоритмов несложно указать примеры бесконечных семейств невыполнимых КНФ с полиномиально растущей длиной кратчайшего $A$\=/доказательства на этих~КНФ.

Пусть теперь $A$ и $B$ \--- два полных алгоритма решения SAT.
Пусть $C$ \--- произвольная формула из $\Sigma_{U}$.
Если существует полиномиальный алгоритм, который произвольное $A$\=/доказательство невыполнимости~$C$ преобразует в $B$\=/доказательство ее невыполнимости, то говорят, что система доказательств~$B$ полиномиально моделирует систему доказательств~$A$.
Если $A$~полиномиально моделирует~$B$, а~$B$~полиномиально моделирует~$A$, то данные системы доказательств называются полиномиально эквивалентными.
Если на некотором (бесконечном) семействе противоречий ${\Sigma'}_{U} \subset \Sigma_{U}$ длина кратчайшего $A $\=/доказательства растет как полином от длины КНФ, а длина кратчайшего $B$\=/доказательства как экспонента, то $B$~не может полиномиально моделировать~$A$.
Если при этом $A$ полиномиально моделирует~$B$, то разумно считать систему~$A$ мощнее системы~$B$.

В серии работ конца 90\=/х\--начала 00\=/х годов были получены результаты, касающиеся сложности доказательств в системах, связанных с методом резолюций~\cite{robinson1965}.
В~данном контексте важнейший для нас результат содержится в статьях~\cite{beame2003,beame2004}, где было показано, что \enquote{общая резолюция} (\textit{general resolution}) в ее пропозициональном варианте полиномиально эквивалентна алгоритму CDCL с рестартами.
Данный факт, в частности, означает, что CDCL имеет экспоненциальную сложность.
Действительно, в работе~\cite{haken1985} было установлено, что общая резолюция экспоненциальна на семействе логических противоречий, известных как \enquote{формулы Дирихле} (\textit{Pigeon Hole Principle formulas} \--- $\mathrm{PHP}_{n}^{n + 1}$~\cite{ben-sasson2004}).
В~силу сказанного выше, это означает, что и CDCL будет иметь на $\mathrm{PHP}_{n}^{n + 1}$ экспоненциальную сложность.
В~то же время, CDCL является более мощной системой доказательств, чем DPLL~\cite{beame2003,beame2004,ben-sasson2004}.


\subsection{SAT-решатели}
\label{sub:sat-solvers}

На практике для решения задачи SAT используются специализированные программные средства \--- SAT-\textit{решатели}.
Несмотря на то, что задача SAT имеет экспоненциальную оценку сложности (при условии, что $P \neq NP$), современные SAT-решатели способны решать формулы с миллионами переменных за обозримое время.
Для выбора наиболее эффективного SAT-решателя можно руководствоваться результатами соревнования SAT~Comptetition~\cite{sat-competition}: среди текущих лидеров можно выделить MapleCOMSPS~\cite{liang-2016}, CaDiCaL~\cite{cadical}, CryptoMiniSat~\cite{cryptominisat}, Glucose~\cite{glucose} и Plingeling~\cite{lingeling-and-friends}, хотя на практике эффективность решателей может значительно отличаться, в зависимости от класса рассматриваемых задач.
В некоторых случаях хорошие результаты также показывает MiniSat~\cite{minisat}, являющийся минимальной реализацией CDCL-решателя (\textit{Conflict-Driven Clause Learning}~\cite{grasp}) и служащий основой для многих других решателей (например, CryptoMiniSat и Glucose).

% \todo{Инкрементальность}


% \section{Методы сведения задач к SAT}
% \label{sub:sat-encodings}

% \todo{Пример сведения задачи раскраски графа к SAT -- описание переменных и ограничений. Дополнительно -- оптимизационная постановка задачи.}

% \todo{LEC as SAT}

% \todo{ATPG as SAT}


\section{Ограничения на кардинальность}%
\label{sec:cardinality}

\emph{Ограничение на кардинальность} некоторого заданного множества булевых переменных~$E$ обозначается~$\Psi(E, \mu, \rho)$ и заключается в следующем: минимум~$\mu$, но максимум~$\rho$ переменных из~$E$ должны иметь истинное значение.
Такое ограничение может быть выражено в виде псевдо-булевого ограничения $\mu \leq \bigsumnolim_{e \in E} \fun{bool2int}(e) \leq \rho$, где $\mu$ и~$\rho$ \--- нижняя и верхняя границы, а $\fun{bool2int}(e)$ обозначает функцию, которая переводит булевы значения \texttt{true} и \texttt{false} переменнойЁ$e$ в целые числа $1$~и~$0$, соответственно.

Для того, чтобы закодировать такое ограничение в SAT \--- сформировать формулу в конъюктивной нормальной форме (КНФ), которая выполнима тогда и только тогда, когда выполняется ограничение на кардинальность $\Psi(E, \mu, \rho)$.
В~данной работе используется техника \textit{totalizer}~\cite{sat-cardinality}, которая заключается в формировании унарной записи числа, представляющего сумму элементов заданного множества булевых переменных.

В оригинальной статье~\cite{sat-cardinality} используется следующая нотация: множество \textit{входных переменных} обозначается $E = \Set{e_1, \dotsc, e_n}$, где $n$ \--- их число; множество \textit{выходных переменных}, соответствующих \textit{битам} в унарной записи суммы, обозначается $S = \Set{s_1, \dotsc, s_n}$ и состоит из~$n$ новых переменных; множество \textit{связующих переменных} обозначается~$L$.
Взаимосвязь между этими множествами может быть представлена с помощью бинарного дерева следующим образом.
Начнем построение бинарного дерева с корня: отметим эту вершину числом~$n$, а затем будем итеративно добавлять к каждому листу дерева, отмеченному число~$m > 1$, двух потомков, отмеченных числами $\floor{m/2}$ и $(m - \floor{m/2})$.
В результате будет получено бинарное дерево с $n$~листьями, отмеченных единицами.
С~каждым листом дерева~$i$ ассоциируется множество~$\Set{e_i}$, содержащее одну соответствующую входную переменную $e_i \in E$.
С~корнем ассоциируется все множество выходных переменных~$S$.
С~каждой внутренней вершиной дерева (кроме корня), отмеченной числом~$m$, ассоциируется множество из $m$ новых переменных, которые впоследствии будут иметь смысл \emph{битов} в унарной записи суммы числа, представляемого соответствующим поддеревом.
Совокупность всех новых переменных во внутренних вершинах дерева образует множество связующих переменных~$L$.
На рисунке~\ref{fig:example-cardinality-tree} изображен пример бинарного дерева, построенного для $n = 5$.

%% Picture: Cardinality
\begin{figure}[htb]
    \centering
    \begin{adjustbox}{max width=\textwidth}
        \subfile{tex/tikz-totalizer}
    \end{adjustbox}
    \caption{Бинарное дерево для $n = 5$ со связующими переменными $L =\nobreak \Set{\nu_{1}^{1},\nobreak \nu_{2}^{1}, \nu_{1}^{2}, \nu_{2}^{2}, \nu_{3}^{2}, \nu_{1}^{3},\nobreak \nu_{2}^{3}}$}%
    \label{fig:example-cardinality-tree}
\end{figure}

Рассмотрим внутреннюю вершину~$r$, имеющую двух потомков $a$ и~$b$.
Эта вершина~$r$ представляет собой унарную сумму $\alpha + \beta$, где $\alpha$ и~$\beta$ \--- числа, которыми отмечены вершины $a$ и~$b$, соответственно.
Пусть ${R = \Set{r_1, \dotsc, r_m}}$ \--- множество переменных, ассоциированных с вершиной~$r$.
Аналогично, ${A = \Set{a_1, \dotsc, a_{m_1}}}$ и ${B = \Set{b_1, \dotsc, b_{m_2}}}$ \--- множества, ассоциированные с потомками $a$~и~$b$.
Для вершины~$r$ объявляется следующий набор конъюнктов:
\[
    \biglandclap{\substack{
        0 \leq \alpha \leq m_1 \\
        0 \leq \beta \leq m_2 \\
        0 \leq \sigma \leq m \\
        \alpha + \beta = \sigma
    }}
    \bigl[
        C_1(\alpha, \beta, \sigma)
        \land
        C_2(\alpha, \beta, \sigma)
    \bigr] ,
\]
где используется следующая нотация:
\begin{align*}
    a_0 &= b_0 = r_0 = 1 \\
    a_{m_1 + 1} &= b_{m_2 + 1} = r_{m + 1} = 0 \\
    C_1(\alpha, \beta, \sigma) &= \overline{a_{\alpha}} \lor \overline{b_{\beta}} \lor r_{\sigma} \\
    C_2(\alpha, \beta, \sigma) &= a_{\alpha + 1} \lor b_{\beta + 1} \lor \overline{r_{\sigma + 1}}
\end{align*}
Отметим, что $C_1(\alpha, \beta, \sigma)$ отвечает за соотношение $\sigma \geq \alpha + \beta$, в то время как $C_2(\alpha, \beta, \sigma)$ отвечает за соотношение $\sigma \leq \alpha + \beta$.

Множество всех $C_1$ и~$C_2$ для всех троек вершин $r$,~$a$,~$b$ в бинарном дереве образует так называемый \textit{сумматор} (\textit{totalizer})~$\Phi(E)$.
Для~того, чтобы полученный сумматор представлял желаемый интервал, соответствующий ограничению на кардинальность $\mu \leq \card[big]{\Set{e_i \in E \given e_i = \top}} \leq \rho$, необходимо объявить так называемый \textit{компаратор}~$\Omega(S, \mu, \rho)$, где $S = \Set{s_1, \dotsc, s_n}$ \--- множество из $n = \card{E}$ \textit{вспомогательных} переменных, создаваемых дополнительно во время кодирования:
\[
    \Omega(S, \mu, \rho) =
    \biglandclap{1 \leq i \leq \mu} s_i
    \biglandsmashr{\rho+1 \leq j \leq n} \overline{s_j}
\]
В результате, ограничение на кардинальность $\Psi(E, \mu, \rho)$ представляется в виде конъюнкции сумматора~$\Phi(E)$ и компаратора~$\Omega(S, \mu, \rho)$.


\section{Разбиение задачи SAT}
\label{sec:sat-partitioning}

Для эффективного решения сложных задач SAT часто целесообразно использовать некоторые методы разделения исходной задачи на более простые.
Естественный способ декомпозиции задачи SAT на подзадачи называется \enquote{разбиением} (\textit{partitioning approach})~\cite{hyvarinen2011}.

Рассмотрим произвольную КНФ-формулу~$C$ над множеством булевых переменных~$X$ и множество $\Pi = \{G_1, \dots, G_s\}$, где $G_i$, $i \in \{1, \dots, s\}$, являются некоторыми булевыми формулами над~$X$.
Будем говорить, что $\Pi$~является \textit{разбиением (partitioning) задачи SAT для~$C$}, если выполняются следующие условия:
\begin{enumerate}
    \item Формулы $C$ и $C \land (G_1 \lor \dots \lor G_s)$ эквивалентны по выполнимости.
    \item Для любых $i \neq j \in \{1, \dots, s\}$, формула $C \land G_i \land G_j$ невыполнима.
\end{enumerate}


\subsection{Необходимые основы теории вероятности}

Ниже будут использоваться некоторые вероятностные рассуждения для оценки сложности кодировок SAT для LEC задач с использованием разбиения SAT.
Ниже описаны некоторые основные факты из теории вероятностей.

Пусть $\xi$ \--- некоторая случайная величина с конечным набором числовых значений (спектром) $\Spec(\xi) = \{\xi_1, \dots, \xi_M\}$ и вероятностным распределением $P_\xi = \{p_1, \dots, p_M\}$.
В~дальнейшем будем считать, что $0 < \xi_i < \infty$ для каждого $i \in \{1, \dots, M\}$.
Тогда, математическое ожидание (среднее значение) величины~$\xi$ определяется как $\E[\xi] = \sum_{i = 1}^{M} \xi_i p_i$.
Во многих практических приложениях знание~$\E[\xi]$ оказывается весьма важным.
Однако, часто невозможно точно вычислить значение~$\E[\xi]$ за разумное время.
В~таких случаях можно вместо этого оценить~$\E[\xi]$ с некоторой заранее заданной точностью~$\varepsilon$.
Соответствующие алгоритмы используют случайные выборки и традиционно относятся к методу Монте-Карло~\cite{metropolis1949}.

Более точно, пусть $\xi^1, \dots, \xi^N$ \--- независимые наблюдения случайной величины~$\xi$.
Тогда неравенство Чебышёва~\cite{feller1971} утверждает (см., например,~\cite{semenov2021}):
\begin{equation}\label{eq:cheb}
    \Pr\left\{
        \Bigl|
            \E[\xi] - \frac{1}{N} \sum_{j = 1}^N \xi^j
        \Bigr|
        \leq
        \varepsilon \cdot \E[\xi]
    \right\} \geq
    1 - \delta,
\end{equation}
где $\delta = \frac{\Var[\xi]}{\varepsilon^2 \cdot N \cdot \E^2[\xi]}$, а~$\Var[\xi]$~обозначает дисперсию случайной величины~$\xi$.
Из~\eqref{eq:cheb} следует, что для конечных $\E[\xi]$ и~$\Var[\xi]$ математическое ожидание~$\E[\xi]$ может быть аппроксимировано (в смысле~\eqref{eq:cheb}) значением $\frac{1}{N} \sum_{j = 1}^N \xi^j$ с любыми заранее заданными $\varepsilon, \delta \in (0,1)$ путём увеличения числа наблюдений~$N$.


\subsection{Декомпозиционная трудность}
\label{sub:dhardness}

Концепция \textit{лазеек} (\textit{backdoors}) была введена в классической работе~\cite{williams2003}.
В частности, множество переменных~$B$ в произвольной КНФ-формуле~$C$ является \textit{сильной лазейкой} (Strong Backdoor Set \--- SBS) для~$C$ относительно некоторого полиномиального алгоритма~$P$ (называемого вспомогательным решателем (\textit{subsolver})), если формула~$C[\beta/B]$ решается с помощью~$P$ (получается ответ SAT/UNSAT за полиномиальное время) для любого~$\beta \in \{0,1\}^{|B|}$.
Здесь через~$C[\beta/B]$ обозначается формула, полученная подстановкой значений~$\beta$ в переменные из~$B$ в~$C$.
Можно заметить~\cite{ansotegui2008}, что если $B$ \--- некоторый SBS, то сложность~$C$ ограничена сверху значением $\fun{poly}(|C|) \cdot 2^{|B|}$, где $\fun{poly}({\cdot})$ \--- некоторый полином.

В статье~\cite{semenov2021} было предложено использовать полный детерминированный SAT-решатель~$A$ в качестве вспомогательного решателя, вместо традиционного полиномиального алгоритма~$P$.
Для оценки производительности решателя, введём следующие обозначения.
Пусть~$t_A(C)$ обозначает время работы~$A$ на КНФ-формуле~$C$.
Сложность формулы~$C$ относительно множества~$B$ и солвера~$A$ может быть определена следующим образом:
\begin{equation}\label{eq:dhardness}
    \mu_{A,B}(C) = \bigsumclap{\beta \in \{0,1\}^{|B|}} t_A(C[\beta/B])
\end{equation}
Минимальное значение~\eqref{eq:dhardness} по всем возможным множествам~$B \in 2^X$ называется \textit{декомпозиционной трудностью} (\textit{decomposition hardness}) формулы~$C$ относительно алгоритма~$A$.

Как показано в~\cite{semenov2021}, значение~\eqref{eq:dhardness} можно выразить с использованием математического ожидания случайной величины~$\xi_B$, связанной с множеством~$B$, которая задается следующим соотношением:
\begin{equation}\label{eq:dh_mc}
    \mu_{A,B}(C) = 2^{|B|} \cdot \E[\xi_B]
\end{equation}
Для оценки значения~\eqref{eq:dhardness} можно использовать метод Монте-Карло и формулу~\eqref{eq:cheb}.
Это сводит задачу оценки сложности декомпозиции к задаче псевдо-булевой \textit{black-box} оптимизации, которая включает перебор различных множеств~$B$ и оценку сложности~$C$ относительно каждого~$B$ в попытке минимизировать это значение в пространстве~$2^X$.
В~\cite{semenov2021} для этой цели использовались метаэвристические алгоритмы.


\subsection{Вероятностный подход к оцениванию трудности булевых формул}

Предлагаемые в главе~\ref{ch:partitionings} конструкции для декомпозиции формул, кодирующих трудные примеры LEC, основаны на концепции \textit{декомпозиционной трудности}, предложенной в~\cite{semenov2021}. Данная концепция в свою очередь базируется на понятии \textit{лазейки}, введенном в~\cite{williams2003}.

Пусть рассматривается произвольная формула~$C$ в КНФ над множеством переменных~$X$.
Для произвольного $B \subseteq X$ через $\{ 0,1 \}^{|B|}$ обозначается множество всех возможных наборов значений переменных из~$B$.
Пусть $P$ \--- некоторый полиномиальный алгоритм, который получает на вход произвольную КНФ, а на выход выдает ответ из следующего множества $\{ \textrm{SAT}, \textrm{UNSAT}, \textrm{INDET} \}$, ответ INDET соответствует ситуации, когда $P$ не может за отведенное время решить, выполнима ли рассматриваемая КНФ.
Если применение алгоритма~$P$ к формуле~$C$ выдает ответ из множества $\{ \textrm{SAT}, \textrm{UNSAT} \}$, то будем обозначать данный факт через $C \in \Sigma(P)$.
Если же результатом применения~$P$ к~$C$ является ответ INDET, то обозначим данную ситуацию через $C \notin \Sigma(P)$.

Через $C[\beta/B]$ обозначим формулу, полученную из~$C$ в результате подстановки набора $\beta$ значений переменных из~$B$.
Тогда множество~$B$ называется сильной лазейкой (Strong Backdoor Set, SBS), если для любого $\beta \in \{ 0,1 \}^{|B|}$ имеет место $C[\beta/B] \in \Sigma(P)$.

В статье~\cite{ansotegui2008} было отмечено, что любое SBS~$B$, существенно меньшее~$X$, дает нетривиальную верхнюю оценку трудности формулы~$C$, поскольку существует алгоритм со сложностью $\fun{poly}(|C|) \cdot 2^{|B|}$, определяющий выполнимость~$C$, для некоторого полинома $\fun{poly}(\cdot)$.

На основании этой идеи в статье~\cite{semenov2021} был предложен подход к оцениванию трудности произвольных булевых формул, используя их декомпозиционные представления, называемые также разбиениями (\textit{partitioning}~\cite{hyvarinen2011}).
Более того, оказалось, что оценивать декомпозиционную трудность можно при помощи вероятностных алгоритмов, традиционно относимых к методу Монте-Карло~\cite{metropolis1949}. Кратко изложим здесь основную суть данного подхода.

Прежде всего напомним понятие разбиения.
Согласно~\cite{hyvarinen2011} разбиение (\textit{partitioning}) произвольной КНФ над множеством переменных $X$ \--- это множество формул $\Pi = \{ G_{1},\dots,G_{s} \}$ над~$X$, такое что выполнены два следующих требования:
\begin{enumerate}
    \item Для любых $i \neq j \in \{ 1,\dots,s \}$, формула $C \land G_{i} \land G_{j}$ невыполнима.
    \item Формула $C$ выполнима тогда и только тогда, когда выполнима формула $C \land (G_{1} \lor \ldots \lor G_{s})$.
\end{enumerate}

Если $\Pi$ \--- некоторое разбиение, то на задачу о выполнимости~$C$, можно смотреть как на семейство аналогичных задач для формул вида $C \land G_{i}$, где $i \in \{ 1,\dots,s \}$.
Для решения последних можно использовать параллельные вычисления.
SAT задачи для формул вида $C \land G_{i}$ могут быть существенно проще SAT для~$C$: если, например, множество~$\Pi$ \--- это множество из $2^{k}$ различных кубов над переменными $B =\nobreak \{ {\widetilde{x}}_{1},\dots,{\widetilde{x}}_{k} \}$.

Если $\Pi$ \--- некоторое разбиение~$C$, то по аналогии с понятием лазейки можно определить трудность формулы~$C$ относительно~$\Pi$ и некоторого алгоритма~$A$ решения SAT.
Если рассмотреть в роли~$A$ некоторый полный SAT-решатель, то иногда удается найти относительно небольшие по размеру разбиения, которые вполне можно использовать для решения трудных индустриальных примеров SAT (в том числе верификационной природы).
Таким образом, имеет смысл определить трудность~$C$ относительно разбиения~$\Pi$ как следующую величину:
\[
    \mu_{A,\Pi}(C) = \sum_{G \in \Pi}^{}{t_{A}(G \land C)} ,
\]
где через $t_{A}(C)$ обозначено время работы полного SAT-решателя~$A$ на формуле~$C$.

Возникает вопрос: как для конкретного разбиения~$\Pi$ вычислить, или хотя бы оценить величину $\mu_{A,\Pi}(C)$ в ситуации, когда $s$~велико?
Именно для этой цели может быть использован метод Монте-Карло.

Зададим на~$\Pi$ равномерное распределение, приписав каждому~$G_{i} \in \Pi$ вероятность~$1/s$, где $i \in \{1, \dots, s\}$, и получим таким способом некоторое пространство элементарных исходов.
С~каждым~$G_{i} \in \Pi$ свяжем значение случайной величины $\xi_{\Pi} \colon {\Pi \to R^{+}}$, которое на произвольном $G \in \Pi$ равно~$t_{A}(G \land C)$.
Пусть $\Spec(\xi_{\Pi}) = \{\xi_{1}, \dots, \xi_{r}\}$ \--- спектр величины $\xi_{\Pi}$, а $P(\xi_{\Pi}) = \{p_{1}, \dots, p_{r}\}$ \--- закон распределения данной величины.
Как показано в~\cite{semenov2021}, имеет место следующий факт:
\[
    \mu_{A,\Pi}(C) = s \cdot \E[\xi_{\Pi}] ,
\]
где $\E[\xi_{\Pi}]$ \--- математическое ожидание величины~$\xi_{\Pi}$.
В соответствии с методом Монте-Карло можно оценить величину $\mu_{A,\Pi}(C)$ через значение выборочного среднего $\overline{\xi}_{\Pi} = \frac{1}{N} \cdot \sum_{j = 1}^{N}\xi^{j}$, где $\xi^{j}$ \--- это независимые наблюдения величины~$\xi_{\Pi}$.
Использование неравенства Чебышёва~\cite{feller1971} даёт следующее соотношение:
\begin{equation}\label{eq:eq1}
    \Pr\left\{
        (1 - \varepsilon) \E[\xi_{\Pi}] \leq \overline{\xi}_{\Pi} \leq (1 + \varepsilon) \E[\xi_{\Pi}]
    \right\} \geq 1 - \delta ,
\end{equation}
справедливое для любых фиксированных $\varepsilon,\delta \in (0,1)$ и натурального числа~$N$ (число наблюдений), связанных следующим образом:
\begin{equation}\label{eq:eq2}
    \delta = \frac{\Var[\xi_{\Pi}]}{\varepsilon^{2} N \E^2[\xi_{\Pi}]}
\end{equation}

Таким образом, с увеличением~$N$ точность оценивания~$\mu_{A,\Pi}(C)$ величиной~$\overline{\xi}_{\Pi}$ будет возрастать.
Следует особо отметить, что не существует полных гарантий точности таких оценок: при большой дисперсии и малом~$N$ получаемые оценки могут быть сколь угодно неточны.
Тем не менее, можно использовать стандартные статистические аргументы точности получаемых оценок.
Один метод такого рода описан в~\cite{semenov2021} и основан на периодическом увеличении объема выборки~$N$ до тех пор, пока не выполнится неравенство
% TODO: re-check
\begin{equation}\label{eq:eq3}
    N \geq \frac{s^{2}(\xi_{\Pi})}{\delta\varepsilon^{2}\overline{\xi}^{2}_{\Pi}} ,
\end{equation}
в котором $s^{2}(\xi_{\Pi})$ \--- выборочная дисперсия.
Неравенство~\eqref{eq:eq3} является статистическим аналогом неравенства
% TODO: re-check
\begin{equation}\label{eq:eq4}
    N \geq \frac{\Var[\xi_{\Pi}]}{\varepsilon^{2}\delta \E^2[\xi_{\Pi}]}
\end{equation}
Заметим, что условие~\eqref{eq:eq1} при фиксированных $\varepsilon,\delta \in (0,1)$ имеет место для любого~$N$, для которого выполнено~\eqref{eq:eq4}.

Также возможно построение доверительных интервалов для~$\mu_{A,\Pi}(C)$ с использованием центральной предельной теоремы.
Здесь $\sigma = \sqrt{\Var[\xi]}$ обозначает стандартное отклонение, $\gamma$ \--- уровень доверия, $\gamma = \Phi(\delta_\gamma)$, где $\Phi(\cdot)$ \--- нормальная кумулятивная функция распределения.
Это означает, что при рассмотренных предположениях значение выборочного среднего~$\overline{\xi}_{\Pi} = \frac{1}{N} \cdot \sum_{j = 1}^{N}\xi^{j}$ является хорошим приближением~$\E[\xi]$, когда количество наблюдений~$N$ достаточно велико.
Для любого заданного~$N$ качество этого приближения зависит от значения $\Var[\xi]$.
На практике для оценки $\Var[\xi]$ используется несмещённая выборочная дисперсия~$s^2 = \frac{1}{N-1} \sum_{j = 1}^{N} (\xi^{j} - \overline{\xi})^2$.
В~этом случае вместо~\eqref{eq:eq2} применяется следующая формула~\cite{wilks1962}:
\[
    \Pr\left\{
        \left|
            \E[\xi] - \overline{\xi}_{\Pi}
        \right| < \frac{s \cdot t_{\gamma, N-1}}{\sqrt{N}}
    \right\} \geq \gamma ,
\]
где $t_{\gamma, N-1}$ \--- это квантиль распределения Стьюдента с $N-1$ степенями свободы, соответствующий уровню доверия~$\gamma$.
Если, например, $\gamma = 0.999$ и $N \geq\nobreak 10000$, то~$t_{\gamma, N-1} \approx\nobreak 3.29$.

% В нашем случае важно отметить, что $N$~может быть значительно меньше, чем~$2^d$.
% Это означает, что этап предобработки может быть использован для оценки общего времени, необходимого для обработки всей семейства декомпозиций~$\Delta(C, \tilde{X})$.


% \chapterconclusion

% \todo{Завершение обзора}
