\usepackage{physics}
\usepackage[Export]{adjustbox}
\usepackage{bm}
\usepackage{suffix}
\usepackage[shortcuts,shortemdash,wordspacearound]{extdash}
\usepackage{subfiles}
\usepackage[np]{numprint}
\usepackage[normalem]{ulem}

\usepackage{tikz}
\usetikzlibrary{
    arrows.meta,
    automata,
    calc,
    decorations.pathreplacing,
    fit,
    positioning,
}
\usepackage{forest}

\usepackage[ruled,vlined,linesnumbered]{algorithm2e}
\SetAlgoInsideSkip{smallskip}
\SetAlgorithmName{Алгоритм}{algorithm}{Список алгоритмов}
\SetKw{Break}{break}
\SetKw{Let}{let}
\SetKw{Label}{label}
\SetKw{Goto}{goto}
\SetKwBlock{Loop}{loop}{end loop}

\makeatletter
\newcommand{\onlyinpreamble}[1]{% {<text>}
    \ifx\@onlypreamble\@notprerr% after \begin{document}
    \else% before \begin{document}
        #1
    \fi}
\def\input@path{{../}}
\makeatother

%% Usage:
%%   place "\enabletikzpreview%"
%%   after "\documentclass[../main.tex]{subfiles}%"
%%   in your "tikz-something.tex" subfile.
\newcommand{\enabletikzpreview}{%
    \onlyinpreamble{%
        \usepackage[active,tightpage]{preview}
        \PreviewBorder=2pt
        \PreviewEnvironment{tikzpicture}
        \PreviewEnvironment{forest}
    }}

% Выводы по главе
\newcommand{\chapterconclusion}{%
    \section*{Выводы по главе~\thechapter}%
    \addcontentsline{toc}{section}{Выводы по главе~\thechapter}%
}

%% Hyphenation setup
\hyphenpenalty=500 % Default is 50
\tolerance=500 % Default is 200
\lefthyphenmin=3
\righthyphenmin=3

%% makecell setup
\renewcommand\theadfont{\bfseries\normalsize}
\renewcommand\theadgape{}

%% Column types
\newcolumntype{R}{>{$}r<{$}}

%% Big operators factory
\newcommand\newbigop[2]{% {<name>}{<symbol>}
    \expandafter\newcommand\csname big#1\endcsname{%
        #2\limits}
    \expandafter\newcommand\csname big#1nolim\endcsname{%
        #2\nolimits}
    \expandafter\newcommand\csname big#1clap\endcsname[1]{% {<sub>}
        #2\limits_{\mathclap{##1}}}
    \expandafter\newcommand\csname big#1smashl\endcsname[1]{% {<sub>}
        \smashoperator[l]{#2_{##1}}}
    \expandafter\newcommand\csname big#1smashr\endcsname[1]{% {<sub>}
        \smashoperator[r]{#2_{##1}}}
}
%% Big operators
\newbigop{land}{\bigwedge}
\newbigop{lor}{\bigvee}
\newbigop{union}{\bigcup}
\newbigop{intersection}{\bigcap}
\newbigop{sum}{\sum}
\newbigop{prod}{\prod}
\newbigop{xor}{\bigoplus}

%% Some sets
\newcommand\Bool{\mathbb{B}}
\newcommand\Natural{\mathbb{N}}
\newcommand\NaturalPlus{\Natural^{+}}
\newcommand\NaturalZero{\Natural_{0}}
\newcommand\Integer{\mathbb{Z}}
\newcommand\PositiveInteger{\Integer^{+}}
\newcommand\Rational{\mathbb{Q}}
\newcommand\Real{\mathbb{R}}

%% Set declaration syntax (from mathtools manual)
\providecommand\given{}
\newcommand\GivenSymbol[1][]{%
    \nonscript\:
    #1\vert\allowbreak
    \nonscript\:
    \mathopen{}
}
\DeclarePairedDelimiterX\Set[1]\{\}{%
    \renewcommand\given{\GivenSymbol[\delimsize]}
    #1
}

%% Pair and tuple declaration syntax
\DeclarePairedDelimiter{\Pair}\langle\rangle
\DeclarePairedDelimiter{\Tuple}()
\DeclarePairedDelimiter{\Vector}\langle\rangle

%% Floor and ceil operations
\DeclarePairedDelimiter{\floor}\lfloor\rfloor
\DeclarePairedDelimiter{\ceil}\lceil\rceil

%% Simple implication and iff
\let\implies\rightarrow
\let\iff\leftrightarrow

%% Union and intersection operators
\newcommand\union{\cup}
\newcommand\intersection{\cap}

%% Xor operator
\newcommand\xor{\oplus}
\newcommand\lxor{\mathbin{\veebar}}

%% (mod n) operator
\newcommand\Mod[1]{(\mathrm{mod}\ #1)}

%% Entailment operator
\newcommand\entails{\vDash}
\newcommand\nentails{\nvDash}

%% Empty and universal sets
\renewcommand\emptyset{\varnothing}
\newcommand\universalset{\mathfrak{U}}

%% Powerset operation (via "Weierstrass p")
\newcommand{\powerset}[1]{% {<arg>}
    \mathcal{P}({#1})}

%% Set cardinality operation
\DeclarePairedDelimiter{\card}\lvert\rvert

%% Absolute value and norm operations
% Note: use "physics" package instead
% \DeclarePairedDelimiter{\abs}\lvert\rvert
% \DeclarePairedDelimiter{\norm}\lVert\rVert

%% Two dots for integer intervals: [1..C] = [1 \dd C]
% \newcommand{\twodots}{\mathinner{\ldotp\ldotp}}
\newcommand{\twodots}{\mathinner{..}}
\let\dd\twodots

%% Range operation [low .. high]
% Note: outer braces ensure non-breakable range
\newcommand{\Range}[2]{% {<low>}{<high>}
    {[#1 \dd #2]}}

%% Boolean vector
\newcommand{\boolvec}[1]{\Bool^{\card{#1}}}
\let\BoolVec\boolvec

%% Struts
\newcommand\TopStrut{\rule{0pt}{2.6ex}}
\newcommand\BotStrut{\rule[-1.2ex]{0pt}{0pt}}

%% Single-line arrows for implication and iff
% \def\implies{\mathrel{\rightarrow}}
% \def\impliedby{\mathrel{\leftarrow}}
% \def\iff{\mathrel{\leftrightarrow}}

%% Fix small caps when babel is loaded with russian language
%% Note: font must support english small caps!
\newcommand{\smallcaps}[1]{% {<text>}
    \texorpdfstring{%
        \begingroup%
            \selectlanguage{english}%
            \textsc{#1}%
        \endgroup%
    }{#1}}

% %% Maximum height of float (on 'p' page)
\newcommand{\maxheight}[1]{% {<lines>}
    \textheight% Total text area height
    -#1\baselineskip% Line height
    -\abovecaptionskip% Space above caption
    -\belowcaptionskip% Space below caption
    -0.1pt% Just because something is a little bit off after all
}

%% Input/Output actions
\newcommand{\ActionMM}[2]{% {<event>}{<values>}
    \ensuremath{#1[#2]}%
}
\newcommand{\ActionTT}[2]{% {<event>}{<values>}
    \texttt{#1[#2]}%
}
\newcommand{\ActionMT}[2]{% {<event>}{<values>}
    \ensuremath{#1}\texttt{[#2]}%
}
\newcommand{\ActionTM}[2]{% {<event>}{<values>}
    \texttt{#1}\ensuremath{[#2]}%
}
\let\Action\ActionMM
\newcommand{\EmptyActionM}[1]{% {<event>}}
    \ensuremath{#1}%
}
\newcommand{\EmptyActionT}[1]{% {<event>}}
    \texttt{#1}%
}
\newcommand{\InputEvent}{e^{I}}
\newcommand{\OutputEvent}{e^{O}}
\newcommand{\InputValues}{\bar{x}}
\newcommand{\OutputValues}{\bar{z}}
\newcommand{\InputAction}[1][]{% [<index>]
    \Action{\InputEvent_{#1}}{\InputValues_{#1}}}
\newcommand{\OutputAction}[1][]{% [<index>]
    \Action{\OutputEvent_{#1}}{\OutputValues_{#1}}}
% \newcommand{\InputActionEU}[1][]{% [<index>]
%     \Action{e_{#1}}{u_{#1}}}
% \newcommand{\OutputActionOY}[1][]{% [<index>]
%     \Action{o_{#1}}{y_{#1}}}

%% NuSMV execution state
\newcommand{\ExecutionState}[2]{% {<state>}{<output-action>}
    [#1 / #2]}

%% Over/under brace/braket with equality
\newcommand{\overbraceeq}[4][\;=\;]{% [<mid>]{<lhs>}{<rhs>}{<body>}
    \overbrace{#4}^{\smash[b]{\mathllap{#2} #1 \mathrlap{#3}}}}
\newcommand{\underbraceeq}[4][\;=\;]{% [<mid>]{<lhs>}{<rhs>}{<body>}
    \underbrace{#4}_{\mathllap{#2} #1 \mathrlap{#3}}}
\newcommand{\overbracketeq}[4][\;=\;]{% [<mid>]{<lhs>}{<rhs>}{<body>}
    \overbracket{#4}^{\mathllap{#2} #1 \mathrlap{#3}}}
\newcommand{\underbracketeq}[4][\;=\;]{% [<mid>]{<lhs>}{<rhs>}{<body>}
    \underbracket{#4}^{\mathllap{#2} #1 \mathrlap{#3}}}

%% AtLeastOne / AtMostOne / ExactlyOne operations
\newcommand{\AtLeastOne}[2][]{\mathrm{ALO}_{#1}(#2)}
\newcommand{\AtMostOne}[2][]{\mathrm{AMO}_{#1}(#2)}
\newcommand{\ExactlyOne}[2][]{\mathrm{EO}_{#1}(#2)}

%% AtLeastOne / AtMostOne / ExactlyOne operations
% \newcommand{\ALO}[2][]{\mathrm{ALO}_{#1}(#2)}
% \newcommand{\AMO}[2][]{\mathrm{AMO}_{#1}(#2)}
% \newcommand{\EO}[2][]{\mathrm{EO}_{#1}(#2)}

%% Temporal operators
\newcommand{\Temp}[1]{\textbf{#1}}

%% Some mysterious stuff
\newcommand{\com}[1]{\ensuremath{\mathtt{#1}}}

%% Null
\newcommand{\Null}{\com{null}}

%% Add hat for NegativeScenarioTree-related stuff
\newcommand{\Negative}[1]{\vphantom{#1}\smash{\widehat{#1}}}

%% Function name
\newcommand{\funcname}[1]{% {<name>}
    \operatorname{\mathit{#1}}}

%% Domain function
\newcommand{\Dom}{\operatorname{Dom}}

%% Function with parens
\newcommand{\NewFunction}[2]{% {<command>}{<name>}
    \expandafter\newcommand\csname #1\endcsname[1]{% {<arg>}
        \funcname{#2}(##1)%
    }%
}
\newcommand{\NewNegativeFunction}[2]{% {<command>}{<name>}
    \expandafter\newcommand\csname #1\endcsname[1]{% {<arg>}
        \Negative{\funcname{#2}}(##1)%
    }%
}
\newcommand{\NewFunctionWithNegative}[3][neg]{% [<neg-cmd-prefix>]{<command>}{<name>}
    \NewFunction{#2}{#3}
    \NewNegativeFunction{#1#2}{#3}
}

%% Assignment
\newcommand{\Assign}[2]{% {<lhs>}{<rhs>}
    #1 \gets #2%
}
\newcommand{\AssignT}[1]{% {<lhs>}
    \Assign{#1}{1}%
}
\newcommand{\AssignF}[1]{% {<lhs>}
    \Assign{#1}{0}%
}

%% Const Factory
\newcommand{\NewConst}[2]{% {<name>}{<symbol>}
    \expandafter\newcommand\csname #1\endcsname{#2}%
}
\newcommand{\NewModularConst}[1]{% <name>}
    \expandafter\newcommand\csname Modular#1\endcsname[1][m]{% [<index>]
        \csname #1\endcsname^{(##1)}%
    }
}

%% Variables factory
% \newcommand{\NewVariableEx}[4]{% {<name>}{<symbol>}{<pre-super>}{<pre-sub>}
%     \expandafter\newcommand\csname #1\endcsname[2][]{% [<super>]{<sub>}
%         #2^{#3 ##1}_{#4 ##2}}
%     \WithSuffix\expandafter\newcommand\csname #1\endcsname*{#2}
% }
% \newcommand{\NewVariable}[2]{% {<name>}{<symbol>}
%     \NewVariableEx{#1}{#2}{}{}}
% \newcommand{\NewModularVariable}[1]{% {<variable>}
%     \expandafter\newcommand\csname Modular#1\endcsname[2][m]{% [<module>]{<sub>}
%         \csname #1\endcsname[(##1)]{##2}}}
% \newcommand{\NewVariable}[2]{% {<name>}{<symbol>}
%     \expandafter\newcommand\csname #1\endcsname{#2}%
% }
% \newcommand{\NewModularVariable}[1]{% {<name>}
%     \expandafter\newcommand\csname Modular#1\endcsname[1][m]{% [<index>]
%         \csname #1\endcsname^{(##1)}}%
% }

%% Common
\NewConst{Automaton}{\mathcal{A}}
\NewConst{FunctionBlock}{FB}
\NewConst{SetStates}{Q}
\NewConst{SetStatesAux}{\SetStates_{0}}
\NewConst{InitialState}{q_{\text{init}}}
\NewConst{SetInputEvents}{E^{I}}
\NewConst{SetOutputEvents}{E^{O}}
\NewConst{SetInputVariables}{\mathcal{X}}
\NewConst{SetOutputVariables}{\mathcal{Z}}
\NewConst{AssociationInput}{W^{I}}
\NewConst{AssociationOutput}{W^{O}}
\NewConst{SetTreeInputs}{\mathcal{U}}
\NewConst{SetTreeOutputs}{\mathcal{Y}}
\NewConst{SetScenarios}{\mathcal{S}}
\NewConst{SetPositiveScenarios}{\SetScenarios^{(+)}}
\NewConst{SetNegativeScenarios}{\SetScenarios^{(-)}}
\NewConst{LTLSpec}{\mathcal{L}}
\NewConst{Tree}{\mathcal{T}}
\NewConst{PositiveTree}{\Tree^{(+)}}
\NewConst{NegativeTree}{\Tree^{(-)}}
\NewConst{TreeRoot}{\rho}
\NewConst{SetTreeNodes}{V}
\NewConst{SetTreeNodesActive}{\SetTreeNodes^{\text{(active)}}}
\NewConst{SetTreeNodesPassive}{\SetTreeNodes^{\text{(passive)}}}
\NewConst{SetAllInputs}{\SetTreeInputs^{*}}
\NewConst{InitialOutputValues}{\OutputValues_{\text{init}}}
\NewConst{PreviousOutputValues}{\OutputValues^{\text{prev}}}
\NewConst{NegativeTreeRoot}{\Negative{\TreeRoot}}
\NewConst{SetNegativeTreeNodes}{\Negative{\SetTreeNodes}}
\newcommand{\negv}{\Negative{v}}
\NewConst{SetNegativeTreeNodesActive}{\SetNegativeTreeNodes^{\text{(active)}}}
\NewConst{SetNegativeTreeNodesPassive}{\SetNegativeTreeNodes^{\text{(passive)}}}
\NewConst{SetNegativeTreeNodesEnds}{\SetNegativeTreeNodes^{\text{(ends)}}}
\NewFunctionWithNegative{tp}{\funcname{tp}}
\NewFunctionWithNegative{tpa}{\funcname{tpa}}
\NewFunctionWithNegative{tie}{\funcname{tie}}
\NewFunctionWithNegative{toe}{\funcname{toe}}
\NewFunctionWithNegative{tin}{\funcname{tin}}
% \NewFunctionWithNegative{ton}{\funcname{ton}}
\NewFunctionWithNegative{tiv}{\funcname{tiv}}
\NewFunctionWithNegative{tov}{\funcname{tov}}
\NewNegativeFunction{negtlb}{\funcname{tlb}}
\NewNegativeFunction{negtbe}{\funcname{t\smash{b}e}}
\newcommand{\toeMulti}[2][r]{% {<arg>}
    \funcname{\funcname{toe}}^{{[#1]}}(#2)%
}
\newcommand{\tovMulti}[2][r]{% {<arg>}
    \funcname{\funcname{tov}}^{{[#1]}}(#2)%
}

%% Modular common
\NewConst{SetPins}{\Pi}
\NewConst{SetInboundVarPins}{\SetPins_{\mathit{in~var}}}
\NewConst{SetOutboundVarPins}{\SetPins_{\mathit{out~var}}}
\NewConst{SetInboundEventPins}{\SetPins_{\mathit{in~ev}}}
\NewConst{SetOutboundEventPins}{\SetPins_{\mathit{out~ev}}}
\NewModularConst{SetInboundVarPins}
\NewModularConst{SetOutboundVarPins}
\NewModularConst{SetInboundEventPins}
\NewModularConst{SetOutboundEventPins}
\NewConst{SetExternalInboundVarPins}{\SetInboundVarPins^{(\mathit{ext})}}
\NewConst{SetExternalOutboundVarPins}{\SetOutboundVarPins^{(\mathit{ext})}}
\NewConst{SetExternalInboundEventPins}{\SetInboundEventPins^{(\mathit{ext})}}
\NewConst{SetExternalOutboundEventPins}{\SetOutboundEventPins^{(\mathit{ext})}}

%% Modular constants
\newcommand{\Modules}{\Range{1}{M}}
\NewModularConst{Automaton}
\NewModularConst{SetStates}
\NewModularConst{SetStatesAux}
\NewModularConst{InitialState}
% \NewModularConst{SetInputEvents}
% \NewModularConst{SetOutputEvents}
\newcommand{\ModularSetInputEvents}[1][m]{% [<module>]
    E^{I (#1)}%
}
\newcommand{\ModularSetOutputEvents}[1][m]{% [<module>]
    E^{O (#1)}%
}
\NewModularConst{SetInputVariables}
\NewModularConst{SetOutputVariables}
\NewModularConst{SetTreeInputs}
\NewModularConst{SetTreeOutputs}
\NewModularConst{SetScenarios}
\newcommand{\ModularSetPositiveScenarios}[1][m]{% [<module>]
    \SetScenarios^{(+)(#1)}%
}
\newcommand{\ModularPositiveTree}[1][m]{% [<module]
    \Tree^{(+)(#1)}%
}
\newcommand{\ModularNegativeTree}[1][m]{% [<module]
    \Tree^{(-)(#1)}%
}
\NewModularConst{SetTreeNodes}

%% Interface variables
% \NewVariable{TransitionFunction}{\lambda}
% \NewVariable{OutputEventFunction}{\psi}
% \NewVariable{OutputFunction}{\omega}
% \NewVariable{AlgorithmFunction}{\OutputFunction*}

%% BFS variables
\NewConst{BfsTransitionSymbol}{\tau}
\NewConst{BfsParentSymbol}{\pi}
\NewConst{BfsTransition}%
    {\BfsTransitionSymbol^\text{BFS}}{}
\NewConst{BfsParent}%
    {\BfsParentSymbol^\text{BFS}}
\NewConst{BfsTransitionAutomaton}{%
    \BfsTransitionSymbol^\text{BFS-$\mathcal{A}$}}
\NewConst{BfsParentAutomaton}{%
    \BfsParentSymbol^\text{BFS-$\mathcal{A}$}}
\NewConst{BfsTransitionGuard}{%
    \BfsTransitionSymbol^\text{BFS-$\mathcal{G}$}}
\NewConst{BfsParentGuard}{%
    \BfsParentSymbol^\text{BFS-$\mathcal{G}$}}
\NewConst{BfsTransitionAutomatonOrder}%
    {\BfsTransitionSymbol^\text{BFS-$\mathcal{A}$ (order)}}
\NewConst{BfsParentAutomatonOrder}%
    {\BfsParentSymbol^\text{BFS-$\mathcal{A}$ (order)}}
\NewConst{BfsTransitionModified}%
    {\BfsTransitionSymbol^\text{BFS-mod}}
\NewConst{BfsParentModified}%
    {\BfsParentSymbol^\text{BFS-mod}}

%% Core variables
\NewConst{StateOutputEvent}{\phi}
\NewConst{StateAlgorithm}{\gamma}
\NewConst{StateUsed}{\alpha}
\NewConst{TransitionDestination}{\tau}
\NewConst{TransitionInputEvent}{\xi}
\NewConst{TransitionTruthTable}{\theta}
\NewConst{TransitionFiring}{\delta}
\NewConst{TransitionFirstFired}{\mathit{ff}}
\NewConst{TransitionNotFired}{\mathit{nf}}
\NewConst{ActualTransitionFunction}{\lambda}

%% Mapping variables
\NewConst{Mapping}{\mu}
\NewConst{MemoryOutputValue}{\nabla}
\NewConst{ComputedValue}{\nabla}

%% Guards variables
\NewConst{NodeType}{\eta}
\newcommand{\NodeTypeTerminal}{\square}
\newcommand{\NodeTypeAnd}{\mathord{\land}}
\newcommand{\NodeTypeOr}{\mathord{\lor}}
\newcommand{\NodeTypeNot}{\mathord{\neg}}
\newcommand{\NodeTypeNone}{\mathord{\bullet}}
\NewConst{NodeInputVariable}{\chi}
\NewConst{NodeChild}{\sigma}
\NewConst{NodeParent}{\pi}
\NewConst{NodeValue}{\vartheta}

%% Negative variables
% \NewConst{NegativeTransitionFunction}{\Negative{\TransitionFunction*}}
\NewConst{NegativeActualTransitionFunction}{\Negative{\ActualTransitionFunction}}
% \NewConst{NegativeOutputEventFunction}{\Negative{\OutputEventFunction*}}
% \NewConst{NegativeOutputFunction}{\Negative{\OutputFunction*}}
% \NewConst{NegativeAlgorithmFunction}{\Negative{\AlgorithmFunction*}}
\NewConst{NegativeMapping}{\Negative{\Mapping}}

%% Modular variables
\NewConst{ModuleControllingOutputVariable}{\zeta}
\NewModularConst{StateOutputEvent}
\NewModularConst{StateAlgorithm}
\NewModularConst{StateUsed}
\NewModularConst{TransitionDestination}
\NewModularConst{TransitionInputEvent}
\NewModularConst{TransitionTruthTable}
\NewModularConst{TransitionFiring}
\NewModularConst{TransitionFirstFired}
\NewModularConst{TransitionNotFired}
\NewModularConst{TransitionFunction}
\NewModularConst{ActualTransitionFunction}
\NewModularConst{OutputEventFunction}
\NewModularConst{OutputFunction}
\NewModularConst{AlgorithmFunction}
\NewModularConst{Mapping}
\NewModularConst{ComputedValue}
\NewConst{VarPinParent}{\pi^{\mathit{var}}}
\NewConst{EventPinParent}{\pi^{\mathit{ev}}}
\NewConst{InboundVarPinComputedValue}{\ComputedValue^{\mathit{in}}}
\NewConst{OutboundVarPinComputedValue}{\ComputedValue^{\mathit{out}}}
\NewConst{InputIndex}{\partial}
\NewModularConst{InputIndex}

%% Negative modular variables
\NewModularConst{NegativeMapping}
% \NewModularConst{NegativeStateOutputEvent}
% \NewModularConst{NegativeStateAlgorithm}
% \NewModularConst{NegativeActualTransitionFunction}


%% Compound constants
\NewConst{SetCompoundScenarios}{\mathcal{CS}}
\NewConst{SetPositiveCompoundScenarios}{\SetCompoundScenarios^{(+)}}
\NewConst{SetNegativeCompoundScenarios}{\SetCompoundScenarios^{(-)}}
\NewConst{CompoundTree}{\mathcal{CT}}
\NewConst{PositiveCompoundTree}{\CompoundTree^{(+)}}
\NewConst{SetCompoundTreeNodes}{\mathcal{CV}}
\NewConst{SetCompoundTreeNodesActive}{\SetCompoundTreeNodes_{\text{(act)}}}
\NewConst{SetCompoundTreeNodesPassive}{\SetCompoundTreeNodes_{\text{(pass)}}}
\NewConst{SetCompoundTreeNodesIdle}{\SetCompoundTreeNodes_{\text{(idle)}}}
\NewConst{SetCompoundTreeNodesEnds}{\SetCompoundTreeNodes_{\text{(ends)}}}

%% Negative compound constants
\NewConst{NegativeCompoundTree}{\CompoundTree^{(-)}}
\NewConst{SetNegativeCompoundTreeNodes}{\Negative{\SetCompoundTreeNodes}}
\NewConst{SetNegativeCompoundTreeNodesActive}{\SetNegativeCompoundTreeNodes_{\text{(act)}}}
\NewConst{SetNegativeCompoundTreeNodesPassive}{\SetNegativeCompoundTreeNodes_{\text{(pass)}}}
\NewConst{SetNegativeCompoundTreeNodesIdle}{\SetNegativeCompoundTreeNodes_{\text{(idle)}}}
\NewConst{SetNegativeCompoundTreeNodesEnds}{\SetNegativeCompoundTreeNodes_{\text{(ends)}}}


%% Algorithms
\newcommand{\NewAlgorithm}[2]{% {<command>}{<name>}
    \expandafter\newcommand\csname Algo#1\endcsname{%
        \smallcaps{#2}}
    \expandafter\newcommand\csname Algo#1Full\endcsname{%
        \texorpdfstring{\ensuremath{\smallcaps{#2}^{*}}}{#2*}}
}
\NewAlgorithm{Basic}{Basic}
\NewAlgorithm{Extended}{Extended}
\NewAlgorithm{Complete}{Complete}
\NewAlgorithm{BasicMin}{Basic-min}
\NewAlgorithm{BasicMinC}{Basic-minC}
\NewAlgorithm{BasicMinT}{Basic-minT}
\NewAlgorithm{ExtendedMin}{Extended-min}
\NewAlgorithm{ExtendedMinN}{Extended-minN}
\NewAlgorithm{ExtendedMinUB}{Extended-min-UB}
\NewAlgorithm{CompleteMin}{Complete-min}
% \NewAlgorithm{CompleteMinN}{Complete-minN}
\NewAlgorithm{CompleteMinUB}{Complete-min-UB}
\NewAlgorithm{Cegis}{CEGIS}
\NewAlgorithm{CegisMin}{CEGIS-min}
\NewAlgorithm{CegisUB}{CEGIS-UB}
% Modular Parallel
\NewAlgorithm{ModularParallelBasic}{Parallel-Basic}
\NewAlgorithm{ModularParallelExtended}{Parallel-Extended}
\NewAlgorithm{ModularParallelComplete}{Parallel-Complete}
\NewAlgorithm{ModularParallelBasicMin}{Parallel-Basic-min}
\NewAlgorithm{ModularParallelExtendedMin}{Parallel-Extended-min}
% Modular Consecutive
\NewAlgorithm{ModularConsecutiveBasic}{Consecutive-Basic}
\NewAlgorithm{ModularConsecutiveExtended}{Consecutive-Extended}
\NewAlgorithm{ModularConsecutiveComplete}{Consecutive-Complete}
\NewAlgorithm{ModularConsecutiveBasicMin}{Consecutive-Basic-min}
\NewAlgorithm{ModularConsecutiveExtendedMin}{Consecutive-Extended-min}
% Modular Arbitrary
\NewAlgorithm{ModularArbitraryBasic}{Arbitrary-Basic}
\NewAlgorithm{ModularArbitraryExtended}{Arbitrary-Extended}
\NewAlgorithm{ModularArbitraryComplete}{Arbitrary-Complete}
\NewAlgorithm{ModularArbitraryBasicMin}{Arbitrary-Basic-min}
\NewAlgorithm{ModularArbitraryExtendedMin}{Arbitrary-Extended-min}
% Distributed
\NewAlgorithm{DistributedCegisMin}{Distributed-Cegis-min}


%% Term Factory
\newcommand{\NewTerm}[2]{% {<name>}{<text>}
    \expandafter\newcommand\csname #1\endcsname{%
        \texttt{#2}%
    }
}

%% Terms
\NewTerm{REQ}{REQ}
\NewTerm{CNF}{CNF}
\NewTerm{EventReq}{REQ}
\NewTerm{EventCnf}{CNF}
\NewTerm{EventR}{R}
\NewTerm{EventA}{A}
\NewTerm{EventB}{B}
\NewTerm{EventC}{C}


%% Extra shortcuts
\newcommand{\Cmin}{C_{\text{min}}}
\newcommand{\Tmin}{T_{\text{min}}}
\newcommand{\Nmin}{N_{\text{min}}}
\newcommand{\AutomatonBest}{\Automaton_{\text{best}}}


%% LTL-properties
\newcommand{\Prop}[1]{% {<index>}
    \IfInteger{#1}{%
        \ensuremath{\varphi_{#1}}%
    }{%
        \PackageError{Prop}{Undefined option to Prop: #1}{}%
    }%
}
\newcommand{\LTLProp}[1]{% {<index>}
    \IfEqCase{#1}{%
        {1}{\Temp{G}(\neg(\texttt{c1Extend} \land \texttt{c1Retract}))}%
        {2}{\Temp{G}(\neg(\texttt{c2Extend} \land \texttt{c2Retract}))}%
        {3}{\Temp{G}(\neg(\texttt{vacuum\_on} \land \texttt{vacuum\_off}))}%
        {4}{\Temp{G}(\neg\texttt{vcHome} \land \neg\texttt{vcEnd} \implies \texttt{c1Home} \lor \texttt{c1End})}%
        {5}{\Temp{G}(\neg\texttt{c1Home} \land \neg\texttt{c1End} \implies \texttt{vcHome} \lor \texttt{vcEnd})}%
        {6}{\Temp{G}(\texttt{all\_home} \land \neg\texttt{pp1} \land \neg\texttt{pp2} \land \neg\texttt{pp3} \land \neg\texttt{lifted} \implies \Temp{X}(\neg\texttt{c1Extend} \land \neg\texttt{c2Extend} \land \neg\texttt{vcExtend}))}%
        {7}{\Temp{G}(\texttt{lifted} \implies \Temp{F}(\texttt{dropped}))}%
        {8}{\Temp{G}(\texttt{pp1} \implies \Temp{F}(\texttt{vp1}))}%
        {9}{\Temp{G}(\texttt{pp2} \implies \Temp{F}(\texttt{vp2}))}%
        {10}{\Temp{G}(\texttt{pp3} \implies \Temp{F}(\texttt{vp3}))}%
    }[\PackageError{LTLProp}{Undefined option to LTLProp: #1}{}]
}
\newcommand{\PropWP}[1]{% {<1-3>}
    \IfEqCase{#1}{%
        {1}{\Prop{8}}%
        {2}{\Prop{9}}%
        {3}{\Prop{10}}%
    }[\PackageError{PropWP}{Undefined option to PropWP: #1}{}]
}
\newcommand{\LTLPropWP}[1]{% {<1-3>}
    \IfEqCase{#1}{%
        {1}{\LTLProp{8}}%
        {2}{\LTLProp{9}}%
        {3}{\LTLProp{10}}%
    }[\PackageError{LTLPropWP}{Undefined option to LTLPropWP: #1}{}]
}
\newcommand{\PropertiesConst}{\Prop{1}\text{\==}\Prop{7}}
\newcommand{\PropertiesVar}{\PropWP{1}\text{\==}\PropWP{3}}

\newcommand{\Vin}{V^\text{in}}
\newcommand{\Vout}{V^\text{out}}
\newcommand{\Xin}{X^{\text{in}}}
\newcommand{\Xout}{X^{\text{out}}}
\newcommand{\fun}[1]{\funcname{#1}}
\newcommand{\glue}{\mathbin{\triangle}}
% \newcommand{\E}{\mathrm{E}}
\DeclareMathOperator*{\E}{\mathbb{E}}
\newcommand{\Var}{\mathrm{Var}}
\newcommand{\Cov}{\mathrm{Cov}}
\newcommand{\Spec}{\mathrm{Spec}}

%% Instance
\newcommand{\Instance}[1]{% {<name>}
    \ensuremath{\mathtt{#1}}}
\newcommand{\instance}[2]{% {<name>}{<params>}
    \ensuremath{\Instance{#1}_{#2}}}

%% Auto-switch mathtt to texttt for dashes
\usepackage{xstring}
\newcommand{\autott}[1]{%
    \noexpandarg%
    \ensuremath{\mathtt{\protect\StrSubstitute{#1}{-}{\texttt{-}}}}%
}

%% Partitioning
\newcommand{\Part}[1]{% {<partitioning-type>}
    % \texttt{#1}}
    \autott{#1}}

\colorlet{myr}{red!12}
\colorlet{myg}{green!12}
\colorlet{myb}{blue!10}
\colorlet{myy}{yellow!15}

%% Guard
\newcommand{\Guard}[2][\EventReq]{
    #1 \mathbin{\&} #2}

%% Some macros for Distributed-CEGIS-min algo
\newcommand{\CommonArgs}{{*}\mkern-0.5mu{*}\mkern-0.5mu{*}}
\newcommand{\ModularC}[1][m]{% [<module>]
    C^{(#1)}}
\newcommand{\CompoundNumberOfStates}{D}
\newcommand{\StatesUsed}{C^{\textnormal{used}}}
\newcommand{\StatesUsedMin}{\StatesUsed_{\min}}
\newcommand{\NodesUsed}{N^{\textnormal{used}}}
\newcommand{\NodesUsedMin}{\NodesUsed_{\min}}
\newcommand{\SetNegativeCompoundScenariosNew}{\SetNegativeCompoundScenarios_{\text{new}}}

%% Some macros for Distributed-CEGIS experiments
% \newcommand{\SetScenariosCustom}[2]{% {<number>}{<length>}
%     \SetScenarios_{#1 \times #2}}
% \newcommand{\LTLSpecSafety}{\LTLSpec_{\textit{safe}}}
% \newcommand{\LTLSpecLiveness}{\LTLSpec_{\textit{live}}}
% \newcommand{\LTLSpecSafetyLiveness}{\LTLSpecSafety + \LTLSpecLiveness}
% \newcommand{\LTLSpecAll}{\LTLSpec_{\text{\itshape all}}}

\newcommand{\eg}{\textit{e.g.},\ }
\newcommand{\ie}{\textit{i.e.}\ }
\newcommand{\etc}{\textit{etc}.\@\spacefactor=\the\sfcode`\c}

%% Paragraph style
\AtBeginDocument{
    % beforeskip = vertical skip before \paragraph
    \setbeforeparaskip{1ex plus 1ex minus 0.2ex}
    % indent = horizontal indent of \paragraph
    \setparaindent{2.5em}
    % afterskip = horizontal skip after \paragraph
    \setafterparaskip{-0.5em}
}
