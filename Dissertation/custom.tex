\usepackage{physics}
\usepackage[Export]{adjustbox}
\usepackage{bm}
\usepackage{suffix}

% Выводы по главе
\newcommand{\chapterconclusion}{\section*{Выводы по главе~\thechapter}\addcontentsline{toc}{section}{Выводы по главе~\thechapter}}

%% Hyphenation penalty
\hyphenpenalty=500 % Default is 50
\tolerance=200 % Default is 200

%% Big operators factory
\newcommand\newbigop[2]{% {<name>}{<symbol>}
    \expandafter\newcommand\csname big#1\endcsname{%
        #2\limits}
    \expandafter\newcommand\csname big#1nolim\endcsname{%
        #2\nolimits}
    \expandafter\newcommand\csname big#1clap\endcsname[1]{% {<sub>}
        #2\limits_{\mathclap{##1}}}
    \expandafter\newcommand\csname big#1smashl\endcsname[1]{% {<sub>}
        \smashoperator[l]{#2_{##1}}}
    \expandafter\newcommand\csname big#1smashr\endcsname[1]{% {<sub>}
        \smashoperator[r]{#2_{##1}}}
}
%% Big operators
\newbigop{land}{\bigwedge}
\newbigop{lor}{\bigvee}
\newbigop{union}{\bigcup}
\newbigop{intersection}{\bigcap}
\newbigop{sum}{\sum}
\newbigop{prod}{\prod}
\newbigop{xor}{\bigoplus}

%% Set declaration syntax (from mathtools manual)
\providecommand\given{}
\newcommand\GivenSymbol[1][]{%
    \nonscript\:
    #1\vert\allowbreak
    \nonscript\:
    \mathopen{}
}
\DeclarePairedDelimiterX\Set[1]\{\}{%
    \renewcommand\given{\GivenSymbol[\delimsize]}
    #1
}

%% Pair and tuple declaration syntax
\DeclarePairedDelimiter{\Pair}\langle\rangle
\DeclarePairedDelimiter{\Tuple}\langle\rangle

%% Rounding up/down (ceil/floor) operations
\DeclarePairedDelimiter{\Ceil}\lceil\rceil
\DeclarePairedDelimiter{\Floor}\lfloor\rfloor
\DeclarePairedDelimiter{\Round}\lbrack\rbrack

%% Union and intersection operators
\newcommand\union{\cup}
\newcommand\intersection{\cap}

%% Xor operator
\newcommand\xor{\oplus}
\newcommand\lxor{\mathbin{\veebar}}

%% (mod n) operator
\newcommand\Mod[1]{(\mathrm{mod}\ #1)}

%% Entailment operator
\newcommand\entails{\vDash}
\newcommand\nentails{\nvDash}

%% Empty and universal sets
\renewcommand\emptyset{\varnothing}
\newcommand\universalset{\mathfrak{U}}

%% Powerset operation (via "Weierstrass p")
\newcommand{\powerset}[1]{% {<arg>}
    \mathcal{P}({#1})}

%% Set cardinality operation
\DeclarePairedDelimiter{\card}\lvert\rvert

%% Absolute value and norm operations
% Note: use "physics" package instead
% \DeclarePairedDelimiter{\abs}\lvert\rvert
% \DeclarePairedDelimiter{\norm}\lVert\rVert

%% Two dots for integer intervals: [1..C] = [1 \dd C]
% \newcommand{\twodots}{\mathinner{\ldotp\ldotp}}
\newcommand{\twodots}{\mathinner{..}}
\let\dd\twodots

%% Range operation [low .. high]
% Note: outer braces ensure non-breakable range
\newcommand{\Range}[2]{% {<low>}{<high>}
    {[#1 \dd #2]}}

%% Boolean space (B = {True, False})
\newcommand{\Bool}{\mathbb{B}}

%% Boolean vector
\newcommand{\boolvec}[1]{\Bool^{\card{#1}}}
\let\BoolVec\boolvec

%% Struts
\newcommand\TopStrut{\rule{0pt}{2.6ex}}
\newcommand\BotStrut{\rule[-1.2ex]{0pt}{0pt}}

%% Single-line arrows for implication and iff
% \def\implies{\mathrel{\rightarrow}}
% \def\impliedby{\mathrel{\leftarrow}}
% \def\iff{\mathrel{\leftrightarrow}}

%% Fix small caps when babel is loaded with russian language
%% Note: font must support english small caps!
\newcommand{\smallcaps}[1]{% {<text>}
    \texorpdfstring{%
        \begingroup%
            \selectlanguage{english}%
            \textsc{#1}%
        \endgroup%
    }{#1}}

% %% Maximum height of float (on 'p' page)
% \newcommand{\maxheight}[1]{% {<lines>}
%     \textheight% Total text area height
%     -#1\baselineskip% Line height
%     -\abovecaptionskip% Space above caption
%     -\belowcaptionskip% Space below caption
%     -0.1pt% Just because something is a little bit off after all
% }

%% Input/Output actions
\newcommand{\ActionMM}[2]{% {<event>}{<values>}
    \ensuremath{#1[#2]}}
\newcommand{\ActionTT}[2]{% {<event>}{<values>}
    \texttt{#1[#2]}}
\newcommand{\ActionMT}[2]{% {<event>}{<values>}
    \ensuremath{#1}\texttt{[#2]}}
\newcommand{\ActionTM}[2]{% {<event>}{<values>}
    \texttt{#1}\ensuremath{[#2]}}
\let\Action\ActionMM
\newcommand{\InputEvent}{e^{I\!}}
\newcommand{\OutputEvent}{e^{O\!}}
\newcommand{\InputValues}{\bar{x}}
\newcommand{\OutputValues}{\bar{z}}
\newcommand{\InputAction}[1][]{% [<index>]
    \Action{\InputEvent_{#1}}{\InputValues_{#1}}}
\newcommand{\OutputAction}[1][]{% [<index>]
    \Action{\OutputEvent_{#1}}{\OutputValues_{#1}}}
% \newcommand{\InputActionEU}[1][]{% [<index>]
%     \Action{e_{#1}}{u_{#1}}}
% \newcommand{\OutputActionOY}[1][]{% [<index>]
%     \Action{o_{#1}}{y_{#1}}}

%% NuSMV execution state
\newcommand{\ExecutionState}[2]{% {<state>}{<output-action>}
    [#1 / #2]}

%% Over/under brace/braket with equality
\newcommand{\overbraceeq}[4][\;=\;]{% [<mid>]{<lhs>}{<rhs>}{<body>}
    \overbrace{#4}^{\smash[b]{\mathllap{#2} #1 \mathrlap{#3}}}}
\newcommand{\underbraceeq}[4][\;=\;]{% [<mid>]{<lhs>}{<rhs>}{<body>}
    \underbrace{#4}_{\mathllap{#2} #1 \mathrlap{#3}}}
\newcommand{\overbracketeq}[4][\;=\;]{% [<mid>]{<lhs>}{<rhs>}{<body>}
    \overbracket{#4}^{\mathllap{#2} #1 \mathrlap{#3}}}
\newcommand{\underbracketeq}[4][\;=\;]{% [<mid>]{<lhs>}{<rhs>}{<body>}
    \underbracket{#4}^{\mathllap{#2} #1 \mathrlap{#3}}}

%% AtLeastOne / AtMostOne / ExactlyOne operations
\newcommand{\AtLeastOne}[2][]{\mathrm{ALO}_{#1}(#2)}
\newcommand{\AtMostOne}[2][]{\mathrm{AMO}_{#1}(#2)}
\newcommand{\ExactlyOne}[2][]{\mathrm{EO}_{#1}(#2)}

%% AtLeastOne / AtMostOne / ExactlyOne operations
% \newcommand{\ALO}[2][]{\mathrm{ALO}_{#1}(#2)}
% \newcommand{\AMO}[2][]{\mathrm{AMO}_{#1}(#2)}
% \newcommand{\EO}[2][]{\mathrm{EO}_{#1}(#2)}

% Temporarily add \temp command for temporal operators...
% \newcommand{\temp}[1]{\textbf{#1}}
% Some mysterious stuff for LTL formulas
% \newcommand{\com}[1]{\ensuremath{\mathtt{#1}}}

%% Add hat for NegativeScenarioTree-related stuff
\newcommand{\Negative}[1]{\vphantom{#1}\smash{\widehat{#1}}}

%% Function name
\newcommand{\funcname}[1]{% {<name>}
    \operatorname{\mathit{#1}}}

%% Domain function
\newcommand{\Dom}{\operatorname{Dom}}

%% Function with parens
\newcommand{\NewFunction}[2]{% {<command>}{<name>}
    \expandafter\newcommand\csname #1\endcsname[1]{% {<arg>}
        \funcname{#2}(##1)}}
\newcommand{\NewNegativeFunction}[2]{% {<command>}{<name>}
    \expandafter\newcommand\csname #1\endcsname[1]{% {<arg>}
        \Negative{\funcname{#2}}(##1)}}
\newcommand{\NewFunctionWithNegative}[3][neg]{% [<neg-cmd-prefix>]{<command>}{<name>}
    \NewFunction{#2}{#3}
    \NewNegativeFunction{#1#2}{#3}
}

%% Variables factory
\newcommand{\NewVariableEx}[4]{% {<name>}{<symbol>}{<pre-super>}{<pre-sub>}
    \expandafter\newcommand\csname #1\endcsname[2][]{% [<super>]{<sub>}
        #2^{#3 ##1}_{#4 ##2}}
    \WithSuffix\expandafter\newcommand\csname #1\endcsname*{#2}
}
\newcommand{\NewVariable}[2]{% {<name>}{<symbol>}
    \NewVariableEx{#1}{#2}{}{}}
\newcommand{\NewModularVariable}[1]{% {<variable>}
    \expandafter\newcommand\csname Modular#1\endcsname[2][m]{% [<module>]{<sub>}
        \csname #1\endcsname[(##1)\!]{##2}}}

%% Tightly spaced text
\newcommand{\tight}[1]{% {<text>}
    \textls[-50]{#1}}
\newcommand{\tightit}[1]{% {<text>}
    % \tight{$\mathit{#1}$}}
    \textit{\tight{#1}}}

%% Common
\newcommand{\Automaton}{\mathcal{A}}
\newcommand{\FunctionBlock}{FB}
\newcommand{\SetStates}{Q}
\newcommand{\SetStatesAux}{\SetStates_{0}}
\newcommand{\InitialState}{q_{\text{init}}}
\newcommand{\SetInputEvents}{E^{I\!}}
\newcommand{\SetOutputEvents}{E^{O\!}}
\newcommand{\SetInputVariables}{\mathcal{X}}
\newcommand{\SetOutputVariables}{\mathcal{Z}}
\newcommand{\AssociationInput}{W^{I\!}}
\newcommand{\AssociationOutput}{W^{O\!}}
\newcommand{\SetTreeInputs}{\mathcal{U}}
\newcommand{\SetTreeOutputs}{\mathcal{Y}}
\newcommand{\SetScenarios}{\mathcal{S}}
\newcommand{\SetPositiveScenarios}{\SetScenarios^{+}}
\newcommand{\SetNegativeScenarios}{\SetScenarios^{-}}
\newcommand{\LTLSpec}{\mathcal{L}}
\newcommand{\Tree}{\mathcal{T}}
\newcommand{\PositiveTree}{\Tree^{+}}
\newcommand{\NegativeTree}{\Tree^{-}}
\newcommand{\TreeRoot}{\rho}
\newcommand{\SetTreeNodes}{V}
\newcommand{\SetTreeNodesActive}{\SetTreeNodes^{\text{(active)}}}
\newcommand{\SetTreeNodesPassive}{\SetTreeNodes^{\text{(passive)}}}
\newcommand{\Modules}{\Range{1}{M}}
\newcommand{\SetAllInputs}{\SetTreeInputs^{*}}
\newcommand{\InitialOutputValues}{\OutputValues_{\text{init}}}
\newcommand{\PreviousOutputValues}{\OutputValues^{\text{prev}}}
% \let\oldv\v
% \def\v{v}
\newcommand{\NegativeTreeRoot}{\Negative{\TreeRoot}}
\newcommand{\SetNegativeTreeNodes}{\Negative{\SetTreeNodes}}
\newcommand{\negv}{\Negative{v}}
\newcommand{\SetNegativeTreeNodesActive}{\SetNegativeTreeNodes^{\text{(active)}}}
\newcommand{\SetNegativeTreeNodesPassive}{\SetNegativeTreeNodes^{\text{(passive)}}}
\newcommand{\SetNegativeTreeNodesEnds}{\SetNegativeTreeNodes^{\text{(ends)}}}
\NewFunctionWithNegative{tp}{\funcname{tp}}
\NewFunctionWithNegative{tpa}{\funcname{tpa}}
\NewFunctionWithNegative{tie}{\funcname{tie}}
\NewFunctionWithNegative{toe}{\funcname{toe}}
\NewFunctionWithNegative{tin}{\funcname{tin}}
% \NewFunctionWithNegative{ton}{\funcname{ton}}
\NewFunctionWithNegative{tiv}{\funcname{tiv}}
\NewFunctionWithNegative{tov}{\funcname{tov}}
\NewNegativeFunction{negtlb}{\funcname{tlb}}
\NewNegativeFunction{negtbe}{\funcname{t\smash{b}e}}
\newcommand{\toeMulti}[2][r]{% {<arg>}
    \funcname{\funcname{toe}}^{{[#1]}}(#2)}
\newcommand{\tovMulti}[2][r]{% {<arg>}
    \funcname{\funcname{tov}}^{{[#1]}}(#2)}

%% Modular common
\newcommand{\SetPins}{\Pi}
\newcommand{\SetInboundVarPins}{\SetPins^{\mathit{in\;var\!}}}
\newcommand{\SetOutboundVarPins}{\SetPins^{\mathit{out\;var\!}}}
\newcommand{\SetInboundEventPins}{\SetPins^{\mathit{in\;ev\!}}}
\newcommand{\SetOutboundEventPins}{\SetPins^{\mathit{out\;ev\!}}}
\newcommand{\ModularSetInboundVarPins}[1][m]{\SetInboundVarPins_{(#1)}}
\newcommand{\ModularSetOutboundVarPins}[1][m]{\SetOutboundVarPins_{(#1)}}
\newcommand{\ModularSetInboundEventPins}[1][m]{\SetInboundEventPins_{(#1)}}
\newcommand{\ModularSetOutboundEventPins}[1][m]{\SetOutboundEventPins_{(#1)}}
\newcommand{\SetExternalInboundVarPins}{\SetInboundVarPins_{(\mathit{ext})}}
\newcommand{\SetExternalOutboundVarPins}{\SetOutboundVarPins_{(\mathit{ext})}}
\newcommand{\SetExternalInboundEventPins}{\SetInboundEventPins_{(\mathit{ext})}}
\newcommand{\SetExternalOutboundEventPins}{\SetOutboundEventPins_{(\mathit{ext})}}

%% Interface variables
% \NewVariable{TransitionFunction}{\lambda}
% \NewVariable{OutputEventFunction}{\psi}
% \NewVariable{OutputFunction}{\omega}
% \NewVariable{AlgorithmFunction}{\OutputFunction*}

%% BFS variables
\newcommand{\BfsTransitionSymbol}{\tau}
\newcommand{\BfsParentSymbol}{\pi}
\NewVariableEx{BfsTransitionAutomaton}{%
    \BfsTransitionSymbol}{\smallcaps{bfs-}\mathcal{A}}{}
\NewVariableEx{BfsParentAutomaton}{%
    \BfsParentSymbol}{\smallcaps{bfs-}\mathcal{A}}{}
\NewVariableEx{BfsTransitionGuard}{%
    \BfsTransitionSymbol}{\smallcaps{bfs-}\mathcal{G}}{}
\NewVariableEx{BfsParentGuard}{%
    \BfsParentSymbol}{\smallcaps{bfs-}\mathcal{G}}{}

%% Core variables
\NewVariable{StateOutputEvent}{\phi}
\NewVariable{StateAlgorithm}{\gamma}
\NewVariable{TransitionDestination}{\tau}
\NewVariable{TransitionInputEvent}{\xi}
\NewVariable{TransitionTruthTable}{\theta}
\NewVariable{TransitionFiring}{\delta}
\NewVariable{TransitionFirstFired}{\mathit{ff}} % ligature!
\NewVariable{TransitionNotFired}{\mathit{nf}}
\NewVariable{ActualTransitionFunction}{\lambda}

%% Mapping variables
\NewVariable{Mapping}{\mu}
% \NewVariable{MemoryOutputValue}{\zeta}
\NewVariable{MemoryOutputValue}{\nabla}
\NewVariable{ComputedValue}{\nabla}

%% Guards variables
\NewVariable{NodeType}{\eta}
\newcommand{\NodeTypeTerminal}{\square}
\newcommand{\NodeTypeAnd}{\mathord{\land}}
\newcommand{\NodeTypeOr}{\mathord{\lor}}
\newcommand{\NodeTypeNot}{\mathord{\neg}}
\newcommand{\NodeTypeNone}{\mathord{\bullet}}
\NewVariable{NodeInputVariable}{\chi}
\NewVariable{NodeChild}{\sigma}
\NewVariable{NodeParent}{\pi}
\NewVariable{NodeValue}{\vartheta}

%% Negative variables
% \NewVariable{NegativeTransitionFunction}{\Negative{\TransitionFunction*}}
\NewVariable{NegativeActualTransitionFunction}{\Negative{\ActualTransitionFunction*}}
% \NewVariable{NegativeOutputEventFunction}{\Negative{\OutputEventFunction*}}
% \NewVariable{NegativeOutputFunction}{\Negative{\OutputFunction*}}
% \NewVariable{NegativeAlgorithmFunction}{\Negative{\AlgorithmFunction*}}
\NewVariable{NegativeMapping}{\Negative{\Mapping*}}

%% Modular variables
\NewVariable{ModuleControllingOutputVariable}{\zeta}
\NewModularVariable{StateOutputEvent}
\NewModularVariable{StateOutput}
\NewModularVariable{StateAlgorithm}
\NewModularVariable{TransitionDestination}
\NewModularVariable{TransitionInputEvent}
\NewModularVariable{TransitionTruthTable}
\NewModularVariable{TransitionFiring}
\NewModularVariable{TransitionFirstFired}
\NewModularVariable{TransitionNotFired}
\NewModularVariable{TransitionFunction}
\NewModularVariable{ActualTransitionFunction}
\NewModularVariable{OutputEventFunction}
\NewModularVariable{OutputFunction}
\NewModularVariable{AlgorithmFunction}
\NewModularVariable{Mapping}
\NewModularVariable{ComputedValue}
\NewVariableEx{VarPinParent}{\pi}{\mathit{var}}{}
\NewVariableEx{EventPinParent}{\pi}{\mathit{ev}}{}
\NewVariableEx{InboundVarPinComputedValue}{\ComputedValue*}{\mathit{in}}{}
\NewVariableEx{OutboundVarPinComputedValue}{\ComputedValue*}{\mathit{out}}{}
\NewVariable{InputIndex}{\partial}
\NewModularVariable{InputIndex}{}

%% Extra
\newcommand{\Cmin}{C_{\text{min}}}
\newcommand{\Tmin}{T_{\text{min}}}
\newcommand{\Nmin}{N_{\text{min}}}
\newcommand{\AutomatonBest}{\Automaton_{\text{best}}}

%% Algorithms
\newcommand{\NewAlgorithm}[2]{% {<command>}{<name>}
    \expandafter\newcommand\csname Algo#1\endcsname{%
        \smallcaps{#2}}
    \expandafter\newcommand\csname Algo#1Full\endcsname{%
        \texorpdfstring{\ensuremath{\smallcaps{#2}^{*}}}{#2*}}
}
\NewAlgorithm{Basic}{Basic}
\NewAlgorithm{Extended}{Extended}
\NewAlgorithm{Complete}{Complete}
\NewAlgorithm{BasicMin}{Basic-min}
\NewAlgorithm{BasicMinC}{Basic-minC}
\NewAlgorithm{BasicMinT}{Basic-minT}
\NewAlgorithm{ExtendedMin}{Extended-min}
\NewAlgorithm{ExtendedMinN}{Extended-minN}
\NewAlgorithm{ExtendedMinUB}{Extended-min-UB}
\NewAlgorithm{CompleteMin}{Complete-min}
% \NewAlgorithm{CompleteMinN}{Complete-minN}
\NewAlgorithm{CompleteMinUB}{Complete-min-UB}
\NewAlgorithm{Cegis}{CEGIS}
\NewAlgorithm{CegisMin}{CEGIS-min}
\NewAlgorithm{CegisUB}{CEGIS-UB}
% Modular Parallel
\NewAlgorithm{ModularParallelBasic}{Parallel-Basic}
\NewAlgorithm{ModularParallelExtended}{Parallel-Extended}
\NewAlgorithm{ModularParallelComplete}{Parallel-Complete}
\NewAlgorithm{ModularParallelBasicMin}{Parallel-Basic-min}
\NewAlgorithm{ModularParallelExtendedMin}{Parallel-Extended-min}
% Modular Consecutive
\NewAlgorithm{ModularConsecutiveBasic}{Consecutive-Basic}
\NewAlgorithm{ModularConsecutiveExtended}{Consecutive-Extended}
\NewAlgorithm{ModularConsecutiveComplete}{Consecutive-Complete}
\NewAlgorithm{ModularConsecutiveBasicMin}{Consecutive-Basic-min}
\NewAlgorithm{ModularConsecutiveExtendedMin}{Consecutive-Extended-min}
% Modular Arbitrary
\NewAlgorithm{ModularArbitraryBasic}{Arbitrary-Basic}
\NewAlgorithm{ModularArbitraryExtended}{Arbitrary-Extended}
\NewAlgorithm{ModularArbitraryComplete}{Arbitrary-Complete}
\NewAlgorithm{ModularArbitraryBasicMin}{Arbitrary-Basic-min}
\NewAlgorithm{ModularArbitraryExtendedMin}{Arbitrary-Extended-min}


%% LTL-properties
\newcommand{\Prop}[1]{% {<index>}
    \IfInteger{#1}{%
        \ensuremath{\varphi_{#1}}%
    }{%
        \PackageError{Prop}{Undefined option to Prop: #1}{}%
    }%
}
\newcommand{\LTLProp}[1]{% {<index>}
    \IfEqCase{#1}{%
        {1}{\temp{G}(\neg(\texttt{c1Extend} \land \texttt{c1Retract}))}%
        {2}{\temp{G}(\neg(\texttt{c2Extend} \land \texttt{c2Retract}))}%
        {3}{\temp{G}(\neg(\texttt{vacuum\_on} \land \texttt{vacuum\_off}))}%
        {4}{\temp{G}(\neg\texttt{vcHome} \land \neg\texttt{vcEnd} \implies \texttt{c1Home} \lor \texttt{c1End})}%
        {5}{\temp{G}(\neg\texttt{c1Home} \land \neg\texttt{c1End} \implies \texttt{vcHome} \lor \texttt{vcEnd})}%
        {6}{\temp{G}(\texttt{all\_home} \land \neg\texttt{pp1} \land \neg\texttt{pp2} \land \neg\texttt{pp3} \land \neg\texttt{lifted} \implies \temp{X}(\neg\texttt{c1Extend} \land \neg\texttt{c2Extend} \land \neg\texttt{vcExtend}))}%
        {7}{\temp{G}(\texttt{lifted} \implies \temp{F}(\texttt{dropped}))}%
        {8}{\temp{G}(\texttt{pp1} \implies \temp{F}(\texttt{vp1}))}%
        {9}{\temp{G}(\texttt{pp2} \implies \temp{F}(\texttt{vp2}))}%
        {10}{\temp{G}(\texttt{pp3} \implies \temp{F}(\texttt{vp3}))}%
    }[\PackageError{LTLProp}{Undefined option to LTLProp: #1}{}]
}
\newcommand{\PropWP}[1]{% {<1-3>}
    \IfEqCase{#1}{%
        {1}{\Prop{8}}%
        {2}{\Prop{9}}%
        {3}{\Prop{10}}%
    }[\PackageError{PropWP}{Undefined option to PropWP: #1}{}]
}
\newcommand{\LTLPropWP}[1]{% {<1-3>}
    \IfEqCase{#1}{%
        {1}{\LTLProp{8}}%
        {2}{\LTLProp{9}}%
        {3}{\LTLProp{10}}%
    }[\PackageError{LTLPropWP}{Undefined option to LTLPropWP: #1}{}]
}
\newcommand{\PropertiesConst}{\Prop{1}\text{\==}\Prop{7}}
\newcommand{\PropertiesVar}{\PropWP{1}\text{\==}\PropWP{3}}
