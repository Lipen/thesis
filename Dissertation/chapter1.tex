% \chapter{Базовые понятия и результаты}
\chapter{Обзор предметной области}
\label{ch:overview}

В данной главе представлены основные понятия и существующие результаты.


\section{Дискретные управляющие модели}
\label{sec:discrete-control-models}

Здесь описываются дискретные управляющие модели и их применение.

\subsection{Конечные автоматы}

Конечный автомат (КА) --- это модель вычислений, которая описывает систему с конечным числом состояний и переходов между ними.
Конечный автомат может быть задан в виде пятерки $\mathcal{A} = \Tuple{\Sigma, Q, q_0, F, \delta}$, где:
\begin{itemize}
    \item $\Sigma$ --- алфавит входных символов;
    \item $Q$ --- (\emph{конечное}) множество состояний;
    \item $q_0 \in Q$ --- начальное состояние;
    \item $F \subseteq Q$ --- множество терминальных (принимающих) состояний;
    \item $\delta \colon Q \times \Sigma \to Q$ --- функция переходов.
\end{itemize}
Конечный автомат \emph{принимает} (\textit{accepts}) слово $w = w_1 w_2 \ldots w_n \in \Sigma^*$, если после прочтения $w$ автомат оказывается в одном из терминальном состоянии $s_n \in F$, то есть существует последовательность состояний $s_0, s_1, \ldots, s_n$ такая, что $s_0 = q_0$, $s_{i+1} = \delta(s_i, w_{i+1})$ для всех $i \in \Set{0, 1, \ldots, n-1}$ и $s_n \in F$.


\subsection{Булевы схемы}

% Булева схема может быть задана в виде кортежа $\mathcal{C} = \Tuple{I, O, G, L}$, где:
% \todo{пофиксить кортеж}
% \begin{itemize}
%     \item $I$ --- множество входных вершин (\textit{inputs});
%     \item $O$ --- множество выходных вершин (\textit{outputs});
%     \item $G$ --- множество внутренних вершин (\textit{gates});
% % \item $W \subseteq I \union G$ --- множество входных вершин, которые не являются входами схемы (\textit{wires});
%     \item $L \colon G \to \Set{\land, \lor, \neg, \dots}$ --- функция, которая каждой внутренней вершине сопоставляет логическую функцию.
% \end{itemize}

Булева схема \--- это направленный ациклический ориентированный граф $G = \Pair{V, E}$, где $V$ \=== множество вершин, а $E \subseteq V^2$ \=== множество рёбер (\textit{дуг}).
Вершины такого графа делятся на три типа: входные вершины (\textit{inputs}), выходные вершины (\textit{outputs}) и внутренние вершины (\textit{gates}).
\textit{Ребро} (\textit{дуга}) представляет собой упорядоченную пару вершин.
Для каждой дуги $(u,v) \in E$, вершина $u$ называется \textit{родителем}~$v$, а $v$ \=== \textit{потомком} $u$.
Множество всех родителей вершины~$v$ обозначается как~$P_v$.
Вершина называется \textit{входной}, если у нее нет родителей, и \textit{выходной}, если у нее нет потомков\footnote{Здесь стоит учитывать, что вполне возможны вариации данных определений. В некоторых ситуациях \textit{входными}/\textit{выходными} вершинами в схеме считаются некоторые заранее выбранные вершины, но при этом у них могут быть родители/потомки, соотвественно. В зависимости от контекста, эти родители/потомки игнорируются в соответствующих определениях, связанных с обходом вершин графа от \textit{входов} к \textit{выходам}.}.
Множества входов и выходов обозначаются как $\Vin \subseteq V$ и $\Vout \subseteq V$ соответственно.
Любая вершина $v \in V \setminus \Vin$ называется \textit{гейтом} (\textit{логическим вентилем}).
В булевой схеме каждому гейту сопоставляется некоторый \textit{логический элемент}~\cite{wegener1987} из предопределенного набора, называемого \textit{базисом} (например, $\{\land, \neg\}$).
Таким образом, любой логический элемент интерпретирует некоторую элементарную булеву функцию.
Пример булевой схемы представлен на \cref{fig:boolean-circuit-example}.

\begin{figure}[ht]
    \centering
    % \includegraphics[max width=0.5\textwidth]{example-image}
    \begin{adjustbox}{max width=\linewidth}
        \subfile{tex/tikz-circuit-example}
    \end{adjustbox}%
    \caption{Пример булевой схемы с тремя входами ($i_1$,~$i_2$,~$i_3$) и восьмью гейтами}
    \label{fig:boolean-circuit-example}
\end{figure}

Булева схема с $n$~входами и $m$~выходами естественным образом задает (тотальную) дискретную функцию $f \colon \{0, 1\}^n \to \{0, 1\}^m$, где под $\{0,1\}^k$ мы понимаем множество всех возможных двоичных слов длины~$k \in \Natural^{+}$.
Имея это в виду, мы будем использовать обозначение~$S_f$ для представления булевой схемы, задающей функцию~$f$.
Каждому гейту схемы $S_f$ сопоставлена булева функция, которая соответствует логическому элементу, назначенному этому гейту.

Пусть $\alpha\in\{0,1\}^n$ произвольное слово, поданное на вход~$S_f$.
Проходя по гейтам схемы в фиксированном порядке (обычно указанном топологической сортировкой~\cite{cormen90}) и вычисляя значения элементарных функций, сопоставленных гейтам, мы получаем значение функции~$f$ на входном слове~$\alpha$ в качестве результата.
Этот процесс называется \textit{интерпретацией} схемы~$S_f$ на входе~$\alpha$.

Каждой вершине в схеме~$S_f$ сопоставим булеву переменную.
Обозначим множество переменных, ассоциированных с входами $\Vin$ схемы~$S_f$, как $\Xin = \{x_1,\ldots,x_n\}$.
Переменные, связанные с гейтами, мы будем называть \textit{вспомогательными} (\textit{auxiliary}).
Пусть $u$ \=== некоторая вспомогательная переменная, соответствующая гейту~$v$, и $U_v$ \=== множество переменных, связанных с вершинами из~$P_v$.
Предположим, что $h_v$ \=== булева функция, соответствующая логическому элементу, назначенному гейту~$v$, и $F(h_v)$ \=== булева формула над~$U_v$, которая задает функцию~$h_v$.
Обозначим через $C_v$ КНФ-представление формулы $F(h_v) \equiv u$.

Рассмотрим следующую КНФ:
\begin{equation}\label{eq1}
    C_f = \biglandclap{v \in V \setminus \Vin} C_v
\end{equation}
Мы будем обозначать~\eqref{eq1} как \textit{шаблонную КНФ} для функции~$f$.
Заметим, что $C_f$~является КНФ-формулой, полученной применением преобразований Тсейтинина~\cite{tseitin1970} к схеме~$S_f$.

Ниже, следуя работе~\cite{DBLP:journals/jar/Szeider05}, будем использовать обозначение~$x^{\sigma}$, где $\sigma \in \{0,1\}$, предполагая, что $x^0$ обозначает отрицательный литерал~$\neg x$, а $x^1$ обозначает положительный литерал~$x$, а также обозначение $\{0,1\}^{|B|}$, что означает множество всех возможных назначений переменных из~$B$.
%Пусть~$F$ будет произвольной булевой формулой над переменными~$X$.
%Обозначим через~$F|_{x=\sigma}$ формулу, полученную подстановкой~$x$ на место~$\sigma$ в~$F$~\cite{chang1973}.
%Очевидно, что формы $x^\sigma\land F$ и $F|_{x=\sigma}$ являются равноправными по удовлетворимости.
%Таким образом, когда мы работаем с формулой $x^\sigma \land F$, мы можем рассматривать единичную дизъюнкцию~$x^\sigma$ как значение~$\sigma$ переменной~$x$ в~$F$.
%Для произвольного набора булевых переменных~$B$ через $\{0,1\}^{|B|}$ мы обозначаем множество всех возможных назначений переменных из~$B$.
Следующий факт был многократно установлен в литературе, например, см.~\cite{bessiere2009,drechsler2009}.
Он использует простой механизм булевой дедукции, известный как правило распространения единичного дизъюнкта (Unit Propagation \--- UP)~\cite{marques-silva2009}.

\begin{lemma}\label{lem1}
    Применение UP к КНФ-формуле $x_1^{\alpha_1} \land \dots \land x_n^{\alpha_n} \land C_f$ для любого $\alpha = (\alpha_1, \dots, \alpha_n)$, $\alpha \in \{0,1\}^{|\Xin|}$ выводит (в форме единичных дизъюнкций) значения всех переменных, связанных с гейтами из $V \setminus \Vin$, включая переменные $y_1, \dots, y_m$, связанные с выходами схемы $S_{f}$: $y_1=\gamma_1, \dots, y_m=\gamma_m$, $f(\alpha) = \gamma = (\gamma_1, \dots, \gamma_m)$.
\end{lemma}

Стоит отметить, что \cref{lem1} в сущности означает, что процесс интерпретации схемы~$S_f$ на входном слове~$\alpha$ может быть смоделирован последовательным применением UP к КНФ $x_1^{\alpha_1} \land \dots \land x_n^{\alpha_n} \land C_f$ для любого $\alpha = (\alpha_1, \dots, \alpha_n)$.
\Cref{lem1} очень полезно при доказательстве свойств, связанных с булевыми схемами и SAT.


\section{Задачи синтеза и верификации дискретных управляющих моделей}
\label{sec:synthesis-and-verification}

\subsection{Международный стандарт IEC~61499}%
\label{sub:iec61499}

% TODO: Здесь должно быть описание стандарта IEC~61499, функциональных блоков и так далее.

Международный стандарт распределенных систем управления и автоматизации IEC~61499~\cite{vyatkin-tii} нацелен на упрощение разработки распределенных киберфизических систем.
Стандарт отличается от \enquote{предыдущего} стандарта IEC~61131\cite{iec-61131} тем, что в IEC~61499 используется событийная модель исполнения.
Этот стандарт предлагает использование так называемых \emph{функциональных блоков} (ФБ), являющихся, по-сути, контейнерами для базовых элементов \--- управляющих конечных автоматов.
Основные типы описываемых в стандарте IEC~61499 функциональных блоков \--- \emph{базовые} и \emph{композитные}.
Функционал композитных блоков определяется сетью базовых ФБ.
Базовые ФБ являются совокупностью интерфейса (описания входных и выходных событий и переменных) и управляющего конечного автомата (\textit{Execution Control Chart} \--- ECC).
% Подробное описание формальной модели~\cite{dubinin-2006} базового функционого блока предоставлено в разделе~\ref{sec:basic-fb-model}.

% TODO: more?

\subsection{Методы синтеза конечно-автоматных моделей}
\label{sub:automata-synthesis}

% TODO: More intro here?

Задача поиска минимального детерминированного конечно автомата по примерам поведения является NP-полной задачей~\cite{gold}, а сложность задачи LTL-синтеза дважды экспоненциальная от размера LTL\-/спецификации.
Не смотря на это, синтез различных типов конечно-автоматных моделей по примерам поведения и/или формальной спецификации был исследован во многих научных работах~\cite{heule2010,efsm-tools,zakirzyanov2019,buzhinsky-tii,bosy,tsarev-egorov-gecco,giantamidis-tripakis,petrenko,petrenko2,neider,g4ltl-st,smetsers-lata}, где используются методы, основанные на эвристическом объединении состояний (\emph{state merging}), эволюционные алгоритмы, а также методы, основанные на применении SAT- и SMT-решателей.
В данной работе рассматриваются только точные и эффективные методы, поэтому внимание уделяется методам с применением SAT-решателей.

Расширенный конечный автомат (\emph{Extended Finite State Machine} \--- EFSM) является моделью, наиболее близкой к рассматриваемой в данной работе модели ECC\@. EFSM является объединением автомата Мили и Мура, расширенный условными переходами. Переходы в EFSM помечены входными событиями и охранными условиями \--- булевыми функциями от входных переменных, а состояния EFSM имеют ассоциированные выходные действия.
Для синтеза EFSM по примерам поведения и LTL\-/спецификации существует несколько подходов~\cite{efsm-tools,walkinshaw}, основанных на сведении к задаче SAT. В~\cite{efsm-tools} LTL\-/спецификация учитывается путём применения итеративного подхода запрета контрпримеров.
Существенным недостатком~\cite{efsm-tools} является то, что охранные условия должны быть известны заранее, а также то, что синтезируемые алгоритмы в состояниях EFSM являются лишь указаниями на некоторые внешние процедуры.
В~\cite{walkinshaw} решается задача синтеза вычислимых выходных алгоритмов, однако предполагается, что базовая модель автомата (то есть его структура \--- состояния и переходы между ними) известна заранее или получается отдельно.
В общем случае, при использовании исходных данных, получаемых при black-box тестировании системы, информация о внутреннем устройстве системы, а также о доступных внешних процедурах и их поведении, оказывается недоступной, поэтому существующие методы синтеза EFSM не подходят для решения задачи синтеза модели ECC, рассматриваемой в данной работе.

Программное средство BoSy~\cite{bosy,not-bosy} реализует так называемый ограниченный синтез (\emph{bounded synthesis}) системы переходов (\emph{transition system}) по LTL\-/спецификации. Синтез \enquote{ограничен} в том смысле, что позволяет синтезировать систему заданного размера, либо гарантировать отсутствие решения заданного размера.
В BoSy реализовано не только сведение задачи LTL-синтеза к SAT, но также разработано более эффективное (при рассмотренной авторами постановке задачи) сведение с использованием Quantified SAT (QSAT). При использовании SAT-кодировки, синтезируемые системы переходов являются \enquote{явными} (\emph{explicit}) \--- охранные условия на переходах являются полными зависят от всех входных переменных. При использовании QSAT-кодировки, системы получаются \enquote{символьными} (\emph{symbolic}) \--- охранные условия синтезируются в виде полноценных булевых формул.
\mbox{Используемые} в BoSy подход ограниченного синтеза позволяет синтезировать минимальные модели в терминах числа состояний, однако важный вопрос о размере охранных условий обходится стороной \--- синтезируемые модели, как правило, обладают огромными охранными условиями, что сильно затрудняет их восприятие человеком, а также ограничивает применимость таких моделей во встраиваемых системах.
В~\cite{bounded-cycle} предлагается способ упрощения генерируемых моделей, заключающийся в дополнении SAT сведения специальными ограничениями для минимизации числа циклов в системе переходов, однако это слабо влияет на размеры и форму охранных условий.
Также стоит упомянуть, что отличительной особенностью LTL-синтеза является то, что в качестве входных данных не используются примеры поведения, так как предполагается полнота входной спецификации \--- в том смысле, что она описывает все желаемое поведение системы.
Не смотря на то, что примеры поведения могут быть представлены в \mbox{виде} LTL-свойств, этот подход становится крайне неэффективным уже на небольших наборах данных.
Другие программные средства для LTL-синтеза, например G4LTL\=/ST~\cite{g4ltl-st} и Strix~\cite{strix}, обладают аналогичными недостатками по отношению к рассматриваемой задаче, а именно, отсутствие минимизации охранных условий и невозможность (эффективного) учета примеров поведения.

В статье~\cite{fbCSP} предлагается метод \smallcaps{fbCSP} для синтеза конечно-автоматных моделей функциональных блоков по примерам поведения, основанный на сведении к задаче удовлетворения ограничений (\textit{Constraint Satisfaction Problem} \--- CSP).
Однако методу \smallcaps{fbCSP} присущи следующие ограничения.
Получаемые модели обладают \emph{полными} охранными условиями \--- соответствующие булевы формулы зависят от \emph{всех} входных переменных.
Такие модели практически не обобщаются (\textit{generalize}), то есть некорректно работают на входных данных, которые не были использованы в процессе \enquote{обучения} (синтеза).
В~\cite{fbCSP} это отчасти исправляется дополнительной жадной минимизацией охранных условий, однако жадный подход не гарантирует, что охранные условия будут наименьшими.
В работе~\cite{chivilikhin-18} метод \smallcaps{fbCSP} был расширен процедурой запрета контрпримеров для учета LTL\-/спецификации (в дальнейшем это расширение будет называться \smallcaps{fbCSP+LTL}), аналогично работе~\cite{efsm-tools}.
При этом охранные условия в генерируемых моделях представляются в виде конъюнкции литералов входных переменных.
Основным недостатком этого подхода является его низкая эффективность в тех случаях, когда темпоральная спецификация покрыта сценариями выполнения не полностью.

В работе~\cite{chivilikhin-19} разработан двухэтапных подход:
сначала генерируется \emph{базовая} модель с использованием метода, основанного на SAT, а затем охранные условия полученной модели отдельно минимизируются с помощью CSP \--- деревья разбора булевых формул, соответствующих охранным условиям, кодируются в CSP, а затем минимизируется их суммарный размер.
Таким образом, получаемая модель является минимальной, однако независимость двух этапов приводит к тому, что модель не является наименьшей (в терминах суммарного размера охранных условий).

Резюмируя, ни один из рассмотренных методов, каждый из которых по своему хорош при конкретной постановке задачи, не позволяет \emph{одновременно} и \emph{эффективно} учитывать при синтезе конечно-автоматных моделей как (1)~примеры поведения, так и (2)~LTL\-/спецификацию, а также (3)~минимальность генерируемых моделей.
В ходе выполнения данной работы был разработан метод, который фактически является расширением~\cite{chivilikhin-19} \--- объединением двух независимых этапов в один \--- и вносит вклад в расширение \textit{state-of-the-art} конечно-автоматного синтеза с применением SAT-решателей, а именно: \emph{одновременно} поддерживает учет позитивных примеров поведения, реализует индуктивный синтез, основанный на контрпримерах \--- для учета LTL\-/спецификации, а также позволяет генерировать минимальные модели \--- как в терминах числа состояний, так и в терминах суммарного размера охранных условий.

\subsection{Верификация конечно-автоматных моделей}
\label{sub:automata-verification}

\subsection{Линейная темпоральная логика}%
\label{sub:ltl}

% TODO
% TODO
% TODO !
% TODO
% TODO
% TODO.
% Здесь должно быть описание LTL~\cite{ltl} и контрпримеров, получаемых с помощью NuSMV~\cite{NuSMV}. Негативные сценарии выполнения и соотвествующее дерево негативных сценариев описываются отдельно в разделе~\ref{sec:negative-scenarios}

Формальная спецификация может быть проверена с помощью верификатора (\emph{model checker}) \--- специализированного программного средства, которое проверяет выполнение заданных свойств в системе и генерирует контрпримеры к нарушенным свойствам.
В данной работе был использован символьный верификатор NuSMV~\cite{NuSMV}, а рассмотренные спецификация систем были составлены на языке линейной темпоральной логики (\textit{Linear Temporal Logic} \--- LTL)~\cite{ltl}, полностью поддерживаемом NuSMV\@.
Для так называемых \enquote{свойств безопасности} (\textit{safety properties}), выражающих отсутствие нежелательного поведения (например, \enquote{с системой никогда не произойдёт ничего плохого}), контрпримером является конечная последовательность вычислительных состояний (\textit{execution state}), приводящая к нежелательному поведению.
Для так называемых \enquote{свойств живости} (\textit{liveness properties}), выражающих присутствие желаемого поведения (например, \enquote{с системой точно произойдет что-то хорошее}), контрпримером является бесконечная, но циклическая последовательность состояний, представляющая нежелательное циклическое поведение системы, и которая может быть представлена в виде конечного префикса с последующим циклом конечной длины~\cite{clarke1999model}.



\subsection{Синтез булевых формул и схем}
\label{sub:circuits-synthesis}

\subsection{Верификация булевых схем}
\label{sub:circuits-verification}

Задачи LEC и ATPG.

\subsubsection{Задача проверки эквивалентности булевых схем}
\label{subsub:lec}

\subsubsection{Задача генерации тестов для булевых схем}
\label{subsub:atpg}


\section{Задача булевой выполнимости}
\label{sec:sat}

\subsection{Базовые определения}
\label{sub:sat-definitions}

Задача выполнимости булевой формулы (\textit{Boolean satisfiability problem} \--- SAT) формулируется следующим образом: для заданной булевой формулы $\phi(x_1, \dotsc, x_n)$ требуется определить, существует ли такая \emph{удовлетворяющая подстановка} значений переменных $X_{\text{SAT}}$ (также называемая \enquote{модель}), что формула становится истинной: $\phi(X_{\text{SAT}}) = \top$, либо доказать, что такой подстановки не существует~\cite{handbook-sat}.
Задача SAT является первой задачей, для которой была доказана NP-полнота~\cite{cook}.
Различают также задачу FSAT (\textit{Functional} SAT), в которой требуется непосредственно найти удовлетворяющую подстановку (либо также доказать, что ее не существует).
В данной работе эти термины используются неразличимо, то есть под задачей SAT всегда подразумевается ее функциональный вариант.

Если булева формула $\phi$ представлена в конъюктивной нормальной форме (КНФ), то соответствующую задачу называют CNF-SAT.
Любая булева формула может быть преобразована в эквивалентную КНФ, однако при этом размер формулы может увеличиться экспоненциально, например:
\[
    \text{$n$ конъюнкций}
    \left\{
    \begin{aligned}
        & (x_1 \land y_1) \lor \\
        & (x_2 \land y_2) \lor \\
        & ~\dots \\
        & (x_n \land y_n)
    \end{aligned}
    \right.
    \quad
    \xRightarrow{\text{КНФ~}}
    \quad
    \left.
    \begin{aligned}
        & (x_1 \lor x_2 \lor \dotsb \lor x_n) \land \\
        & (y_1 \lor x_2 \lor \dotsb \lor x_n) \land \\
        & ~\dotsb \\
        & (y_1 \lor y_2 \lor \dotsb \lor y_n)
    \end{aligned}
    \right\}
    \text{$2^n$ дизъюнкций}
\]
С помощью преобразований Цейтина~\cite{tseitin} возможно привести любую булеву формулу в КНФ \--- с сохранением выполнимости (\textit{equisatisfiable CNF}), но с добавлением новых переменных (\textit{auxiliary variable}) \--- при этом размер формулы увеличится лишь линейно.
В данной работе подразумевается, что все булевы выражения, кодирующие задаваемые ограничения, подвергаются либо стандартным логическим преобразованиям, либо преобразованиям Цейтина, то есть по итогу представляются в виде КНФ.

На практике для решения задачи SAT используются специализированные программные средства \--- SAT-\emph{решатели}.
Несмотря на то, что задача SAT имеет экспоненциальную оценку сложности (при условии, что $P \neq NP$), современные SAT-решатели способны решать формулы с миллионами переменных за обозримое время.
Для выбора наиболее эффективного SAT-решателя можно руководствоваться результатами соревнования SAT~Comptetition~\cite{sat-competition}: среди текущих лидеров можно выделить MapleCOMSPS~\cite{liang-2016}, Cadical~\cite{cadical}, CryptoMiniSat~\cite{cryptominisat}, Glucose~\cite{glucose} и Plingeling~\cite{lingeling-and-friends}, хотя на практике эффективность решателей может значительно отличаться, в зависимости от класса рассматриваемых задач.
В некоторых случаях хорошие результаты также показывает MiniSat~\cite{minisat}, являющийся минимальной реализацией CDCL-решателя (\textit{Conflict-Driven Clause Learning}~\cite{grasp}) и служащий основой для многих других решателей (например, CryptoMiniSat и Glucose).

% Инкрементальность?

\todo{Clause, CNF, SAT, literal, model, etc.}

\subsection{Алгоритмы решения SAT}
\label{sub:sat-algorithms}

DPLL. CDCL.

\subsection{SAT-решатели}
\label{sub:sat-solvers}

\todo{MiniSat, CryptoMiniSat, Glucose, Kissat, Cadical}

\subsection{Методы сведения задач к SAT}
\label{sub:sat-encodings}

\subsection{Декомпозиционная трудность}
\label{sub:sat-decomposition}

\todo{d-hardness}

\todo{Strong backdoors}

\subsection{Вероятностные лазейки}
\label{sub:sat-backdoors}

\todo{$rho$-backdoors}
