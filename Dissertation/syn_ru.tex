\chapter*{Реферат}
% \pdfbookmark{Реферат}{synopsis}
% \addcontentsline{toc}{chapter}{Реферат}

\pdfbookmark{Общая характеристика работы}{characteristic}
\section*{ОБЩАЯ ХАРАКТЕРИСТИКА РАБОТЫ}

% Актуальность.
\actuality
%
В современном мире существенную роль играет проблема эффективной верификации разнообразных автоматизированных систем на предмет удовлетворения конкретным спецификациям, а также задача синтеза систем под конкретные спецификации.
Под термином \enquote{Автоматизированные системы} понимается широкий класс объектов, объединенных общей вычислительной природой \--- любой такой объект, решая задачу, вычисляет значения некоторой вполне конкретной функции.
Для исследования различных свойств таких объектов, в том числе относящихся к практическим приложениям, имеет смысл использовать абстрактные модели, в рамках которых вычисляемые функции задаются программами.
Широкий класс автоматизированных систем допускает описание на основе концепций, опирающихся на понятие состояния вычисляющей модели, впервые введенное, по-видимому, А.~Тьюрингом в фундаментальной статье~\autocite{turing1937}.
Данное понятие является весьма плодотворным, поскольку имеет массу практических приложений, таких как разработка языков программирования, трансляторов и компиляторов, разработка микроконтроллеров под решение конкретных производственных задач, разработка микропроцессоров общего назначения и многое другое.
Многие такие практические системы могут быть представлены моделями, в которых состояния понимаются и рассматриваются в рамках конечно-автоматной парадигмы.
Такой подход к построению программ подразумевает разбиение программы на более простые модули, которые сами по себе могут рассматриваться как вычислительные единицы и находиться в отдельных состояниях.
Данный подход получил известность как концепция \textit{автоматного программирования}~\autocite{polikarpova2009}.
Для автоматизированной системы, сценарий работы которой представляется в форме автоматной программы, проблемы синтеза и верификации сводятся к аналогичным проблемам для формальных моделей данных систем.
В~настоящей диссертации рассматриваются два класса таких моделей: конечные автоматы и булевы схемы.
Между этими двумя моделями имеется тесная взаимосвязь: конечный автомат задает функцию, которая на вход может принимать данные, вообще говоря, произвольной конечной длины.
Булева схема же работает с данными конкретной длины.
Соответственно, функции, задаваемой конечным автоматом, соответствует счетное число функций, задаваемых булевыми схемами. Для решения конкретных вычислительных задач, которые, как правило, являются комбинаторными, конечно же, приходится рассматривать конечные входные данные и, таким образом, переходить к булевым схемам.
Задачи верификации и синтеза для упомянутых моделей являются вычислительно сложными.
Даже для булевых схем, работающих с входными конечными данными, задачи, связанные с синтезом и верификацией, относятся к NP-трудным и, таким образом, не могут быть решены известными алгоритмами за полиномиальное время.
В~данной ситуации (как и во многих других примерах, касающихся NP-трудных задач) для решения конкретных примеров рассматриваемых проблем используются некоторые комбинаторные задачи с хорошо развитой алгоритмической базой.
Одной из таких является задача булевой выполнимости (Boolean Satisfiability Problem \--- SAT), для решения которой за последние 20~лет разработаны весьма эффективные на практике эвристические алгоритмы, применяемые для решения задач символьной верификации~\autocite{kroening2021}, компьютерной безопасности и криптографии~\autocite{bard2009}, построению расписаний и планированию~\autocite{prestwich2021} и~многим другим прикладным областям.
В~применении к перечисленным задачам программные реализации алгоритмов решения SAT (так называемые SAT-решатели) дают мощные вычислительные инструменты, позволяющие решать частные случаи рассматриваемых задач таких размерностей, перед которыми другие подходы оказываются бессильны.

Таким образом, актуальной является проблема разработки алгоритмов и программных комплексов на основе алгоритмов решения SAT, используемых для решения задач верификации и синтеза формальных моделей конечно-автоматных программ \--- конечных автоматов и булевых схем.
При решении поставленной задачи возникает целый ряд новых проблем, основной из которых является отсутствие априорных оценок времени работы SAT-решателя на трудной формуле, кодирующей рассматриваемую задачу.
Грубо говоря, решатель, получив на вход формулу, может работать час, два, неделю, месяц или даже больше, и нет общего способа определить, сколько времени ему потребуется для завершения работы, притом что формально данный алгоритм является полным и на любой формуле завершает свою работу за конечное время.
Описанный феномен известен как \enquote{\textit{heavy-tailed behavior phenomenon}}~\autocite{gomes2009} (HTB).
В~рамках настоящей диссертации для борьбы с HTB используются специальные декомпозиционные представления булевых формул, кодирующих описания рассматриваемых моделей.
С~использованием разработанных алгоритмов удалось решить ряд экстремально сложных задач, относящихся к верификации и синтезу конкретных примеров моделей автоматных программ.

% Учитывая все сказанное выше, сформулируем основные цели и задачи работы.


% Цель работы.
\aim
%
Целью настоящей диссертации является повышение эффективности (снижение времени работы) полных алгоритмов решения задачи булевой выполнимости (SAT) применительно к синтезу и верификации моделей автоматных программ за счет разработки оригинальных методов и техник декомпозиции булевых формул.


% Задачи работы.
\tasks
%
Для достижения поставленной цели были решены следующие научно-технические задачи:
\begin{enumerate}[beginpenalty=10000]
    \item Разработаны методы кодирования в SAT задач синтеза конечно-автоматных моделей с заданным поведением и свойствами, отличающиеся от существующих добавлением кодирования структуры охранных условий в виде деревьев разбора соответствующих формул.
    \item Разработаны методы кодирования в SAT задачи синтеза модульных конечно-автоматных моделей с заданным поведением и свойствами, отличающиеся от существующих автоматизированным модульным разбиением.
    \item Разработаны методы кодирования в SAT задачи синтеза булевых схем и булевых формул по заданной таблице истинности, отличающиеся от существующих возможностью использования произвольных элементарных гейтов.
    \item Разработаны методы декомпозиции булевых формул, кодирующих задачи синтеза конечно-автоматных моделей и верификации булевых схем, отличающиеся от существующих возможностью построения оценок декомпозиционной трудности.
    \item С применением разработанных методов решены трудные примеры синтеза и верификации моделей автоматных программ \--- конечных автоматов и логических схем.
    \item Разработана программная библиотека \texttt{kotlin-satlib}, обеспечивающая взаимодействие с SAT-решателями через программный интерфейс, контроль за различными этапами построения SAT кодировок, а также предоставляющая возможности манипуляции переменными с произвольными конечными доменами.
    \item Разработан программный комплекс \smallcaps{fbSAT} для синтеза и верификации конечно-автоматных моделей с помощью SAT-решателей.
    \item Проведены масштабные вычислительные эксперименты для подтверждения практической эффективности всех разработанных методов.
\end{enumerate}


% Научная новизна.
\novelty
%
Новыми являются все основные результаты, полученные в диссертации, в том числе:
\begin{enumerate}[beginpenalty=10000]
    \item Методы декомпозиции булевых формул, применяемые к задачам тестирования и верификации моделей автоматных программ с использованием SAT-решателей. Отличие от известных методов заключается в применении специальных конструкций SAT-разбиений, что позволяет получать более точные верхние оценки трудности формул.

    \item Метод синтеза минимальных представлений булевых функций в виде формул и схем, основанный на сведении к задаче выполнимости (SAT). В~отличие от существующих подходов, предлагаемый метод использует инкрементальные SAT-решатели, что позволяет достичь более высокой эффективности по времени и точности решения.

    \item Методы синтеза и верификации монолитных и модульных конечно-автоматных моделей, разработанные на основе сведений к задаче выполнимости (SAT) и использования контрпримеров (Counter\-example\-/Guided Inductive Synthesis \--- CEGIS). Отличие состоит в применении техники явного кодирования структуры охранных условий, что значительно повышает эффективность решения по времени.

    \item Программная библиотека \texttt{kotlin-satlib} для взаимодействия с SAT-решателями и обеспечения контроля над всеми этапами построения SAT-кодировок. В~отличие от существующих библиотек, разработанная библиотека предоставляет широкий выбор различных SAT-решателей и возможность манипулировать переменными с произвольными конечными доменами.

    \item Программный комплекс \smallcaps{fbSAT} для синтеза и верификации конечно-автоматных моделей с помощью SAT-решателей, включающий реализацию всех предложенных методов. Этот комплекс позволяет эффективно решать экстремально трудные задачи синтеза моделей автоматных программ.
\end{enumerate}


% Основные положения, выносимые на защиту.
\defpositions
%
% Название
% Ограничительная часть (совпадение)
% Цель
% Отличительная часть (новизна)
%
\begin{enumerate}[beginpenalty=10000]
    \item Методы декомпозиции булевых формул, применяемые к задачам тестирования и верификации моделей автоматных программ и использующие SAT-решатели, отличающиеся от известных методов тем, что с целью получения более точных верхних оценок трудности формул в предлагаемых методах используются специальные конструкции SAT-разбиений.

    \item Метод синтеза минимальных представлений булевых функций в виде формул и схем, использующий сведение к задаче выполнимости (SAT), отличающийся от существующих подходов тем, что с целью достижения более высокой эффективности (относительно времени и точности решения) предлагаемый метод использует инкрементальные SAT-решатели.

    \item Методы синтеза и верификации монолитных и модульных конечно-автоматных моделей по примерам поведения и формальной спецификации, использующие сведения к задаче выполнимости (SAT) и контрпримеры (Counterexample-Guided Inductive Synthesis \--- CEGIS), отличающиеся от существующих подходов тем, что с целью повышения эффективности (относительно времени решения) применяется техник явного кодирования структуры охранных условий.

    \item Программная библиотека \texttt{kotlin-satlib}\footnote{\url{https://github.com/Lipen/kotlin-satlib}} для взаимодействия с SAT\-/решателями и обеспечения контроля над всеми этапами построения SAT-кодировок, отличающаяся от известных библиотек тем, что с целью расширения области применимости разработанная библиотека предоставляет широкий выбор различных SAT-решателей и возможность манипулировать переменными с произвольными конечными доменами.

    \item Программный комплекс \smallcaps{fbSAT}\footnote{\url{https://github.com/ctlab/fbSAT}} для синтеза и верификации конечно-автоматных моделей с помощью SAT-решателей, включающий в себя реализацию всех предложенных методов.
\end{enumerate}


% Теоретическая и практическая значимость работы.
\influence
%
Теоретическая значимость диссертации заключается в разработке новых методов и алгоритмов для синтеза и верификации моделей автоматных программ.
В~работе предложены инновационные подходы к декомпозиции булевых формул и построению оценок их декомпозиционной трудности, что расширяет существующие теоретические основы в области применения SAT-решателей и предоставляет более точные инструменты для анализа сложных задач.

Практическая значимость работы проявляется в разработке программной библиотеки \texttt{kotlin-satlib} и программного комплекса \smallcaps{fbSAT}, которые позволяют эффективно применять новые методы и алгоритмы к реальным задачам проектирования и верификации программного обеспечения.
Эти инструменты демонстрируют высокую эффективность и могут быть интегрированы в существующие программные системы, что подтверждается лучшими результатами по сравнению с известными подходами и инструментами.


% Методы и инструменты исследования.
\methods
%
Теоретическая часть работы основана на методологии дискретной математики, математической логики и теории вычислительной сложности.
Для синтеза конечно-автоматных моделей применялся программный комплекс \smallcaps{fbSAT}, разработанный в рамках данной диссертации.
Верификация этих моделей осуществлялась с помощью символьного верификатора \smallcaps{NuSMV}\footnote{\url{https://nusmv.fbk.eu}}.
При построении вычислительных задач из области проверки логической эквивалентности схем использовалась программная система Transalg\footnote{\url{https://gitlab.com/transalg/transalg}}.
Для решения экземпляров задачи SAT применялись различные современные SAT-решатели, такие как MiniSAT\footnote{\url{https://github.com/niklasso/minisat}}, Glucose\footnote{\url{https://github.com/audemard/glucose}}, Kissat\footnote{\url{https://github.com/arminbiere/kissat}}, CaDiCaL\footnote{\url{https://github.com/arminbiere/cadical}}.
Взаимодействие с SAT-решателями осуществлялось через программную библиотеку \texttt{kotlin-satlib}, разработанную в рамках данной диссертации.
Вычислительные эксперименты проводились с использованием вычислительного кластера.


% Соответствие специальности.
\relevance
%
Содержание диссертации соответствует паспорту специальности 2.3.5 \enquote{Математическое и программное обеспечение вычислительных систем, комплексов и компьютерных сетей} в перечисленных ниже пунктах 1 (\enquote{Модели, методы и алгоритмы проектирования, анализа, трансформации, верификации и тестирования программ и программных систем}) и 3 (\enquote{Модели, методы, архитектуры, алгоритмы, языки и программные инструменты организации взаимодействия программ и программных систем}):
\begin{itemize}[beginpenalty=10000]
    \item в диссертации представлено семейство методов и алгоритмов, применяемых для трансформации автоматных программ в булевы схемы и формулы с целью вычислительного решения задач синтеза и верификации автоматных программ (пункт~1);
    \item представлены алгоритмы решения задач синтеза, верификации и тестирования моделей автоматных программ при помощи декомпозиционных представлений булевых формул, кодирующих исходные задачи (пункт~1);
    \item разработанная программная библиотека \texttt{kotlin-satlib} обеспечивает взаимодействие между алгоритмами кодирования в SAT задач синтеза и верификации автоматных программ и современными эффективными SAT-решателями (пункт~3).
\end{itemize}


% Достоверность научных достижений.
{\reliability}
%
Достоверность полученных в работе результатов обеспечивается теоретической корректностью предложенных алгоритмов, эффективность которых обоснована масштабными вычислительными экспериментами.


% Апробация работы.
\probation
%
Основные результаты диссертации докладывались на следующих конференциях:
\begin{itemize}[beginpenalty=10000]
    \item VIII Конгресс молодых ученых, Университет ИТМО, Санкт-Петербург, 2019.
    \item Конференция СПИСОК-2019, СПбГУ, Санкт-Петербург, 2019.
    \item IX Конгресс молодых ученых, Университет ИТМО, Санкт-Петербург, 2020.
    \item Конференция ППС 2021, Университет ИТМО, Санкт-Петербург, 2021.
    \item X Конгресс молодых ученых, Университет ИТМО, Санкт-Петербург, 2021.
    \item Воркшоп SAT/SMT Solvers: Theory and Practice, Санкт-Петербург, 2021.
    \item XI Конгресс молодых ученых, Университет ИТМО, Санкт-Петербург, 2022.
    \item Конференция MIPRO 2024, Опатия, Хорватия, 2024.
\end{itemize}
%
Диссертационная работа была выполнена при поддержке грантов и проектов:
\begin{itemize}[beginpenalty=10000]
    \item Грант РФФИ №19-07-01195 А «Разработка методов машинного обучения на основе SAT-решателей для синтеза модульных логических контроллеров киберфизических систем».
    \item Грант №19-37-51066 Научное наставничество «Разработка методов синтеза конечно-автоматных алгоритмов управления для программируемых логических контроллеров в распределенных киберфизических системах».
    \item НИР-ФУНД 77051 «Исследование алгоритма тестирования на основе обучения и улучшения его эффективности», 2020-2021.
    \item НИР-ПРИКЛ 222004 «Алгоритмы решения SAT для логических схем и анализа программ», 2021-2023.
    \item НИР-ПРИКЛ 223099 «Алгоритмы решения SAT для логических схем и анализа программ», 2023-2024.
\end{itemize}


% Публикации
\publications
%
% \theAllMyPapers
% \theScopusPapers
%
Основные результаты по теме диссертации изложены в 10~публикациях.
Из них одна издана в журналах, рекомендованных ВАК,
а четыре \--- в изданиях, индексируемых в базе цитирования Scopus.
% Также имеется 1 свидетельство о государственной регистрации программ для ЭВМ.

% В международных изданиях, индексируемых в базе данных Scopus:
% \begin{refsection}[biblio/own.bib]
% \nocite{*}
% \printbibliography[
%     keyword=scopus,
%     % title={В международных изданиях, индексируемых в базе данных Scopus},
%     % heading=subbibliography,
%     heading=none,
%     resetnumbers=true
% ]
% \end{refsection}

% В международных изданиях, индексируемых в базе данных Web of Science:
% \begin{refsection}[biblio/own.bib]
% \nocite{*}
% \printbibliography[
%     keyword=wos,
%     %title={В международных изданиях, индексируемых в базе данных Web of Science},
%     %heading=subbibliography,
%     heading=none,
%     resetnumbers=true
% ]
% \end{refsection}

% Список всех публикаций автора по теме диссертации:
% \begin{refsection}[biblio/own.bib]
% \nocite{*}
% \printbibliography[
%     keyword=own,
%     %title={Список всех публикаций автора по теме диссертации},
%     %heading=subbibliography,
%     heading=none,
%     resetnumbers=true
% ]
% \end{refsection}


% Личный вклад автора.
%
\contribution
%
Методы синтеза монолитных и модульных конечно-автоматных моделей по примерам поведения и формальной спецификации, основанные на сведении к задаче SAT, разработаны соискателем в соавторстве с Чивилихиным~Д.\,С.
Методы синтеза и верификации модульных конечно-автоматных моделей по примерам поведения и формальной спецификации, основанные на сведении к задаче SAT и использовании контрпримеров, разработаны соискателем в соавторстве с Чивилихиным~Д.\,С. и Суворовым~Д.\,М.
Программный комплекс \smallcaps{fbSAT} разработан лично соискателем.
Реализация всех разработанных методов синтеза и верификации конечно-автоматных моделей в программном комплексе \smallcaps{fbSAT} выполнена лично соискателем.
Метод синтеза булевых формул и схем по заданной таблице истинности, основанный на сведении к задаче SAT, предложен и разработан лично соискателем.
Прототип программной библиотеки \texttt{kotlin-satlib} для взаимодействия с SAT-решателями через унифицированный программный интерфейс разработан в соавторстве с Гречишкиной~Д.\,С.
Дальнейшая разработка и расширение программной библиотеки \texttt{kotlin-satlib}, что включает в себя поддержку дополнительных SAT-решателей (например, Kissat) и разработку модуля для манипуляции переменными с конечными доменами, производилась лично соискателем.
Общая стратегия оценивания трудности формул, кодирующих проверку эквивалентности (задача верификации) булевых схем, разработана соискателем в соавторстве с Семёновым~А.\,А., Кондратьевым~В.\,С., Кочемазовым~С.\,Е. и Тарасовым~Е.\,А.
Описанные конструкции декомпозиций формул, кодирующих эквивалентность булевых формул, предложены лично соискателем.
Теоретическое обоснование корректности предложенных конструкций выполнены соискателем в соавторстве с Семёновым~А.\,А.



\pdfbookmark{Основное содержание работы}{content}
\section*{ОСНОВНОЕ СОДЕРЖАНИЕ РАБОТЫ}

Во \textbf{введении} обоснована актуальность темы диссертации, сформулированы цель и задачи исследования, описаны научная новизна и практическая значимость работы, представлены методы исследования, сформулированы основные положения, выносимые на защиту, а также приведены данные об апробации работы и личном вкладе автора.


В \textbf{Главе 1} представлен обзор предметной области исследования, охватывающий основные понятия, концепции и методы, используемые при синтезе и верификации моделей автоматных программ \--- конечных автоматов и булевых схем.
Подробно проанализированы существующие методы и алгоритмы решения задачи выполнимости (SAT), а также их применение к задачам синтеза и верификации моделей автоматных программ.
Затронуты вопросы декомпозиции булевых формул, а также оценки декомпозиционной трудности конкретных экземпляров задачи SAT.
Совокупно, эта глава закладывает теоретическую основу для последующих исследований и предложенных методов.

В разделе 1.1 обсуждаются конечные автоматы, которые представляют собой математическую модель, используемую для описания дискретных систем с конечным числом состояний.
Приводятся определения детерминированных и недетерминированных конечных автоматов, рассматриваются их основные свойства и примеры использования в различных областях, таких как моделирование поведения программного обеспечения и систем управления.

Раздел 1.2 посвящён булевым схемам, которые являются основой для цифровых систем и вычислений.
Описываются элементы булевых схем, такие как логические гейты (AND, OR, NOT и другие), и способы их комбинирования для реализации сложных логических функций.
Рассматриваются методы кодирования булевых схем в виде булевых формул в конъюктивной нормальной форме (КНФ) для последующего решения задачи SAT с помощью SAT-решателей с целью тестирования и верификации исходных схем.
Также, вводится лемма о том, что применение правила единичного дизъюнкта (Unit Propagation) к КНФ-формуле с подставленными значениями входных переменных схемы эквивалентно процессу вычисления (интерпретации) всей схемы, в частности, выходных значений.

В разделе 1.3 рассматривается международный стандарт IEC 61499, который определяет архитектуру для разработки распределённых систем управления.
Описываются основные компоненты стандарта, такие как функциональные блоки, и их использование для моделирования и реализации промышленных систем автоматизации.

Раздел 1.4 посвящён описанию формальной модели базового функционального блока, которая используется в стандарте IEC 61499.
Описываются структура и поведение базового функционального блока, способы его конфигурации и взаимодействия с другими блоками.
Рассматриваются примеры использования базовых функциональных блоков для построения сложных систем управления.
Представленная абстрактная модель впоследствии используется в качестве основы при разработке методов синтеза и верификации конечно-автоматных моделей.

В разделе 1.5 обсуждаются сценарии выполнения, которые представляют собой последовательности действий или событий в системе. Описывается, как сценарии выполнения могут использоваться для спецификации требований к системе и её верификации. Рассматриваются различные методы описания сценариев выполнения и их использование в процессе синтеза и тестирования моделей.

Раздел 1.6 посвящён линейной темпоральной логике (LTL), которая используется для спецификации и верификации временных свойств систем. Описываются основные операторы LTL и способы их использования для выражения различных временных аспектов поведения систем. Приводятся примеры спецификаций на основе LTL и методы их проверки.

В разделе 1.7 рассматриваются методы формальной верификации, основанные на проверке моделей (Model Checking).
Описываются основные этапы процесса проверки моделей и инструменты, используемые для этой цели.
Подробно рассматриваются и сравниваются два основных подхода к проверке моделей: символьный (symbolic) и ограниченный (bounded).

В разделе 1.8 рассматриваются различные методы синтеза конечно-автоматных моделей, которые используются для автоматического построения моделей на основе заданных требований и поведения. Описываются как традиционные, так и современные подходы к синтезу, а также их сравнительный анализ.

Раздел 1.9 посвящён задаче проверки эквивалентности булевых схем, которая заключается в проверке, производят ли две булевы схемы одинаковые выходные значения для всех возможных входных данных. Описываются методы и алгоритмы решения этой задачи, а также примеры их применения в различных областях.

В разделе 1.10 обсуждается задача генерации тестовых шаблонов, используемых для верификации булевых схем. Описываются методы генерации тестов, которые позволяют обнаруживать ошибки в схемах и проверять их корректность. Приводятся примеры применения тестовых шаблонов в практике проектирования цифровых систем.

Раздел 1.11 посвящён задаче булевой выполнимости (SAT), которая является фундаментальной проблемой в теории вычислительной сложности и играет ключевую роль в синтезе и верификации логических схем. Описываются формулировка задачи SAT, её значение и примеры применения в различных областях.

В разделе 1.12 рассматриваются основные алгоритмы решения задачи булевой выполнимости (SAT).
Описываются как традиционные, так и современные методы, включая полные и неполные алгоритмы.
Рассматриваются как полные алгоритмы решения SAT, основанные на алгоритме DPLL, которые гарантируют нахождение решения или доказательство его отсутствия, так и неполные, которые не гарантируют нахождение решения, но могут быть эффективны на практике для больших и сложных задач.
Отдельно рассматривается концепция CDCL (Conflict-Driven Clause Learning), которая является усовершенствованием алгоритма DPLL и лежит в основе современных SAT-решателей.

В разделе 1.13 обсуждаются ограничения на кардинальность, которые часто встречаются в задачах SAT.
Ограничения на кардинальность представляют собой условия, которые ограничивают количество переменных, принимающих значение истинности.
Описывается один из методов, используемых для обработки таких ограничений \--- кодирование в КНФ с помощью метода \textit{totalizer}, который заключается в кодировании унарной записи числа, задающего сумму переменных, в виде булевой формулы.
На итоговое число в унарном представлении накладываются ограничения в виде единичных дизъюнктов, которые обеспечивают выполнение условий на кардинальность, например, вхождение числа истинных переменных в заданный диапазон.

Раздел 1.14 посвящён методам разбиения задачи SAT на подзадачи.
Описываются различные подходы к разбиению и их преимущества.
Важным аспектом при решении задач SAT является декомпозиционная трудность булевых формул.
В~этом контексте рассматриваются методы оценки трудности и их применение на практике.
Также обсуждаются основы теории вероятности, необходимые для понимания вероятностных методов оценки трудности задач SAT, включая основные понятия и методы, используемые в этой области.
Дополнительно рассматривается вероятностный подход к оцениванию трудности булевых формул, который позволяет более точно прогнозировать сложность их решения.
Описываются методы и алгоритмы, использующие вероятностные модели, и примеры их применения.



В \textbf{главе 2} рассматриваются методы синтеза моделей автоматных программ на основе сведения задач к задаче булевой выполнимости (SAT).
Основное внимание уделено разработке и применению методов кодирования задач синтеза в формате SAT, что позволяет значительно улучшить эффективность и точность синтеза конечно-автоматных моделей.
Глава начинается с описания программной библиотеки \texttt{kotlin-satlib}, разработанной для взаимодействия с SAT-решателями и упрощающей процесс построения SAT-кодировок.
Далее рассматриваются методы синтеза булевых формул и методы синтеза конечно-автоматных моделей монолитных логических контроллеров по примерам поведения.
Особое внимание уделено алгоритмам синтеза минимальных моделей, индуктивному синтезу на основе контрпримеров, а также разработке программного средства fbSAT для синтеза и верификации конечно-автоматных моделей.
В~главе~2 также приведены результаты экспериментальных исследований, демонстрирующих практическую значимость и эффективность предложенных методов.

В разделе 2.1 описывается программная библиотека kotlin-satlib, разработанная для взаимодействия с SAT-решателями. Эта библиотека предназначена для упрощения процесса синтеза и верификации моделей автоматных программ за счёт предоставления удобного интерфейса для работы с SAT-решателями. Библиотека включает несколько модулей, каждый из которых имеет специфические функции. Первый модуль обеспечивает взаимодействие с SAT-решателями посредством технологии JNI, что позволяет использовать различные SAT-решатели без существенных изменений в коде. Второй модуль занимается записью ограничений с использованием преобразований Цейтина, что оптимизирует процесс конвертации логических выражений в формат SAT. Третий модуль позволяет манипулировать переменными с ограниченным доменом, что расширяет возможности моделирования. Наконец, четвёртый модуль поддерживает работу с массивами SAT переменных, предоставляя средства для эффективного управления большими логическими структурами.

Раздел 2.2 посвящён методам синтеза булевых формул. В нём рассматриваются различные подходы к построению булевых формул, которые затем могут быть использованы для синтеза моделей автоматных программ. Этот раздел также включает результаты экспериментального исследования, которое демонстрирует эффективность предложенных методов синтеза. Были проведены многочисленные эксперименты для оценки производительности различных подходов к синтезу булевых формул, результаты которых подтверждают высокую эффективность разработанных методов.

В разделе 2.3 описывается метод синтеза конечно-автоматных моделей монолитных логических контроллеров по примерам поведения. Этот метод включает несколько ключевых этапов: сначала происходит кодирование структуры автомата, затем вводятся BFS-предикаты нарушения симметрии для состояний автомата, что помогает избежать избыточных вычислений. Далее, кодируется отображение позитивного дерева сценариев и устанавливаются ограничения на количество переходов между состояниями. В разделе также подробно рассматриваются различные алгоритмы, такие как Basic, Extended и Complete, каждый из которых предлагает свои подходы к синтезу моделей с учётом различных ограничений и требований. Этот метод позволяет значительно сократить время синтеза и повысить точность моделирования.

Раздел 2.4 фокусируется на методах синтеза минимальных монолитных моделей. Здесь рассматриваются алгоритмы Basic-min, Extended-min и Complete-min, которые оптимизируют процесс синтеза с целью получения минимальных по размеру и сложности моделей. В этом разделе также описывается алгоритм Extended-min-UB, который использует дополнительные ограничения для улучшения производительности.

В разделе 2.5 представлен индуктивный синтез, основанный на контрпримерах (CEGIS). Этот метод включает алгоритмы CEGIS и CEGIS-min, которые позволяют улучшить процесс синтеза за счёт использования контрпримеров, выявленных в ходе верификации моделей. Эти алгоритмы помогают итеративно улучшать модели, корректируя их на основе найденных ошибок.

В разделе 2.6 описывается программное средство fbSAT, предназначенное для синтеза и верификации конечно-автоматных моделей с использованием SAT-решателей. Это средство интегрирует разработанные методы и алгоритмы, предоставляя удобный инструмент для практического применения. fbSAT обеспечивает полный цикл разработки и верификации моделей, от начального синтеза до окончательной проверки корректности.

В разделе 2.7 приводится экспериментальное исследование на примере Pick-and-Place манипулятора. В рамках этого исследования синтезировались минимальные конечно-автоматные модели по примерам поведения, а также по примерам поведения и LTL-спецификации. Эти эксперименты демонстрируют практическую значимость разработанных методов и их применимость к реальным задачам.

В разделе 2.8 рассматривается экспериментальное исследование, проведённое в рамках соревнований SYNTCOMP. Результаты этих экспериментов подтверждают высокую эффективность и конкурентоспособность предложенных методов синтеза и верификации.


\textbf{Глава 3} посвящена методам синтеза модульных конечно-автоматных моделей. Рассматриваются подходы к параллельной, последовательной и произвольной композиции модулей по примерам поведения. Описаны методы сведения задач к SAT и приведены алгоритмы, оптимизирующие процесс синтеза для модульных моделей. Также обсуждаются методы синтеза минимальных модульных моделей и переход от монолитного к распределённому синтезу. Глава завершается результатами экспериментальных исследований, которые подтверждают эффективность предложенных методов синтеза модульных моделей и их применимость к реальным задачам.

В разделе 3.1 описан метод синтеза модульных конечно-автоматных моделей с параллельной композицией модулей по примерам поведения. Этот метод включает этапы сведения задачи к SAT, где сначала определяются переменные, затем вводятся ограничения, необходимые для корректного функционирования моделей. В разделе также подробно описываются алгоритмы Parallel-Basic и Parallel-Extended, которые используют эти ограничения для создания эффективных модульных моделей. Эти алгоритмы помогают снизить сложность синтеза за счёт параллельной обработки различных компонентов модели, что ускоряет процесс и улучшает масштабируемость.

В разделе 3.2 представлен метод синтеза конечно-автоматной модели модульного логического контроллера с последовательной композицией модулей по примерам поведения. Здесь также используется сведение к SAT, где переменные и ограничения определяются таким образом, чтобы обеспечить последовательное выполнение различных модулей. Описаны алгоритмы Consecutive-Basic и Consecutive-Extended, которые оптимизируют процесс синтеза для последовательных композиций, улучшая тем самым точность и эффективность моделей.

В разделе 3.3 рассматривается метод синтеза модульных конечно-автоматных моделей с произвольной композицией модулей по примерам поведения. Этот метод позволяет создавать более гибкие модели, которые могут включать произвольное количество и комбинации модулей. В разделе подробно описаны этапы сведения задачи к SAT, включая определение переменных и введение ограничений, а также алгоритмы Arbitrary-Basic и Arbitrary-Extended, которые оптимизируют процесс синтеза для таких моделей.

Раздел 3.4 посвящён методам синтеза минимальных модульных моделей. Здесь рассматриваются различные подходы к минимизации размера и сложности модульных моделей, что позволяет создавать более эффективные и компактные решения.

В разделе 3.5 описан переход от монолитного к распределённому синтезу. Этот метод включает сведение к SAT для распределённого синтеза по примерам поведения, что позволяет разбивать задачу на более мелкие и управляемые компоненты. Также рассматриваются составное негативное дерево сценариев и его отображение, что помогает улучшить процесс синтеза распределённых контроллеров. В разделе приводятся методы нахождения минимального распределённого контроллера, что позволяет оптимизировать распределённые системы.

В разделе 3.6 приводятся результаты экспериментального исследования модульного синтеза. Эти исследования демонстрируют эффективность предложенных методов и подтверждают их применимость к реальным задачам. Эксперименты показывают, что модульный синтез позволяет значительно улучшить производительность и точность моделей.


В \textbf{главе 4} рассматриваются методы оценивания декомпозиционной трудности булевых формул, которые применяются к задачам тестирования и верификации моделей автоматных программ с использованием SAT-решателей.
Описаны методы декомпозиции булевых формул, позволяющие строить более точные верхние оценки трудности формул.
Конкретно, предложены оригинальные конструкции SAT-разбиений, которые улучшают прогнозируемость времени работы SAT-решателей.
Глава включает в себя теоретические аспекты, алгоритмы и результаты вычислительных экспериментов, демонстрирующих эффективность предложенных подходов в различных сценариях.

% 4.1. Трудность относительно разбиения и вероятностный алгоритм её оценки
В разделе 4.1 рассматривается концепция трудности булевых формул относительно разбиения, которая является ключевым фактором при решении задач SAT.
Обсуждается, как структура и сложность формулы влияют на эффективность её решения, и вводится понятие \textit{декомпозиционной трудности}.

Вероятностный алгоритм оценки трудности представляет собой метод, который использует вероятностные модели для прогнозирования трудности SAT-разбиения булевых формул. Описываются основные этапы алгоритма: построение вероятностной модели, определение времени решения подзадач, оценка на основе статистических данных.

% 4.2. Два новых метода разбиения SAT для CircuitSAT
В разделе 4.2 представляются два новых метода разбиения SAT, разработанные специально для задачи CircuitSAT, которая является частным случаем задачи булевой выполнимости, применимым к логическим схемам. Эти методы направлены на повышение эффективности решения за счёт более оптимального разбиения исходной формулы на подформулы.

% TODO: описание конструкций SAT-разбиений

% 4.3. Вычислительные эксперименты
Раздел 4.3 посвящён описанию вычислительных экспериментов, которые были проведены для подтверждения практической эффективности предложенных методов декомпозиции булевых формул.
В этом разделе детально рассматриваются тестовые данные, методология проведения экспериментов и анализ полученных результатов.

% TODO: тестовые данные

Эксперименты по оценке декомпозиционной трудности булевых формул проводились с использованием предложенного вероятностного алгоритма.
Методология проведения этих экспериментов включала в себя сбор и обработку данных о сложност решения задач SAT при различных вариантах разбиения формул.
Полученные результаты сравнены с теоретическими оценками, что позволяет проверить точность и надёжность алгоритма.
Статистический анализ данных показал, что вероятностный алгоритм успешно предсказывает декомпозиционную трудность формул, что подтверждается корреляцией между теоретическими оценками и реальными измерениями трудности.

Кроме того, были проведены эксперименты по поиску прообразов криптографической хеш-функции MD4 с использованием разработанных методов разбиения. В рамках этих экспериментов задача поиска прообразов MD4 была закодирована в виде задачи SAT, и применены новые методы разбиения для её решения. Результаты показали, что предложенные методы существенно повышают эффективность решения таких задач. Анализ временных характеристик и сравнение с другими подходами продемонстрировали, что новые методы разбиения обеспечивают лучшее качество разбиения и более высокую производительность.

В общем, вычислительные эксперименты, описанные в разделе 4.3, демонстрируют высокую практическую значимость предложенных методов декомпозиции булевых формул. Эти методы не только теоретически обоснованы, но и подтверждены на практике, что показывает их применимость для решения сложных задач синтеза и верификации логических схем и конечно-автоматных моделей.

% % 4.3. Вычислительные эксперименты
% Раздел 4.3 посвящён описанию вычислительных экспериментов, которые были проведены для подтверждения практической эффективности предложенных методов декомпозиции булевых формул. Описываются используемые тестовые данные, методология проведения экспериментов и анализ полученных результатов.

% % 4.3.1. Тестовые данные
% В подразделе 4.3.1 рассматриваются тестовые данные, используемые для экспериментов. Приводятся источники данных, их характеристика и обоснование выбора. Тестовые данные включают различные типы булевых формул и схем, типичные для задач синтеза и верификации, что позволяет проверить универсальность и адаптивность предложенных методов.

% % 4.3.2. Эксперименты по оценке декомпозиционной трудности
% Подраздел 4.3.2 описывает эксперименты, направленные на оценку декомпозиционной трудности булевых формул с использованием предложенного вероятностного алгоритма. Приводятся результаты сравнительного анализа между теоретическими оценками и реальными измерениями трудности при решении задач SAT. Описываются методы сбора и обработки данных, а также статистический анализ результатов, подтверждающий точность и надёжность предложенного алгоритма.

% % 4.3.3. Эксперименты по поиску прообразов MD4
% В подразделе 4.3.3 рассматриваются эксперименты по поиску прообразов криптографической хеш-функции MD4 с использованием разработанных методов разбиения. Описывается процесс кодирования задачи поиска прообразов MD4 в виде задачи SAT и применение новых методов разбиения для её решения. Приводятся результаты экспериментов, демонстрирующие эффективность предложенных методов в контексте криптоанализа. Анализируются временные характеристики и сравнение с другими подходами, подтверждая преимущество новых методов в терминах производительности и качества разбиения.

Вся глава 4 подробно иллюстрирует, как предложенные методы и алгоритмы могут применяться для решения сложных задач синтеза и верификации, предоставляя как теоретическое обоснование, так и практическую реализацию, поддержанную экспериментальными данными.


В заключении сформулированы основные результаты научной работы.


\pdfbookmark{Основные результаты диссертации и выводы}{conclusion}
\section*{ОСНОВНЫЕ РЕЗУЛЬТАТЫ ДИССЕРТАЦИИ И ВЫВОДЫ}

В данной диссертации были получены следующие основные результаты и выводы:
\begin{itemize}[beginpenalty=10000]
    \item Разработаны алгоритмы декомпозиции булевых формул, которые могут применяться к задачам верификации моделей автоматных программ. Эти алгоритмы позволяют строить оценки декомпозиционной трудности рассматриваемых формул.

    \item Предложены новые алгоритмы кодирования в SAT задачи синтеза булевых схем и булевых формул по заданной таблице истинности. Эти алгоритмы отличаются от существующих возможностью использования произвольных элементарных гейтов, что расширяет их применение.

    \item Разработаны алгоритмы кодирования в SAT монолитных и модульных конечно-автоматных моделей для решения задач синтеза и верификации. Эти алгоритмы включают явное кодирование структуры охранных условий в виде деревьев разбора соответствующих формул, что позволяет их минимизировать, что в свою очередь повышает человеко-читаемость полученных выражений.

    \item Созданы программные библиотеки для взаимодействия с различными SAT-решателями через специально разработанный программный интерфейс. Эти библиотеки отличаются возможностью контроля всех этапов построения SAT-кодировок и манипулирования переменными с произвольными конечными доменами.

    \item Проведены масштабные вычислительные эксперименты, подтвердившие практическую эффективность всех разработанных методов. Эти эксперименты продемонстрировали значительное улучшение в решении трудных примеров синтеза и верификации моделей автоматных программ \--- конечных автоматов и булевых схем.
\end{itemize}

Таким образом, разработанные методы и алгоритмы значительно повышают эффективность (снижают время работы) полных алгоритмов решения задачи булевой выполнимости (SAT) и вносят важный вклад в области синтеза и верификации моделей автоматных программ.


\pdfbookmark{Публикации по теме диссертации}{publications}
\section*{ПУБЛИКАЦИИ ПО ТЕМЕ ДИССЕРТАЦИИ}

\begin{enumerate}[left=0pt]
    \item Chivilikhin~D., Patil~S., \textbf{Chukharev~K.}, Cordonnier~A., Vyatkin~V. Automatic State Machine Reconstruction from Legacy PLC Using Data Collection and SAT Solver // IEEE Transactions on Industrial Informatics. 2020. Vol.~16, issue~12. P.~7821–7831. (\textbf{Scopus, Q1})
    \item \textbf{Chukharev~K.}, Suvorov~D., Chivilikhin~D. SAT-based Counterexample-Guided Inductive Synthesis of Distributed Controllers // IEEE Access. 2020. Vol.~8. P.~207485–207498. (\textbf{Scopus, Q1})
    \item \textbf{Chukharev~K.}, Chivilikhin~D. fbSAT: Automatic Inference of Minimal Finite-State Models of Function Blocks Using SAT Solver // IEEE Access. 2022. Vol.~10. P.~131592–131610. (\textbf{Scopus, Q1})
    \item Andreev~A., \textbf{Chukharev~K.}, Kochemazov~S., Semenov~A. Solving Influence Maximization Problem under Deterministic Linear Threshold Model using Metaheuristic Optimization // MIPRO, 2024. (\textbf{Scopus})
    \item \textbf{Чухарев К.} Применение инкрементальных SAT-решателей для решения NP-трудных задач на примере задачи синтеза минимальных булевых формул // Научно-технический вестник информационных технологий, механики и оптики. 2020. Т.~20, 6(130). С.~841—847. (\textbf{ВАК})
    \item \textbf{Чухарев К.}, Чивилихин Д. Построение минимальных конечно-автоматных моделей функциональных блоков по обучающим примерам // Сборник тезисов докладов конгресса молодых ученых. Электронное издание. СПб: Университет ИТМО, 2018.
    \item \textbf{Чухарев К.} Построение конечно-автоматных моделей функциональных блоков по примерам поведения и темпоральным свойствам // Сборник тезисов докладов конгресса молодых ученых. Электронное издание. СПб: Университет ИТМО, 2019.
    \item \textbf{Чухарев К.} Автоматический синтез минимальных конечно-автоматных моделей функциональных блоков по примерам поведения и темпоральным свойствам // Материалы 8-й всероссийской научной конференции по проблемам информатики СПИСОК-2019. СПб.: ВВМ, 2019.
    \item \textbf{Чухарев К.} Синтез конечно-автоматных моделей модульных логических контроллеров по примерам поведения с помощью SAT-решателей // Сборник тезисов докладов конгресса молодых ученых. Электронное издание. СПб: Университет ИТМО, 2020.
    \item Гречишкина Д., \textbf{Чухарев К.} Программный интерфейс для SAT-решателей на основе технологии JNI // Сборник тезисов докладов конгресса молодых ученых. Электронное издание. СПб: Университет ИТМО, 2020.
\end{enumerate}

% \ifdefmacro{\microtypesetup}{\microtypesetup{protrusion=false}}{}
% \begin{refsection}[biblio/own.bib]
% \nocite{*}
% \printbibliography[
%     keyword=own,
%     %title={Список всех публикаций автора по теме диссертации},
%     %heading=subbibliography,
%     heading=none,
%     resetnumbers=true
% ]
% \end{refsection}
% \ifdefmacro{\microtypesetup}{\microtypesetup{protrusion=true}}{}
