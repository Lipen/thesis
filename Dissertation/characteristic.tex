{\actuality}
В современном мире существует множество различных приложений, в которых возникают задачи большой размерности.
Особенно это актуально в контексте задач синтеза и верификации дискретных управляющих (ДУ) систем, таких как конечные автоматы и цифровые схемы, которые используются в различных областях, включая робототехнику, электронику, информационную безопасность.
Основыми проблемами в области дискретных систем управления являются их синтез и верификация.
Синтез ДУ-модели заключается в построении модели системы, которая обладает заданными свойствами.
Верификация ДУ-модели заключается в проверке того, что модель удовлетворяет заданной спецификации.
Важно отметить, что задачи синтеза и верификации ДУ-моделей являются NP-полными.
Это означает, что на сегодняшний день не существует эффективных алгоритмов, способных решать эти задачи за разумное время для всех возможных входных данных.
Это приводит к тому, что существующие методы не всегда применимы для решения задач большой размерности.
Таким образом, развитие методов и алгоритмов, способных решать подобные задачи более эффективно, остается актуальной и важной задачей в области дискретной математики и компьютерных наук.

Распространённым подходом к \emph{автоматическому} синтезу и верификации является \emph{сведение} к классическим NP-полным задачам, таким как задача выполнимости булевой формулы (Boolean satisfiability problem, SAT), задача максимальной выполнимости (MaxSAT) и задача выполнимости в теориях (Satisfiability Modulo Theory, SMT), с последующим применением так называемых \emph{решателей}, реализующих современные алгоритмы решения этих задач.
Данные задачи являются \enquote{универсальными}, в том смысле что они могут быть использованы для решения широкого класса задач, исключая тем самым необходимость разработки специализированных алгоритмов для каждой конкретной задачи.

Одной из центральных проблем в области синтеза и верификации ДУ-моделей является отсутствие априорных оценок времени работы алгоритмов решения задачи SAT.
В рамках данной диссертационной работы предлагаются и развиваются методы и алгоритмы, которые позволяют строить такие оценки.
Для этого используются специальные декомпозиционные представления булевых формул.
Идеи построения декомпозиций уже выдвигались ранее, но они обладают меньшей точностью.
Методы декомпозиции, предложенные в данной работе, учитывают особенности исходной задачи, которая решается с помощью сведения к SAT, например, особенности задачи синтеза конечно-автоматных моделей или задачи проверки эквивалентности логических схем.

Резюмируя, предложенные в диссертации методы и алгоритмы позволяют повысить эффективность комбинаторных алгоритмов в применении к задачам синтеза и верификации ДУ-моделей.
Эти методы могут быть использованы для решения различных задач в области автоматизированного проектирования и управления дискретными системами.


% Цель работы.
{\aim}
Целью данной работы является повышение эффективности работы полных алгоритмов решения задачи булевой выполнимости (SAT) в применении к задачам синтеза и верификации ДУ-моделей за счет оригинальных методов и техник декомпозиции булевых формул.


% Задачи работы.
{\tasks}
Для достижения поставленной цели были решены следующие научно-технические задачи:
\begin{enumerate}[beginpenalty=10000]
    \item Разработаны оригинальные алгоритмы кодирования в SAT задач синтеза конечно-автоматных моделей с заданным поведением и свойствами, отличающиеся от существующих добавлением кодирования структуры охранных условий в виде деревье разбора соответствующих формул.
    \item Разработаны оригинальные алгоритмы кодирования в SAT задач синтеза модульных конечно-автоматных моделей с заданным поведением и свойствами, отличающиеся от существующих автоматизированным модульным разбиением.
    \item Разработаны оригинальные методы оценивания декомпозиционной трудности булевых формул, кодирующих задачи синтеза конечно-автоматных моделей и верификации булевых схем, отличающиеся от существующих учётом особенностей исходной задачи, а также низкой дисперсией времени решения подзадач.
    \item \todo{Разработаны новые алгоритмы решения трудных вариантов SAT, использующие понятие вероятностных лазеек (backdoors).}
    \item С применением разработанных алгоритмов решены трудные примеры задач синтеза ДУ-моделей (как конечно-автоматных моделей, так и логических схем).
    \item \todo{Разработан новый SAT решатель, использующий вероятностные лазейки для вывода новой информации при работе с трудными булевыми формулами.}
    \item \todo{Разработана библиотека алгоритмов ... + вычислительные эксперименты.}
\end{enumerate}


% Научная новизна.
{\novelty}
Новыми являются все основные результаты, полученные в диссертации, в том числе:
\begin{enumerate}[beginpenalty=10000]
    \item Новые алгоритмы синтеза конечно-автоматных и модульных конечно-автоматных моделей, основанные на сведении к проблеме булевой выполнимости (SAT).

    \item Новые методы оценивания декомпозиционной трудности булевых формул, кодирующих задачи синтеза ДУ-моделей.

    \item Оригинальные алгоритмы решения трудных инстансов задачи SAT, использующие объединение нескольких вероятностных лазеек.

    \item Решение экстремально трудных задач синтеза ДУ-моделей при помощи разработанных алгоритмов.
\end{enumerate}


% Основные положения, выносимые на защиту.
\defpositions
\begin{enumerate}[beginpenalty=10000]
    \item Оригинальные алгоритмы кодирования в SAT задач синтеза конечно-автоматных моделей, отличающиеся от существующих добавлением кодирования структуры охранных условий в виде деревье разбора соответствующих формул.

    \item Оригинальные алгоритмы оценивания декомпозиционной трудности булевых формул применительно к задачам верификации логических схем, отличающиеся от существующих учётом особенностей исходной задачи.

    \item Новые алгоритмы решения трудных вариантов SAT, использующие понятие вероятностных лазеек ($\rho$-backdoors).

    % \item Алгоритмы решения трудных вариантов SAT на основе объединения нескольких вероятностных лазеек;

    % \item Семейство алгоритмов, вошедших в состав нового SAT решателя, базирующегося на концепции вероятностных лазеек.
\end{enumerate}


% Теоретическая и практическая значимость работы.
\influence
Теоретическая значимость диссертации заключается в разработанных в ней концепциях и алгоритмах решения задач синтеза ДУ-моделей и методах построения оценок трудности таких задач.
Практическая значимость диссертации состоит в том, что основные разработанные в ней алгоритмы применимы к индустриальным задачам синтеза и верификации ДУ-моделей, а также в том, что на целом ряде конкретных примеров практическая реализация и апробация разработанных  алгоритмов демонстрируют лучшую эффективность в сравнении с известными подходами.


% Методы и инструменты исследования.
\methods
\todo{переписать}
Теоретическая часть работы использует методологию дискретной математики и математической логики, теории вычислительной сложности,а также теорию эволюционных вычислений.
\todo{Для синтеза конечно-автоматных моделей был использован программный комплекс fbSAT, разработанный в рамках данной диссертации.} \todo{ссылка}
При построении вычислительных задач из области проверки логической эквивалентности схем использовалась программная система Transalg \todo{ссылка}.
Для решения конкретных инстансов задачи SAT использовались различные современные SAT-решатели, находящиеся в открытом доступе, такие как MiniSAT, Glucose, Kissat, Cadical \todo{ссылки}.
В вычислительных экспериментах задействовался вычислительный кластер.


% Соответствие специальности.
\relevance
Содержание научно-квалификационной работы охватывает такие направления как: синтез и верификация управляющих систем дискретной природы; разработку проблемно-ориентированных комбинаторных алгоритмов, применимых к автоматическому проектированию и верификации ДУ-моделей; разработку алгоритмов декомпозиции сложных эксземпляров комбинаторных задач; разрработку программных средств для эффективного взаимодействия и SAT-решателями; разработку специализированных SAT-решателей, учитывающих особенности решаемых задач.
Таким образом, можно утверждать, что работа соответствует паспорту специальности 2.3.5 (05.13.11) в пунктах 1 и 3.


% Достоверность научных достижений.
\reliability
\todo{}


% Апробация работы.
\probation
Основные результаты диссертации докладывались на следующих конференциях:
\begin{itemize}[beginpenalty=10000]
    \item VIII Конгресс молодых ученых, Университет ИТМО, Санкт-Петербург, 2019.
    \item Конференция СПИСОК-2019, СПбГУ, Санкт-Петербург, 2019.
    \item IX Конгресс молодых ученых, Университет ИТМО, Санкт-Петербург, 2020.
    \item Конференция ППС 2021, Университет ИТМО, Санкт-Петербург, 2021.
    \item X Конгресс молодых ученых, Университет ИТМО, Санкт-Петербург, 2021.
    \item Воркшоп SAT/SMT Solvers: Theory and Practice, Санкт-Петербург, 2021.
    \item XI Конгресс молодых ученых, Университет ИТМО, Санкт-Петербург, 2022.
\end{itemize}

Диссертационная работа была выполнена при поддержке грантов и проектов:
\begin{itemize}[beginpenalty=10000]
    \item Грант РФФИ №19-07-01195 А «Разработка методов машинного
    обучения на основе SAT-решателей для синтеза модульных логических контроллеров киберфизических систем».
    \item Грант №19-37-51066 Научное наставничество «Разработка методов синтеза конечно-автоматных алгоритмов управления для программируемых логических контроллеров в распределенных киберфизических системах».
    \item НИР-ФУНД 77051 «Исследование алгоритма тестирования на основе обучения и улучшения его эффективности», 2020-2021.
    \item НИР-ПРИКЛ 222004 «Алгоритмы решения SAT для логических схем и анализа программ», 2021-2023.
    \item НИР-ПРИКЛ 223099 «Алгоритмы решения SAT для логических схем и анализа программ», 2023-2024.
\end{itemize}



% Публикации
\publications
Основные результаты по теме диссертации изложены в \theAllMyPapers~публикациях.
Из них
%4 изданы в журналах, рекомендованных ВАК,
\theScopusPapers~опубликовано в изданиях, индексируемых в базе цитирования Scopus.
%Также имеется 1 свидетельство о государственной регистрации программ для ЭВМ.

% В международных изданиях, индексируемых в базе данных Scopus:
% \begin{refsection}[biblio/own.bib]
% \nocite{*}
% \printbibliography[
%     keyword=scopus,
%     % title={В международных изданиях, индексируемых в базе данных Scopus},
%     % heading=subbibliography,
%     heading=none,
%     resetnumbers=true
% ]
% \end{refsection}

% В международных изданиях, индексируемых в базе данных Web of Science:
% \begin{refsection}[biblio/own.bib]
% \nocite{*}
% \printbibliography[
%     keyword=wos,
%     %title={В международных изданиях, индексируемых в базе данных Web of Science},
%     %heading=subbibliography,
%     heading=none,
%     resetnumbers=true
% ]
% \end{refsection}

% Список всех публикаций автора по теме диссертации:
% \begin{refsection}[biblio/own.bib]
% \nocite{*}
% \printbibliography[
%     keyword=own,
%     %title={Список всех публикаций автора по теме диссертации},
%     %heading=subbibliography,
%     heading=none,
%     resetnumbers=true
% ]
% \end{refsection}
