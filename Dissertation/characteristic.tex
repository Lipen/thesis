% Актуальность.
\actuality
%
В современном мире существенную роль играет проблема эффективной верификации разнообразных автоматизированных систем на предмет удовлетворения конкретным спецификациям, а также задача синтеза систем под конкретные спецификации.
Под термином \enquote{Автоматизированные системы} понимается широкий класс объектов, объединенных общей вычислительной природой \--- любой такой объект, решая задачу, вычисляет значения некоторой вполне конкретной функции.
Для исследования различных свойств таких объектов, в том числе относящихся к практическим приложениям, имеет смысл использовать абстрактные модели, в рамках которых вычисляемые функции задаются программами.
Широкий класс автоматизированных систем допускает описание на основе концепций, опирающихся на понятие состояния вычисляющей модели, впервые введенное, по-видимому, А.~Тьюрингом в фундаментальной статье~\autocite{turing1937}.
Данное понятие является весьма плодотворным, поскольку имеет массу практических приложений, таких как разработка языков программирования, трансляторов и компиляторов, разработка микроконтроллеров под решение конкретных производственных задач, разработка микропроцессоров общего назначения и многое другое.
Многие такие практические системы могут быть представлены моделями, в которых состояния понимаются и рассматриваются в рамках конечно-автоматной парадигмы.
Такой подход к построению программ подразумевает разбиение программы на более простые модули, которые сами по себе могут рассматриваться как вычислительные единицы и находиться в отдельных состояниях. Данный подход получил известность как концепция автоматного программирования~\autocite{polikarpova2009}.
Для автоматизированной системы, сценарий работы которой представляется в форме автоматной программы, проблемы синтеза и верификации сводятся к аналогичным проблемам для формальных моделей данных систем. В настоящей диссертации рассматриваются два класса таких моделей: конечные автоматы и булевы схемы. Между этими двумя моделями имеется тесная взаимосвязь: конечный автомат задает функцию, которая на вход может принимать данные, вообще говоря, произвольной конечной длины. Булева схема же работает с данными конкретной длины. Соответственно, функции, задаваемой конечным автоматом, соответствует счетное число функций, задаваемых булевыми схемами. Для решения конкретных вычислительных задач, которые, как правило, являются комбинаторными, конечно же, приходится рассматривать конечные входные данные и, таким образом, переходить к булевым схемам. Задачи верификации и синтеза для упомянутых моделей являются вычислительно сложными. Даже для булевых схем, работающих с входными конечными данными, задачи, связанные с синтезом и верификацией, относятся к NP-трудным и, таким образом, не могут быть решены известными алгоритмами за полиномиальное время. В данной ситуации (как и во многих других примерах, касающихся NP-трудных задач) для решения конкретных примеров рассматриваемых проблем используются некоторые комбинаторные задачи с хорошо развитой алгоритмической базой.
Одной из таких является задача булевой выполнимости (Boolean Satisfiability Problem \--- SAT), для решения которой за последние 20~лет разработаны весьма эффективные на практике эвристические алгоритмы, применяемые для решения задач символьной верификации~\autocite{kroening2021}, компьютерной безопасности и криптографии~\autocite{bard2009}, построению расписаний и планированию~\autocite{prestwich2021} и~многим другим прикладным областям.
В применении к перечисленным задачам программные реализации алгоритмов решения SAT (так называемые SAT-решатели) дают мощные вычислительные инструменты, позволяющие решать частные случаи рассматриваемых задач таких размерностей, перед которыми другие подходы оказываются бессильны.

Таким образом, актуальной является проблема разработки алгоритмов и программных комплексов на основе алгоритмов решения SAT, используемых для решения задач верификации и синтеза формальных моделей конечно-автоматных программ \--- конечных автоматов и булевых схем.
При решении поставленной задачи возникает целый ряд новых проблем, основной из которых является отсутствие априорных оценок времени работы SAT-решателя на трудной формуле, кодирующей рассматриваемую задачу.
Грубо говоря, решатель, получив на вход формулу, может работать час, два, неделю, месяц или даже больше, и нет общего способа определить, сколько времени ему потребуется для завершения работы, притом что формально данный алгоритм является полным и на любой формуле завершает свою работу за конечное время.
Описанный феномен известен как \enquote{\textit{heavy-tailed behavior phenomenon}}~\autocite{gomes2009} (HTB).
В~рамках настоящей диссертации для борьбы с HTB используются специальные декомпозиционные представления булевых формул, кодирующих описания рассматриваемых моделей.
С использованием разработанных алгоритмов удалось решить ряд экстремально сложных задач, относящихся к верификации и синтезу конкретных примеров моделей автоматных программ.

Учитывая все сказанное выше, сформулируем основные цели и задачи работы.


% Цель работы.
\aim
%
Целью настоящей диссертации является повышение эффективности (снижение времени работы) полных алгоритмов решения задачи булевой выполнимости (SAT) применительно к синтезу и верификации моделей автоматных программ за счет разработки оригинальных методов и техник декомпозиции булевых формул.


% Задачи работы.
\tasks
%
Для достижения поставленной цели были решены следующие научно-технические задачи:
\begin{enumerate}[beginpenalty=10000]
    \item Разработаны методы кодирования в SAT задач синтеза конечно-автоматных моделей с заданным поведением и свойствами, отличающиеся от существующих добавлением кодирования структуры охранных условий в виде деревьев разбора соответствующих формул.
    \item Разработаны методы кодирования в SAT задачи синтеза модульных конечно-автоматных моделей с заданным поведением и свойствами, отличающиеся от существующих автоматизированным модульным разбиением.
    \item Разработаны методы кодирования в SAT задачи синтеза булевых схем и булевых формул по заданной таблице истинности, отличающиеся от существующих возможностью использования произвольных элементарных гейтов.
    \item Разработаны методы декомпозиции булевых формул, кодирующих задачи синтеза конечно-автоматных моделей и верификации булевых схем, отличающиеся от существующих возможностью построения оценок декомпозиционной трудности.
    \item С применением разработанных методов решены трудные примеры синтеза и верификации моделей автоматных программ \--- конечных автоматов и логических схем.
    \item Разработана программная библиотека kotlin-satlib, обеспечивающая взаимодействие с SAT-решателями через программный интерфейс, контроль за различными этапами построения SAT кодировок, а также предоставляющая возможности манипуляции переменными с произвольными конечными доменами.
    \item Разработан программный комплекс \smallcaps{fbSAT} для синтеза и верификации конечно-автоматных моделей с помощью SAT-решателей.
    \item Проведены масштабные вычислительные эксперименты для подтверждения практической эффективности всех разработанных методов.
\end{enumerate}


% Научная новизна.
\novelty
%
Новыми являются все основные результаты, полученные в диссертации, в том числе:
\begin{enumerate}[beginpenalty=10000]
    \item Алгоритмы синтеза булевых формул и схем, основанные на сведении к проблеме выполнимости (SAT) и использующие инкрементальные SAT-решатели.

    \item Алгоритмы синтеза монолитных и модульных конечно-автоматных моделей, основанные на сведении к проблеме выполнимости (SAT) и содержащие явное кодирование структуры охранных условий в виде деревьев разбора соответствующих формул.

    \item Алгоритмы декомпозиции и методы оценивания декомпозиционной трудности булевых формул, кодирующих задачи синтеза и верификации моделей автоматных программ.

    \item Решение экстремально трудных задач синтеза моделей автоматных программ при помощи разработанных алгоритмов.

    \item Программная библиотека kotlin-satlib для взаимодействия с SAT-решателями и контроля за процессом построения SAT кодировок.

    \item Программный комплекс \smallcaps{fbSAT} для синтеза и верификации конечно-автоматных моделей с помощью SAT-решателей.
\end{enumerate}


% Основные положения, выносимые на защиту.
\defpositions
%
\begin{enumerate}[beginpenalty=10000]
    \item Методы декомпозиции булевых формул, применяемые к задачам верификации моделей автоматных программ и отличающиеся от существующих подходов тем, что позволяют строить оценки декомпозиционной трудности рассматриваемых формул.
    % , с целью оценки верхней границы времени решения задач,

    \item Метод синтеза минимальных представлений булевых функций в виде формул и схем, отличающийся от существующих подходов тем, что основан на сведении к задаче выполнимости (SAT) и использовании инкрементальных SAT-решателей, за счёт чего достигается более высокая эффективность.
    % , с целью достижения более высокой эффективности,

    \item Методы синтеза и верификации монолитных и модульных конечно\-/автоматных моделей по примерам поведения и формальной спецификации, отличающиеся от существующих подходов тем, что основаны на сведении к задаче выполнимости (SAT) и использовании контрпримеров (Counterexample\-/Guided Inductive Synthesis \--- CEGIS), а также техникой явного кодирования структуры охранных условий.
    % , с целью достижения более высокой эффективности,

    \item Программная библиотека kotlin-satlib\footnote{\url{https://github.com/Lipen/kotlin-satlib}} для взаимодействия с SAT\-/решателями через программный интерфейс, отличающаяся тем, что с целью расширения области применимости, предоставляет широкий выбор различных SAT-решателей, обеспечивает контроль над всеми этапами построения SAT кодировок, а также обладает возможностью манипулировать переменными с произвольными конечными доменами.

    \item Программный комплекс \smallcaps{fbSAT}\footnote{\url{https://github.com/ctlab/fbSAT}} для синтеза и верификации конечно-автоматных моделей с помощью SAT-решателей, включающий в себя реализацию всех предложенных методов.
\end{enumerate}

% \item Методы проектирования и синтеза \emph{модульных} конечно-автоматных моделей, основанные на сведения к задаче выполнимости (SAT) и отличающиеся автоматизированным модульным разбиением, что позволяет существенно упростить процесс синтеза моделей.
% \item Методы синтеза \emph{модульных} конечно-автоматных моделей по примерам поведения и формальной спецификации, основанные на использовании контрпримеров (\textit{CounterExample-Guided Inductive Synthesis} \--- CEGIS), получаемых при верификации этих моделей, что позволяет производить синтез системы, удовлетворяющей спецификации, \emph{целиком}, а не по отдельным модулям.


% Теоретическая и практическая значимость работы.
\influence
%
Теоретическая значимость диссертации заключается в разработанных в ней концепциях и алгоритмах решения задач синтеза моделей автоматных программ и методах построения оценок декомпозиционной трудности таких задач.
Практическая значимость диссертации состоит в том, что основные разработанные в ней алгоритмы применимы к индустриальным задачам проектирования, синтеза и верификации моделей автоматных программ, а также в том, что на целом ряде конкретных примеров практическая реализация и апробация разработанных алгоритмов демонстрируют лучшую эффективность (меньшее время решения) в сравнении с известными подходами.


% Методы и инструменты исследования.
\methods
%
Теоретическая часть работы использует методологию дискретной математики, математической логики и теории вычислительной сложности.
Для синтеза конечно-автоматных моделей был использован программный комплекс \smallcaps{fbSAT}, разработанный в рамках данной диссертации.
Для верификации конечно-автоматных моделей использовался символьный верификатор \smallcaps{NuSMV}\footnote{\url{https://nusmv.fbk.eu}}.
При построении вычислительных задач из области проверки логической эквивалентности схем использовалась программная система Transalg\footnote{\url{https://gitlab.com/transalg/transalg}}.
Для решения конкретных инстансов задачи SAT использовались различные современные SAT-решатели, находящиеся в открытом доступе, такие как MiniSAT\footnote{\url{https://github.com/niklasso/minisat}}, Glucose\footnote{\url{https://github.com/audemard/glucose}}, Kissat\footnote{\url{https://github.com/arminbiere/kissat}}, CaDiCaL\footnote{\url{https://github.com/arminbiere/cadical}}.
Для взаимодействия с SAT-решателями через программный интерфейс использовалась программная библиотека kotlin-satlib, разработанная в рамках данной диссертации.
В~вычислительных экспериментах задействовался вычислительный кластер.


% Соответствие специальности.
\relevance
%
Содержание диссертации соответствует паспорту специальности 2.3.5 \enquote{Математическое и программное обеспечение вычислительных систем, комплексов и компьютерных сетей} в перечисленных ниже пунктах 1 (\enquote{Модели, методы и алгоритмы проектирования, анализа, трансформации, верификации и тестирования программ и программных систем}) и 3 (\enquote{Модели, методы, архитектуры, алгоритмы, языки и программные инструменты организации взаимодействия программ и программных систем}):
\begin{itemize}[beginpenalty=10000]
    \item в диссертации представлено семейство методов и алгоритмов, применяемых для трансформации автоматных программ в булевы схемы и формулы с целью вычислительного решения задач синтеза и верификации автоматных программ (пункт~1);
    \item представлены алгоритмы решения задач синтеза, верификации и тестирования моделей автоматных программ при помощи декомпозиционных представлений булевых формул, кодирующих исходные задачи (пункт~1);
    \item разработанная программная библиотека kotlin-satlib обеспечивает взаимодействие между алгоритмами кодирования в SAT задач синтеза и верификации автоматных программ и современными эффективными SAT-решателями (пункт~3).
\end{itemize}


% Достоверность научных достижений.
{\reliability}
%
Достоверность полученных в работе результатов обеспечивается теоретической корректностью предложенных алгоритмов, эффективность которых обоснована масштабными вычислительными экспериментами.


% Апробация работы.
\probation
%
Основные результаты диссертации докладывались на следующих конференциях:
\begin{itemize}[beginpenalty=10000]
    \item VIII Конгресс молодых ученых, Университет ИТМО, Санкт-Петербург, 2019.
    \item Конференция СПИСОК-2019, СПбГУ, Санкт-Петербург, 2019.
    \item IX Конгресс молодых ученых, Университет ИТМО, Санкт-Петербург, 2020.
    \item Конференция ППС 2021, Университет ИТМО, Санкт-Петербург, 2021.
    \item X Конгресс молодых ученых, Университет ИТМО, Санкт-Петербург, 2021.
    \item Воркшоп SAT/SMT Solvers: Theory and Practice, Санкт-Петербург, 2021.
    \item XI Конгресс молодых ученых, Университет ИТМО, Санкт-Петербург, 2022.
\end{itemize}

Диссертационная работа была выполнена при поддержке грантов и проектов:
\begin{itemize}[beginpenalty=10000]
    \item Грант РФФИ №19-07-01195 А «Разработка методов машинного
    обучения на основе SAT-решателей для синтеза модульных логических контроллеров киберфизических систем».
    \item Грант №19-37-51066 Научное наставничество «Разработка методов синтеза конечно-автоматных алгоритмов управления для программируемых логических контроллеров в распределенных киберфизических системах».
    \item НИР-ФУНД 77051 «Исследование алгоритма тестирования на основе обучения и улучшения его эффективности», 2020-2021.
    \item НИР-ПРИКЛ 222004 «Алгоритмы решения SAT для логических схем и анализа программ», 2021-2023.
    \item НИР-ПРИКЛ 223099 «Алгоритмы решения SAT для логических схем и анализа программ», 2023-2024.
\end{itemize}


% Публикации
\publications
%
% \theAllMyPapers
% \theScopusPapers
%
Основные результаты по теме диссертации изложены в 10~публикациях.
Из них одна издана в журналах, рекомендованных ВАК,
а три \--- в изданиях, индексируемых в базе цитирования Scopus.
% Также имеется 1 свидетельство о государственной регистрации программ для ЭВМ.

% В международных изданиях, индексируемых в базе данных Scopus:
% \begin{refsection}[biblio/own.bib]
% \nocite{*}
% \printbibliography[
%     keyword=scopus,
%     % title={В международных изданиях, индексируемых в базе данных Scopus},
%     % heading=subbibliography,
%     heading=none,
%     resetnumbers=true
% ]
% \end{refsection}

% В международных изданиях, индексируемых в базе данных Web of Science:
% \begin{refsection}[biblio/own.bib]
% \nocite{*}
% \printbibliography[
%     keyword=wos,
%     %title={В международных изданиях, индексируемых в базе данных Web of Science},
%     %heading=subbibliography,
%     heading=none,
%     resetnumbers=true
% ]
% \end{refsection}

% Список всех публикаций автора по теме диссертации:
% \begin{refsection}[biblio/own.bib]
% \nocite{*}
% \printbibliography[
%     keyword=own,
%     %title={Список всех публикаций автора по теме диссертации},
%     %heading=subbibliography,
%     heading=none,
%     resetnumbers=true
% ]
% \end{refsection}


% Личный вклад автора.
%
\contribution
%
Методы синтеза монолитных и модульных конечно-автоматных моделей, основанные на сведении к задаче SAT, разработаны соискателем в соавторстве с Чивилихиным~Д.\,С.
% TODO: модульные автоматы -- лично соискателем
% TODO: distributed CEGIS -- with Суворов~Д.\,М.
% TODO: программный комплекс fbSAT --- лично соискателем
Метод синтеза булевых формул и схем по заданной таблице истинности, основанный на сведении к задаче SAT, предложен и разработан лично соискателем.
Программная библиотека kotlin-satlib для взаимодействия с SAT-решателями через программный интерфейс разработана в виде прототипа в соавторстве с Гречишкиной~Д.\,С.
Дальнейшая разработка и расширение программной библиотеки производилась лично соискателем.
Общая стратегия оценивания трудности формул, кодирующих проверку эквивалентности булевых схем, разработана соискателем в соавторстве с Семёновым~А.\,А., Кондратьевым~В.\,С., Кочемазовым~С.\,Е. и Тарасовым~Е.\,А.
Описанные конструкции декомпозиций формул, кодирующих эквивалентность булевых формул, предложены лично соискателем.
Теоретическое обоснование корректности предложенных конструкций выполнены соискателем в соавторстве с Семёновым~А.\,А.
