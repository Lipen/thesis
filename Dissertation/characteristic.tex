%% Актуальность.
\pdfbookmark[1]{Актуальность}{actuality}%
%
\paragraph{Актуальность темы.}
%
В современном информационном обществе автоматизированные системы стали неотъемлемой частью различных областей науки и техники, что подчеркивает важность проблемы эффективной верификации и синтеза автоматных программ.
Под \textit{автоматизированными системами} понимается широкий класс объектов, решающих вычислительные задачи путем выполнения конкретных функций.
Они находят применение в таких разнообразных областях, как программирование, инженерия, робототехника, управление производственными процессами и многое другое.
Для анализа и разработки таких систем используются \textit{абстрактные модели}, которые позволяют формализовать их поведение и свойства.

Одним из ключевых понятий в этой области является понятие \textit{состояния} вычисляющей модели, которое было впервые введено Аланом Тьюрингом~\autocite{turing1937}.
Это понятие является основой для построения абстрактных моделей и играет важную роль в различных прикладных областях, таких как разработка языков программирования, трансляторов, компиляторов, микроконтроллеров и микропроцессоров.

Концепция \enquote{\textit{автоматного программирования}}~\autocite{polikarpova2009}, предлагает подход к построению программ, основанный на конечно-автоматной парадигме.
Этот подход предполагает разбиение программы на более простые модули, которые могут рассматриваться как отдельные вычислительные единицы, находящиеся в различных состояниях.
При этом задачи верификации и синтеза для автоматных программ сводятся к аналогичным задачам для формальных моделей.

В~настоящей диссертации рассматриваются два класса таких моделей: конечные автоматы и булевы схемы.
Несмотря на различия между этими моделями, между ними существует тесная взаимосвязь: конечные автоматы описывают функции, которые могут принимать входные данные произвольной длины, в то время как булевы схемы оперируют данными конкретной длины.
Соответственно, каждой функции, задаваемой конечным автоматом, соответствует счетное число функций, задаваемых булевыми схемами.
Для решения конкретных вычислительных задач приходится рассматривать конечные входные данные и, таким образом, переходить к булевым схемам.

Задачи верификации и синтеза для конечных автоматов и булевых схем являются вычислительно сложными и относятся к классу NP\=/трудных задач.
Это означает, что они не могут быть решены известными алгоритмами за полиномиальное время.
В~таких случаях, как и во многих других ситуациях, касающихся NP\=/трудных задач, для решения конкретных экземпляров рассматриваемых проблем используются комбинаторные задачи с хорошо развитой алгоритмической базой.
Одной из таких является задача булевой выполнимости (Boolean Satisfiability Problem \--- SAT), для решения которой за последние 20~лет разработаны весьма эффективные на практике эвристические алгоритмы, применяемые для решения задач символьной верификации~\autocite{kroening2021}, компьютерной безопасности и криптографии~\autocite{bard2009}, построению расписаний и планированию~\autocite{prestwich2021} и~многим другим прикладным областям.
В~применении к перечисленным задачам программные реализации алгоритмов решения SAT \--- так называемые SAT-решатели \--- дают мощные вычислительные инструменты, позволяющие решать частные случаи рассматриваемых задач таких размерностей, перед которыми другие подходы оказываются бессильны.

Таким образом, актуальной является проблема разработки алгоритмов и программных комплексов, основанных на решении задачи SAT, для верификации и синтеза формальных моделей автоматных программ, таких как конечные автоматы и булевые схемы.
Одной из ключевых проблем при решении этой задачи является отсутствие априорных оценок времени работы SAT-решателя на сложных формулах, кодирующих рассматриваемые задачи.
Грубо говоря, решатель, получив на вход формулу, может работать час, неделю, месяц или даже больше, и нет общего способа определить, сколько времени ему потребуется для завершения работы, притом что формально данный алгоритм является полным и на любой формуле завершает свою работу за конечное время.
Это явление известно как \textit{heavy-tailed behavior phenomenon} (HTB)~\autocite{gomes2009}, при котором время работы SAT-решателя на некоторых формулах может быть непредсказуемо длинным.
В~рамках данной диссертации для борьбы с явлением HTB предлагаются специальные декомпозиционные представления булевых формул.
Разработанные алгоритмы и методы показали высокую эффективность при решении сложных задач, связанных с синтезом и верификацией конкретных примеров автоматных программ.

Учитывая всё сказанное выше, можно утверждать, что разработка новых методов декомпозиции задачи булевой выполнимости (SAT) для синтеза и верификации моделей автоматных программ является актуальной и важной задачей, имеющей значительные теоретические и практические приложения.


\needspace{4\baselineskip}

%% Цель работы.
\pdfbookmark[1]{Цель работы}{aim}%
%
\paragraph{Цель работы.}
%
Целью настоящей диссертации является повышение эффективности (снижение времени работы) полных алгоритмов решения задачи булевой выполнимости (SAT) применительно к синтезу и верификации моделей автоматных программ за счет разработки оригинальных методов и техник декомпозиции булевых формул.


%% Задачи работы.
\pdfbookmark[1]{Задачи работы}{tasks}%
%
\paragraph{Задачи работы.}
%
Для достижения поставленной цели были решены следующие научно-технические задачи:
\begin{enumerate}[beginpenalty=10000]
    \item Разработка методов кодирования в SAT задач синтеза конечно-автоматных моделей с заданным поведением и свойствами. Новые методы включают в себя кодирование структуры охранных условий в виде деревьев разбора соответствующих формул, что отличает их от существующих решений.
    \item Разработка методов кодирования в SAT задач синтеза модульных конечно-автоматных моделей. Эти методы включают автоматизированное модульное разбиение, что улучшает их адаптивность и эффективность.
    \item Создание методов кодирования в SAT задач синтеза булевых схем и булевых формул по заданной таблице истинности. В~отличие от существующих методов, новые подходы позволяют использовать произвольные элементарные гейты, что расширяет их применение.
    \item Разработка методов декомпозиции булевых формул, кодирующих задачи синтеза конечно-автоматных моделей и верификации булевых схем. Новые методы позволяют строить оценки декомпозиционной трудности, что улучшает прогнозируемость времени работы SAT-решателей.
    \item Разработка программной библиотеки \texttt{kotlin-satlib}, обеспечивающей взаимодействие с SAT-решателями через программный интерфейс. Библиотека предоставляет широкий выбор различных SAT-решателей, контроль за различными этапами построения SAT-кодировок и возможность манипуляции переменными с произвольными конечными доменами и массивами переменных.
    \item Создание программного комплекса \smallcaps{fbSAT} для синтеза и верификации конечно-автоматных моделей с использованием SAT-решателей. Этот комплекс интегрирует разработанные методы и алгоритмы, предоставляя удобный инструмент для практического применения.
    \item Проведение вычислительных экспериментов для подтверждения эффективности разработанных методов.
\end{enumerate}


%% Основные положения, выносимые на защиту.
\pdfbookmark[1]{Основные положения, выносимые на защиту}{defpositions}%
%
\paragraph{Основные положения, выносимые на защиту.}
%
% Название
% Ограничительная часть (совпадение)
% Цель
% Отличительная часть (новизна)
%
\begin{enumerate}[beginpenalty=10000]
    \item Методы декомпозиции булевых формул, применяемые к задачам тестирования и верификации моделей автоматных программ и использующие SAT-решатели, отличающиеся от известных методов тем, что, с целью получения более точных верхних оценок трудности формул, в предлагаемых методах используются специальные конструкции SAT-разбиений.

    \item Метод синтеза минимальных представлений булевых функций в виде формул и схем, использующий сведение к задаче выполнимости (SAT), отличающийся от существующих подходов тем, что, с целью достижения более высокой эффективности (относительно времени и точности решения), предлагаемый метод использует инкрементальные SAT-решатели.

    \item Методы синтеза и верификации монолитных и модульных конечно-автоматных моделей по примерам поведения и формальной спецификации, использующие сведения к задаче выполнимости (SAT) и контрпримеры (Counterexample-Guided Inductive Synthesis \--- CEGIS), отличающиеся от существующих подходов тем, что, с целью повышения эффективности (относительно времени решения), применяется техника явного кодирования структуры охранных условий.

    \item Программная библиотека \texttt{kotlin-satlib}\footnote{\url{https://github.com/Lipen/kotlin-satlib}} для взаимодействия с SAT\-/решателями через унифицированный программный интерфейс и обеспечения контроля над всеми этапами построения SAT-кодировок, отличающаяся от известных библиотек тем, что, с целью расширения области применимости, разработанная библиотека предоставляет широкий выбор различных SAT-решателей и возможность манипулировать переменными с произвольными конечными доменами.

    \item Программный комплекс \smallcaps{fbSAT}\footnote{\url{https://github.com/ctlab/fbSAT}} для синтеза и верификации конечно-автоматных моделей с помощью SAT-решателей, отличающийся от известных тем, что с целью расширения функциональных возможностей, в него включены реализации всех предложенных методов, а для построения SAT-кодировок и взаимодействия с SAT-решателями используется библиотека \texttt{kotlin-satlib}.
\end{enumerate}


%% Научная новизна.
\pdfbookmark[1]{Научная новизна}{novelty}%
%
\paragraph{Научная новизна.}
%
Новыми являются все основные результаты, полученные в диссертации, в том числе:
\begin{enumerate}[beginpenalty=10000]
    \item Методы декомпозиции булевых формул, применяемые к задачам тестирования и верификации моделей автоматных программ с использованием SAT-решателей. Отличие от известных методов заключается в применении специальных конструкций SAT-разбиений, что позволяет получать более точные верхние оценки трудности формул.

    \item Метод синтеза минимальных представлений булевых функций в виде формул и схем, основанный на сведении к задаче выполнимости (SAT). В~отличие от существующих подходов, предлагаемый метод использует инкрементальные SAT-решатели, что позволяет достичь более высокой эффективности по времени и точности решения.

    \item Методы синтеза и верификации монолитных и модульных конечно-авто\-мат\-ных моделей, разработанные на основе сведений к задаче выполнимости~(SAT) и использования контрпримеров (Counter\-example-Guided Inductive Synthesis \--- CEGIS). Отличие состоит в применении техники явного кодирования структуры охранных условий, что значительно повышает эффективность по времени решения.

    \item Программная библиотека \texttt{kotlin-satlib} для взаимодействия с SAT-решателями и обеспечения контроля над всеми этапами построения SAT-кодировок. В~отличие от существующих библиотек, разработанная библиотека предоставляет широкий выбор различных SAT-решателей и возможность манипулировать переменными с произвольными конечными доменами.

    \item Программный комплекс \smallcaps{fbSAT} для синтеза и верификации конечно-автоматных моделей с помощью SAT-решателей, включающий реализацию всех предложенных методов. Этот комплекс позволяет эффективно решать экстремально трудные задачи синтеза моделей автоматных программ.
\end{enumerate}


%% Соответствие специальности.
\pdfbookmark[1]{Соответствие специальности}{relevance}%
%
\paragraph{Соответствие специальности.}
%
Содержание диссертации соответствует паспорту специальности 2.3.5 \enquote{Математическое и программное обеспечение вычислительных систем, комплексов и компьютерных сетей} в пунктах \enquote{1.~Модели, методы и алгоритмы проектирования, анализа, трансформации, верификации и тестирования программ и программных систем} и \enquote{3.~Модели, методы, архитектуры, алгоритмы, языки и программные инструменты организации взаимодействия программ и программных систем} в следующих частях:
\begin{itemize}[beginpenalty=10000]
    \item в диссертации представлено семейство методов и алгоритмов, применяемых для трансформации автоматных программ в булевы схемы и формулы с целью вычислительного решения задач синтеза и верификации автоматных программ (пункт~1);
    \item представлены алгоритмы решения задач синтеза, верификации и тестирования моделей автоматных программ при помощи декомпозиционных представлений булевых формул, кодирующих исходные задачи (пункт~1);
    \item разработанная программная библиотека \texttt{kotlin-satlib} обеспечивает взаимодействие между алгоритмами кодирования в SAT задач синтеза и верификации автоматных программ и современными эффективными SAT-решателями (пункт~3).
\end{itemize}


%% Теоретическая значимость работы.
\pdfbookmark[1]{Теоретическая значимость}{influence}%
%
\paragraph{Теоретическая значимость} диссертации заключается в разработке новых методов и алгоритмов для синтеза и верификации моделей автоматных программ.
В~работе предложены инновационные подходы к декомпозиции булевых формул и построению оценок их декомпозиционной трудности, что расширяет существующие теоретические основы в области применения SAT-решателей и предоставляет более точные инструменты для анализа сложных задач.

%% Практическая значимость работы.
\pdfbookmark[1]{Практическая значимость}{influence}%
%
\paragraph{Практическая значимость} работы проявляется в разработке программной библиотеки \texttt{kotlin-satlib} и программного комплекса \smallcaps{fbSAT}, которые позволяют эффективно применять новые методы и алгоритмы к реальным задачам проектирования и верификации программного обеспечения.
Эти инструменты демонстрируют высокую эффективность и могут быть интегрированы в существующие программные системы, что подтверждается лучшими результатами по сравнению с известными подходами и инструментами.


%% Методы и инструменты исследования.
\pdfbookmark[1]{Методы и инструменты исследования}{methods}%
%
\paragraph{Методы и инструменты исследования.}
%
Теоретическая часть работы основана на методологии дискретной математики, математической логики и теории вычислительной сложности.
Для синтеза конечно-автоматных моделей применялся программный комплекс \smallcaps{fbSAT}, разработанный в рамках данной диссертации.
Верификация этих моделей осуществлялась с помощью символьного верификатора \smallcaps{NuSMV}\footnote{\url{https://nusmv.fbk.eu}}.
При построении вычислительных задач из области проверки логической эквивалентности схем использовалась программная система Transalg\footnote{\url{https://gitlab.com/transalg/transalg}}.
Для решения экземпляров задачи SAT применялись различные современные SAT-решатели, такие как MiniSAT\footnote{\url{https://github.com/niklasso/minisat}}, Glucose\footnote{\url{https://github.com/audemard/glucose}}, Kissat\footnote{\url{https://github.com/arminbiere/kissat}}, CaDiCaL\footnote{\url{https://github.com/arminbiere/cadical}}.
Взаимодействие с SAT-решателями осуществлялось через программную библиотеку \texttt{kotlin-satlib}, разработанную в рамках данной диссертации.
Вычислительные эксперименты проводились с использованием вычислительного кластера.


%% Достоверность научных достижений.
\pdfbookmark[1]{Достоверность}{reliability}%
%
\paragraph{Достоверность научных достижений} диссертации подтверждается обоснованностью постановок задач, теоретической корректностью предложенных алгоритмов, а также результатами масштабных вычислительных экспериментов, проведенных для проверки и демонстрации эффективности разработанных методов.


%% Апробация работы.
\pdfbookmark[1]{Апробация работы}{approval}%
%
\paragraph{Апробация работы.}
%
Основные результаты диссертации докладывались на следующих конференциях:
\begin{itemize}[beginpenalty=10000]
    \item VIII Конгресс молодых ученых, Университет ИТМО, Санкт-Петербург, 2019.
    \item Конференция СПИСОК-2019, СПбГУ, Санкт-Петербург, 2019.
    \item IX Конгресс молодых ученых, Университет ИТМО, Санкт-Петербург, 2020.
    \item Конференция ППС 2021, Университет ИТМО, Санкт-Петербург, 2021.
    \item X Конгресс молодых ученых, Университет ИТМО, Санкт-Петербург, 2021.
    \item Воркшоп SAT/SMT Solvers: Theory and Practice, Санкт-Петербург, 2021.
    \item XI Конгресс молодых ученых, Университет ИТМО, Санкт-Петербург, 2022.
    \item Конференция MIPRO 2024, Опатия, Хорватия, 2024.
\end{itemize}
%
Диссертационная работа была выполнена при поддержке грантов и проектов:
\begin{itemize}[beginpenalty=10000]
    \item Грант РФФИ №19-07-01195 А «Разработка методов машинного обучения на основе SAT-решателей для синтеза модульных логических контроллеров киберфизических систем».
    \item Грант №19-37-51066 Научное наставничество «Разработка методов синтеза конечно-автоматных алгоритмов управления для программируемых логических контроллеров в распределенных киберфизических системах».
    \item НИР-ФУНД 77051 «Исследование алгоритма тестирования на основе обучения и улучшения его эффективности», 2020-2021.
    \item НИР-ПРИКЛ 222004 «Алгоритмы решения SAT для логических схем и анализа программ», 2021-2023.
    \item НИР-ПРИКЛ 223099 «Алгоритмы решения SAT для логических схем и анализа программ», 2023-2024.
\end{itemize}


% Личный вклад автора.
\pdfbookmark[1]{Личный вклад}{authorcontribution}%
%
\paragraph{Личный вклад автора.}
%
Методы синтеза монолитных и модульных конечно-автоматных моделей по примерам поведения и формальной спецификации, основанные на сведении к задаче SAT, разработаны соискателем в соавторстве с Чивилихиным~Д.\,С.
Методы синтеза и верификации модульных конечно-автоматных моделей по примерам поведения и формальной спецификации, основанные на сведении к задаче SAT и использовании контрпримеров, разработаны соискателем в соавторстве с Чивилихиным~Д.\,С. и Суворовым~Д.\,М.
Программный комплекс \smallcaps{fbSAT} разработан лично соискателем.
Реализация всех разработанных методов синтеза и верификации конечно-автоматных моделей в программном комплексе \smallcaps{fbSAT} выполнена лично соискателем.
Метод синтеза булевых формул и схем по заданной таблице истинности, основанный на сведении к задаче SAT, предложен и разработан лично соискателем.
Прототип программной библиотеки \texttt{kotlin-satlib} для взаимодействия с SAT-решателями через унифицированный программный интерфейс разработан в соавторстве с Гречишкиной~Д.\,С.
Дальнейшая разработка и расширение программной библиотеки \texttt{kotlin-satlib}, что включает в себя поддержку дополнительных SAT-решателей (например, Kissat) и разработку модуля для манипуляции переменными с конечными доменами, производилась лично соискателем.
Общая стратегия оценивания трудности формул, кодирующих проверку эквивалентности (задача верификации) булевых схем, разработана соискателем в соавторстве с Семёновым~А.\,А., Кондратьевым~В.\,С., Кочемазовым~С.\,Е. и Тарасовым~Е.\,А.
Описанные конструкции декомпозиций формул, кодирующих эквивалентность булевых формул, предложены лично соискателем.
Теоретическое обоснование корректности предложенных конструкций выполнены соискателем в соавторстве с Семёновым~А.\,А.


%% Публикации по теме диссертации.
\pdfbookmark[1]{Публикации}{publications}%
%
\paragraph{Публикации по теме диссертации.}
%
% \theAllMyPapers
% \theScopusPapers
%
Основные результаты по теме диссертации изложены в 11~публикациях.
Из них четыре изданы в изданиях, индексируемых в базе цитирования Scopus, и одна в журнале, рекомендованном ВАК.


\ifsynopsis
% In the synopsis-only document, the number of pages, figures, tables, etc is counted incorrectly.
\else
%% Структура и объем диссертации.
\pdfbookmark[1]{Структура и объём работы}{structure}%
%
\paragraph{Структура и объём работы.}
%
% Диссертация состоит из~введения,
% \formbytotal{totalchapter}{глав}{ы}{}{},
% заключения и
% \formbytotal{totalappendix}{приложен}{ия}{ий}{ий}.
% Полный объём диссертации составляет
% \formbytotal{TotPages}{страниц}{у}{ы}{}, включая
% \formbytotal{totalcount@figure}{рисун}{ок}{ка}{ков} и
% \formbytotal{totalcount@table}{таблиц}{у}{ы}{}.
% Список литературы содержит
% \formbytotal{citenum}{наименован}{ие}{ия}{ий}.
%
Диссертация состоит из введения, трёх глав и заключения.
Полный объём диссертации составляет \formbytotal{TotPages}{страниц}{у}{ы}{}, включая \formbytotal{totalcount@figure}{рисун}{ок}{ка}{ков} и \formbytotal{totalcount@table}{таблиц}{у}{ы}{}.
Список литературы содержит \formbytotal{citenum}{наименован}{ие}{ия}{ий}.
\fi
